\documentclass[11pt,english]{book}
\usepackage{babel}
\usepackage{shortvrb}
\usepackage[latin1]{inputenc}
\usepackage{tabularx}
\usepackage{longtable}
\setlength{\extrarowheight}{2pt}
\usepackage{amsmath}
\usepackage{graphicx}
\usepackage{color}
\usepackage{multirow}
\usepackage[colorlinks=true,linkcolor=blue,urlcolor=blue]{hyperref}
\usepackage[a4paper,margin=2cm]{geometry}
%% generator Docutils: http://docutils.sourceforge.net/
\newlength{\admonitionwidth}
\setlength{\admonitionwidth}{0.9\textwidth}
\newlength{\docinfowidth}
\setlength{\docinfowidth}{0.9\textwidth}
\newcommand{\optionlistlabel}[1]{\bf #1 \hfill}
\newenvironment{optionlist}[1]
{\begin{list}{}
  {\setlength{\labelwidth}{#1}
   \setlength{\rightmargin}{1cm}
   \setlength{\leftmargin}{\rightmargin}
   \addtolength{\leftmargin}{\labelwidth}
   \addtolength{\leftmargin}{\labelsep}
   \renewcommand{\makelabel}{\optionlistlabel}}
}{\end{list}}
% begin: floats for footnotes tweaking.
\setlength{\floatsep}{0.5em}
\setlength{\textfloatsep}{\fill}
\addtolength{\textfloatsep}{3em}
\renewcommand{\textfraction}{0.5}
\renewcommand{\topfraction}{0.5}
\renewcommand{\bottomfraction}{0.5}
\setcounter{totalnumber}{50}
\setcounter{topnumber}{50}
\setcounter{bottomnumber}{50}
% end floats for footnotes
% some commands, that could be overwritten in the style file.
\newcommand{\rubric}[1]{\section*{~\hfill {\it #1} \hfill ~}}
% end of "some commands"
% donot indent first line.
\setlength{\parindent}{0pt}
\setlength{\parskip}{5pt plus 2pt minus 1pt}

% sloppy
% ------
% Less strict (opposite to default fussy) space size between words. Therefore
% less hyphenation.
\sloppy

% fonts
% -----
% times for pdf generation, gives smaller pdf files.
%
% But in standard postscript fonts: courier and times/helvetica do not fit.
% Maybe use pslatex.
\usepackage{times} 

% pagestyle
\pagestyle{headings}

\title{OBJECT ORIENTED PROGRAMMING IN PYTHON}
\author{}
\date{}
\hypersetup{
pdftitle={OBJECT ORIENTED PROGRAMMING IN PYTHON},
pdfauthor={Michele Simionato}
}
\raggedbottom
\begin{document}
\maketitle

%___________________________________________________________________________
\begin{center}
\begin{tabularx}{\docinfowidth}{lX}
\textbf{Version}: &
	0.5 \\
\textbf{Author}: &
	Michele Simionato \\
\textbf{E-mail}: &
	mis6@pitt.edu \\
\textbf{Home-page}: &
	http://www.phyast.pitt.edu/~micheles/ \\
\textbf{Disclaimer}: &
	I release this book to the general public. 
It can be freely distributed if unchanged.
As usual, I don't give any warranty: while I have tried hard to ensure the
correctness of what follows, I disclaim any responsability in case of
errors . Use it at your own risk and peril ! \\
\end{tabularx}
\end{center}



\tableofcontents

\bigskip
\setcounter{chapter}{-1}

%___________________________________________________________________________

\hypertarget{preface}{}
\chapter{Preface}
\begin{quote}
\begin{flushleft}
\emph{There~is~only~one~way~to~learn:~trough~examples}
\end{flushleft}
\end{quote}


%___________________________________________________________________________

\hypertarget{the-philosophy-of-this-book}{}
\section{The philosophy of this book}

This book is written with the intent to help the programmer going trough
the fascinating concepts of Object Oriented Programming (OOP), in their
Python incarnation. Notice that I say to help, not to teach. Actually,
I do not think that a book can teach OOP or any other non-trivial matter 
in Computer Science or other disciplines. Only the
practice can teach: practice, then practice, and practice again. 
You must learn yourself from your experiments, not from the books. 
Nevertheless, books are useful. They cannot teach, but they can help. 
They should give you new ideas that you was not thinking about, they should
show tricks you do not find in the manual, and in general they should be of
some guidance in the uphill road to knowledge. That is the philosophy
of this book. For this reason

1. It is not comprehensive, not systematic; 
it is intended to give ideas and basis: from
that the reader is expected to cover the missing part on his own,
browsing the documentation, other sources and other books, and finally
the definite autority, the source itself.

2. It will not even try to teach the \emph{best} practices. I will show what you can
do with Python, not what you ``should'' do. Often I will show solutions that are
not recommended. I am not a mammy saying this is
good, this is bad, do this do that.

3. You can only learn from your failures. If you think ``it should work, if I do
X and Y'' and it works, then you have learned nothing new. 
You have merely verified
that your previous knowledge was correct, but you haven't create a new
knowledge. On the other hand, when you think ``it should work, if I do
X and Y'' and it doesn't, then you have learned that your previous knowlegde
was wrong or incomplete, and you are forced to learn something new to
overcome the difficulty. For this reason, I think it is useful to report
not only how to do something, but also to report how not to do something, 
showing the pitfalls of wrong approaches.

That's in my opinion is the goal of a good book. I don't know if have
reached this goal or not (the decision is up to the reader), but at least
I have tried to follow these guidelines.

Moreover, this is not a book on OOP, 
it is a book on OOP \emph{in Python}.

In other words, the point of view of this book is not 
to emphasize general topics of OOP that are exportable to other languages, 
but exactly the opposite: I want to emphasize specific techniques that one
can only use in Python, or that are difficult to translate to other 
languages. Moreover, I will not provide comparisons with other 
languages (except for the section ``Why Python?'' in this introduction and
in few selected other places), 
in order to keep the discussion focused.

This choice comes from the initial motivation for this book, which was 
to fulfill a gap in the (otherwise excellent) Python documentation. 
The problem is that the available documentation still lacks an accessible 
reference of the new Python 2.2+ object-oriented features.
Since myself I have learned Python and OOP from scratch, 
I have decided to write this book in order to fill that gap and
help others.

The emphasis in this book is not in giving 
solutions to specific problems (even if most of the recipes of this book
can easily be tailored to solve real life concrete problems), it is in 
teaching  how does it work, why it does work in some cases and why does 
not work in some other cases. Avoiding too specific problems has an
additional bonus, since it allows me to use \emph{short} examples (the majority 
of the scripts presented here is under 20-30 lines) which I think are 
best suited to teach a new matter [\hyperlink{id2}{1}] . Notice, however, that whereas
the majority of the scripts in this book are short, it is also true
that they are pretty \emph{dense}. The density is due to various reasons:
\newcounter{listcnt1}
\begin{list}{\arabic{listcnt1}.}
{
\usecounter{listcnt1}
\setlength{\rightmargin}{\leftmargin}
}
\item 
I am defining a lot of helper functions and classes, that are
reused and enhanced during all the book.

\item 
I am doing a strong use of inheritance, therefore a script at the
end of the book can inherits from the classes defined through all
the book;

\item 
A ten line script involving metaclasses can easily perform the equivalent 
of generating hundreds of lines of code in a language without metaclasses 
such as Java or C++.

\end{list}

To my knowledge, there are no other books covering the same topics with
the same focus (be warned, however, that I haven't read so many Python 
books ;-). The two references that come closest to the present book are
the \texttt{Python Cookbook} by Alex Martelli and David Ascher, and
Alex Martelli's \texttt{Python in a Nutshell}. They are quite recent books and 
therefore it covers (in much less detail) some of the 2.2 features that are 
the central topics to this book. 
However, the Cookbook reserves to OOP only one chapter and has a quite 
different philosophy from the present book, therefore there is 
practically no overlapping. Also \texttt{Python in a Nutshell} covers 
metaclasses in few pages, whereas half of this book is essentially
dedied to them. This means that you can read both ;-)
\begin{figure}[b]\hypertarget{id2}[1]
Readers that prefer the  opposite philosophy of using longer, 
real life-like, examples, have already the excellent ``Dive into 
Python'' book \href{http://diveintopython.org/}{http://diveintopython.org/} at their disposal. This is 
a very good book that I certainly recommend to any (experienced) 
Python programmer; it is also freely available (just like this ;-).
However, the choice of arguments is quite different and there is 
essentially no overlap between my book and ``Dive into Python'' 
(therefore you can read both ;-).
\end{figure}


%___________________________________________________________________________

\hypertarget{for-who-this-book-in-intended}{}
\section{For who this book in intended}

I have tried to make this tutorial useful to a large public of Pythonistas, 
i.e. both people with no previous experience of Object Oriented Programming
and people with experience on OOP, but unfamiliar with the most
recent Python 2.2-2.3 features (such as attribute descriptors,
metaclasses, change of the MRO in multiple inheritance, etc). 
However, this is not a book for beginners: the non-experienced reader should 
check (at least) the Internet sites www.python.org/newbies.com and 
www.awaretek.com, that provide a nice collection of resources for Python 
newbies.

These are my recommendations for the reader, according to her/his level:
\newcounter{listcnt2}
\begin{list}{\arabic{listcnt2}.}
{
\usecounter{listcnt2}
\setlength{\rightmargin}{\leftmargin}
}
\item 
If you are an absolute beginner, with no experience on programming,
this book is \emph{not} for you (yet ;-). Go to 
\href{http://www.python.org/doc/Newbies.html}{http://www.python.org/doc/Newbies.html} and read one of the introductive 
texts listed there, then come back here. I recommend ``How to Think Like 
a Computer Scientist'', available for free on the net (see 
\href{http://www.ibiblio.org/obp/thinkCSpy/}{http://www.ibiblio.org/obp/thinkCSpy/}); I found it useful myself when 
I started learning Python; be warned, however, that it refers to the rather 
old Python version 1.5.2. There are also excellent books 
on the market (see \href{http://www.awaretek.com/plf.html}{http://www.awaretek.com/plf.html}). 
\href{http://www.uselesspython.com/}{http://www.uselesspython.com/} is a good resource to find recensions 
about available Python books. For free books, look at
\href{http://www.tcfb.com/freetechbooks/bookphyton.html}{http://www.tcfb.com/freetechbooks/bookphyton.html} .
This is \emph{not} another Python tutorial.

\item 
If you know already (at least) another programming language, but you don't
know Python, then this book is \emph{not} for you (again ;-). Read the FAQ, the
Python Tutorial and play a little with the Standard Library (all this
material can be downloaded for free from  \href{http://www.python.org}{http://www.python.org}), then
come back here.

\item 
If you have passed steps 1 and 2, and you are confortable with Python
at the level of simple procedural programming, but have no clue about
objects and classes, \emph{then} this book is for you. Read this book till
the end and your knowledge of OOP will pass from zero to a quite advanced 
level (hopefully). Of course, you will have to play with the code in 
this book and write a lot of code on your own, first ;-)

\item 
If you are confortable with Python and you also known OOP from other
languages or from earlier version of Python, then this book is for
you, too: you are ready to read the more advanced chapters.

\item 
If you are a Python guru, then you should read the book, too. I expect
you will find the errors and send me feedback, helping me to improve
this tutorial.

\end{list}


%___________________________________________________________________________

\hypertarget{about-the-scripts-in-this-book}{}
\section{About the scripts in this book}

All the scripts  in this book are free. You are expected to play
with them, to modify them and to improve them.

In order to facilitate the extraction of the scripts from the main text, both
visually for the reader and automatically for Python, I use the
convention of sandwiching the body of the example scripts in blocks like this
\begin{quote}
\begin{ttfamily}\begin{flushleft}
\mbox{{\#}{$<$}myfirstscript.py{$>$}}\\
\mbox{}\\
\mbox{print~"Here~Starts~the~Python~Way~to~Object~Oriented~Programming~!"}\\
\mbox{}\\
\mbox{{\#}{$<$}/myfirstscript.py{$>$}}
\end{flushleft}\end{ttfamily}
\end{quote}

You may extract the source of this script with the a Python program
called ``test.py'' and provided in the distribution. Simply give the 
following command:
\begin{quote}
\begin{ttfamily}\begin{flushleft}
\mbox{{\$}~python~test.py~myfirstscript.py}
\end{flushleft}\end{ttfamily}
\end{quote}

This will create a file called ``myfirstscript.py'', containing the
source of \texttt{myfirstscript.py}; moreover it will execute the script 
and write its output in a file called ``output.txt''. I have tested
all the scripts in this tutorial under Red Hat Linux 7.x and 
Windows 98SE. You should not have any problem in running them,
but if a problem is there, ``test.py'' will probably discover it,
even if, unfortunately, it will not provide the solution :-(.
Notice that test.py requires Python 2.3+ to work, since most of
the examples in this book heavily depends on the new features
introduced in Python 2.2-2.3. Since the installation of Python 
2.3 is simple, quick and free, I think I am requiring to my readers
who haven't upgraded yet a very little effort. This is well worth
the pain since Python 2.3 fixes few bugs of 2.2 (notably in the subject of
attribute descriptors and the \texttt{super} built-in) that makes

You may give more arguments to test.py, as in this example:
\begin{quote}
\begin{ttfamily}\begin{flushleft}
\mbox{{\$}~python~test.py~myfirstscript.py~mysecondscript.py}
\end{flushleft}\end{ttfamily}
\end{quote}

The output of both scripts will still be placed in the file ``output.txt''.
Notice that if you give an argument which is not the name of a script in the
book, it will be simply ignored. Morever, if you will not give any argument,
``test.py'' will automatically executes all the tutorial scripts, writing their 
output in ``output.txt'' [\hyperlink{id4}{2}] . You may want to give a look at this file, once 
you have finished the tutorial. It also contains the source code of 
the scripts, for better readability.

Many examples of this tutorial depend on utility functions defined
in a external module called \texttt{oopp} (\texttt{oopp} is an obvious abbreviation 
for the title of the tutorial). The module \texttt{oopp} is automatically generated 
by ``test.py'', which works by extracting from the tutorial 
text blocks of code of the form \texttt{{\#}{$<$}oopp.py{$>$} something {\#}{$<$}/oopp.py{$>$}} 
and saving them in a file called ``oopp.py''. 
Let me give an example. A very recent enhancement to Python (in 
Python 2.3) has been the addition of a built-in boolean type with
values True and False:
\begin{quote}
\begin{ttfamily}\begin{flushleft}
\mbox{{\$}~python}\\
\mbox{Python~2.3a1~({\#}1,~Jan~~6~2003,~10:31:14)}\\
\mbox{[GCC~2.96~20000731~(Red~Hat~Linux~7.2~2.96-108.7.2)]~on~linux2}\\
\mbox{Type~"help",~"copyright",~"credits"~or~"license"~for~more~information.}\\
\mbox{{$>$}{$>$}{$>$}~1+1==2}\\
\mbox{True}\\
\mbox{{$>$}{$>$}{$>$}~1+1==3}\\
\mbox{False}\\
\mbox{{$>$}{$>$}{$>$}~type(True)}\\
\mbox{{$<$}type~'bool'{$>$}}\\
\mbox{{$>$}{$>$}{$>$}~type(False)}\\
\mbox{{$<$}type~'bool'{$>$}}
\end{flushleft}\end{ttfamily}
\end{quote}

However, previous version of Python use the integers 1 and 0 for 
True and False respectively.
\begin{quote}
\begin{ttfamily}\begin{flushleft}
\mbox{{\$}~python}\\
\mbox{Python~2.2~({\#}1,~Apr~12~2002,~15:29:57)}\\
\mbox{[GCC~2.96~20000731~(Red~Hat~Linux~7.2~2.96-109)]~on~linux2}\\
\mbox{Type~"help",~"copyright",~"credits"~or~"license"~for~more~information.}\\
\mbox{{$>$}{$>$}{$>$}~1+1==2}\\
\mbox{1}\\
\mbox{{$>$}{$>$}{$>$}~1+1==3~}\\
\mbox{0}
\end{flushleft}\end{ttfamily}
\end{quote}

Following the 2.3 convension, in this tutorial I will use the names 
\texttt{True} and \texttt{False} to denotes the numbers 1 and 0 respectively. 
This is automatic in Python 2.2.1+, but not in Python 2.2. Therefore, 
for sake of compatibility, it is convenient to set the values \texttt{True} 
and \texttt{False} in our utility module:
\begin{quote}
\begin{ttfamily}\begin{flushleft}
\mbox{{\#}{$<$}oopp.py{$>$}}\\
\mbox{}\\
\mbox{import~{\_}{\_}builtin{\_}{\_}}\\
\mbox{try:~}\\
\mbox{~~~~{\_}{\_}builtin{\_}{\_}.True~~~{\#}look~if~True~is~already~defined}\\
\mbox{except~AttributeError:~{\#}~if~not~add~True~and~False~to~the~builtins}\\
\mbox{~~~~{\_}{\_}builtin{\_}{\_}.True~=~1}\\
\mbox{~~~~{\_}{\_}builtin{\_}{\_}.False~=~0}\\
\mbox{}\\
\mbox{{\#}{$<$}/oopp.py{$>$}}
\end{flushleft}\end{ttfamily}
\end{quote}

Here there is an example of usage:
\begin{quote}
\begin{ttfamily}\begin{flushleft}
\mbox{{\#}{$<$}mysecondscript.py{$>$}}\\
\mbox{}\\
\mbox{import~oopp}\\
\mbox{print~"True~=",True,}\\
\mbox{print~"False~=",False}\\
\mbox{}\\
\mbox{{\#}{$<$}/mysecondscript.py{$>$}}
\end{flushleft}\end{ttfamily}
\end{quote}

The output is ``True = 1 False = 0'' under Python 2.2 and 
``True = True False = False'' under Python 2.3+.
\begin{figure}[b]\hypertarget{id4}[2]
``test.py'', invoked without arguments, does not create '.py' files, 
since I don't want to kludge the distribution with dozens of ten-line
scripts. I expect you may want to save only few scripts as standalone
programs, and cut and paste the others.
\end{figure}


%___________________________________________________________________________

\hypertarget{conventions-used-in-this-book}{}
\section{Conventions used in this book}

Python expressions are denoted with monospaced fonts when in the text.
Sections marked with an asterisk can be skipped in a first reading.
Typically they have the purpose of clarifying some subtle point and
are not needed for the rest of the book. These sections are intended
for the advanced reader, but could confuse the beginner.
An example is the section about the difference between methods and
functions, or the difference between the inheritance constraint and
the metaclass constraint.


%___________________________________________________________________________

\hypertarget{introduction}{}
\chapter{Introduction}
\begin{quote}
\begin{flushleft}
\emph{A~language~that~doesn't~affect~the~way~you~think~about~programming,~\\
is~not~worth~knowing.}~--~Alan~Perlis
\end{flushleft}
\end{quote}


%___________________________________________________________________________

\hypertarget{why-oop}{}
\section{Why OOP ?}

I guess some of my readers, like me, have started programming in the mid-80's,
when traditional (i.e. non object-oriented) Basic and Pascal where popular as 
first languages. At the time OOP was not as pervasive in software development 
how it is now, most of the mainstream languages were non-object-oriented and 
C++ was just being released. That was a time when the transition from 
spaghetti-code to structured code was already well accomplished, but 
the transition from structured programming to (the first phase of) 
OOP was at the beginning.

Nowaydays, we live in a similar time of transition . Today, the transition 
to (the first phase of) OOP is well accomplished and essentially all 
mainstream
languages support some elementary form of OOP. To be clear, when I say
mainstream langauges, I have in mind Java and C++: C is a remarkable 
exception to the  rule, since it is mainstream but not object-oriented.

However, both Java an C++ (I mean standard Java and C++, not special
extension like DTS C++, that have quite powerful object oriented features)
are quite poor object-oriented languages: they provides only the most 
elementary aspects of OOP, the features of the \emph{first phase} of OOP.

Hence, today the transition to the \emph{second phase} of OOP is only at the 
beginning, i.e mainstream language are not yet really OO, but they will
become OOP in the near future.

By second phase of OOP I mean the phase in which the primary
objects of concern for the programmer are no more the objects, but the
metaobjects. In elementary OOP one works on objects, which have attributes
and methods (the evolution of old-fashioned data and functions)  defined
by their classes; in the second phase of OOP one works on classes 
which behavior is described by metaclasses. We no more modify objects 
trough classes: nowadays we modify classes and class hierarchies 
through metaclasses and multiple inheritance.

It would be tempting to represent the history of programming in the last
quarter of century with an evolutionary table like that:

\begin{longtable}[c]{|p{0.28\linewidth}|p{0.23\linewidth}|p{0.26\linewidth}|p{0.08\linewidth}|}
\hline
\textbf{
{\~{ }}1975
} & \textbf{
{\~{ }}1985
} & \textbf{
{\~{ }}1995
} & \textbf{
{\~{ }}2005
} \\ \hline
\endhead
%[visit_tbody]

procedural programming
 & 
OOP1
 & 
OOP2
 & 
?
 \\ \hline

data,functions
 & 
objects,classes
 & 
classes,metaclasses
 & 
?
 \\ \hline
%[depart_tbody]
\end{longtable}

The problem is that table would be simply wrong, since in truth
Smalltalk had metaclasses already 25 years ago! And also Lisp
had \emph{in nuce} everything a long \emph{long} time ago.
The truth is that certains languages where too much ahead of their 
time ;-)

Therefore, today we already have all the ideas 
and the conceptual tools to go beyond the first phase of OOP 
(they where invented 20-30 years ago), nevertheless those ideas are  
not yet universally known, nor implemented in mainstream languages.

Fortunately, there are good languages
where you can access the bonus of the second phase of OOP (Smalltalk, CLOS,
Dylan, ...): unfortunately
most of them are academic and/or little known in the real world
(often for purely commercial reasons, since typically languages are not
chosen accordingly to their merits, helas!). Python is an exception to this
rule, in the sense that it is an eminently practical language (it started
as a scripting language to do Operating System administrative jobs), 
which is relatively known and used in that application niche (even if some
people \emph{wrongly} think that should not be used for 'serious' things).

There are various reasons why most mainstream languages are rather
poor languages, i.e. underfeatured languages (as Java) or powerful, but too
tricky to use, as C++. Some are good reasons (for instance \emph{efficiency}: if
efficiency is the first concern, then poor languages can be much
better suited to the goal: for instance Fortran for number crunching
and C for system programming), some are less good (economical 
monopoly). There is nothing to do against these reasons: if you
need efficiency, or if you are forced to use a proprietary language
because it is the language used by your employer. However, if you
are free from these restrictions, there is another reason why you
could not choose to use a poweful language. The reason is that, 
till now, programmers working in the industrial world mostly had simple 
problems (I mean conceptually simple problems). In order to solve
simple problems one does not need a powerful language, and the effort
spent in learning it is not worth.

However, nowadays the situations has changed. Now, with Internet and graphics
programming everywhere, and object-oriented languages so widespread,
now it is the time when actually people \emph{needs} metaprogramming, the
ability to changing classes and programs. Now everybody is programming
in the large.

In this situation, it is justified to spend some time to learn better
way of programming. And of course, it is convenient to start from
the language with the flattest learning curve of all.


%___________________________________________________________________________

\hypertarget{why-python}{}
\section{Why Python ?}
\begin{quote}
\begin{flushleft}
\emph{In~many~ways,~it's~a~dull~language,~borrowing~solid~old~concepts~from~~\\
many~other~languages~{\&}~styles:~~boring~syntax,~unsurprising~semantics,~\\
few~automatic~coercions,~etc~etc.~~But~that's~one~of~the~things~I~like~\\
about~it.}~~--Tim~Peters~on~Python,~16~Sep~93
\end{flushleft}
\end{quote}

If you are reading this book, I assume you already  have some experience
with Python. If this is the case, you already know the obvious advantages
of Python such as readability, easy of use and short development time.
Nevertheless, you could only have used Python as a fast and simple
scripting language. If you are in this situation, then your risk to
have an incorrect opinion on the language like ``it is a nice little
language, but too simple to be useful in 'real' applications''. The
truth is that Python is designed to be \emph{simple}, and actually it
is; but by no means it is a ``shallow'' language. Actually, it goes
quite \emph{deep}, but it takes some time to appreciate this fact.

Let me contrast Python with Lisp, for instance. From the beginning,
Lisp was intended to be a language for experts, for people with difficult 
problems to solve. The first
users of Lisp were academicians, professors of CS and scientists.
On the contrary, from the beginning Python 
was intended to be language for everybody (Python predecessor was ABC, 
a language invented to teach CS to children). Python makes great a first 
language for everybody, whereas Lisp would require especially
clever and motivated students (and we all know that there is lack
of them ;-)

From this difference of origins, Python inherits an easy to learn syntax,
whereas Lisp syntax is horrible for the beginner (even if not as
horrible as C++ syntax ;-)
\begin{quote}
\begin{flushleft}
\emph{Macros~are~a~powerful~extension~to~weak~languages.~\\
Powerful~languages~don't~need~macros~by~definition.}~~\\
--~Christian~Tismer~on~c.l.p.~(referring~to~C)
\end{flushleft}
\end{quote}

Despite the differences, Python borrows quite a lot from Lisp and it
is nearly as expressive as it (I say nearly since Python is 
not as powerful as Lisp: by tradition, Lisp has always been on the top of 
hierarchy of programming language with respect to power of abstraction).
It is true that Python lacks some powerful Lisp features: for instance 
Python object model lacks multiple dispatching (for the time being ;-) 
and the language lacks Lisp macros (but this unlikely to change in the 
near future since Pythonistas see the lack of macro as a Good Thing [\hyperlink{id6}{3}]): 
nevertheless, the point is that Python is much \emph{much} easier to learn. 
You have (nearly) all the power, but without the complexity.

One of the reasons, is that Python
try to be as \emph{less} innovative as
possible: it takes the proven good things from others, more innovative
languages, and avoids their pitfalls. If you are an experienced
programmer , it will be even  easier to you to learn Python, since
there is more or less nothing which is really original to Python.
For instance:
\newcounter{listcnt3}
\begin{list}{\arabic{listcnt3}.}
{
\usecounter{listcnt3}
\setlength{\rightmargin}{\leftmargin}
}
\item 
the object model is took from languages that are good at it, such
as Smalltalk;

\item 
multiple inheritance has been modeled from languages good in it. such
as CLOS and Dylan;

\item 
regular expression follows the road opened by Perl;

\item 
functional features are borrowed from functional languages;

\item 
the idea of documentation strings come from Lisp;

\item 
list comprehension come from Haskell;

\item 
iterators and generators come from Icon;

\item 
etc. etc. (many other points here)

\end{list}

I thinks the really distinctive feature of Python with respect to
any other serious language I know, is that Python is \emph{easy}. You have the 
power (I mean power in conceptual sense, not computational power: in
the sense of computational power the best languages are
non-object-oriented ones) 
of the most powerful languages with a very little investement.
In addition to that, Python has a relatively large user base 
(as compared to Smalltalk or Ruby, or the various fragmented Lisp
communities). Of course, 
there is quite a difference between the user base of Python with
respect to the user base of, let say, VisualBasic or Perl. But 
I would never take in consideration VisualBasic for anything serious, 
whereas Perl is too ugly for my taste ;-).  
Finally, Python is \emph{practical}. With this I mean the fact that 
Python has libraries that
allow the user to do nearly everything, since you can access all the C/C++ 
libraries with little or no effort, and all the Java libraries, though the
Python implementation known as Jython. In particular, one has the choice
between many excellent GUI's trough PyQt, wxPython, Tkinter, etc.

Python started as an Object Oriented Programming
Languages from the beginning, nevertheless is was never intended to be
a \emph{pure} OOPL as SmallTalk or, more recently, Ruby. Python is a 
\emph{multiparadigm}
language such a Lisp, that you choose your programming style according
to your problem: spaghetti-code, structured programming, functional
programming, object-oriented programming are all supported. You can
even write bad code in Python, even if it is less simple than in other
languages ;-). Python is a language which has quite evolved in its twelve
years of life (the first public release was released in February 1991)
and many new features have been integrated in the language with time. 
In particular, Python 2.2 (released in 2002) was a major breakthrough 
in the history of the language
for what concerns support to Object Oriented Programming (OOP). 
Before the 2.2 revolution, Python Object
Orientation was good; now it is \emph{excellent}. All the fundamental features
of OOP, including pretty sophisticated ones, as metaclasses and multiple
inheritance, have now a very good support (the only missing thing is
multiple dispatching).
\begin{figure}[b]\hypertarget{id6}[3]
Python lacks macros for an intentional design choice: many people
in the community (including Guido itself) feel that macros are 
``too powerful''. If you give the user the freedom to create her
own language, you must face at least three problems: i) the risk
to split the original language in dozens of different dialects;
ii) in collaborative projects, the individual programmer must 
spend an huge amount of time and effort would be spent in learning
macro systems written by others; iii) not all users are good
language designers: the programmer will have to fight with badly
designed macro systems. Due to these problems, it seems unlikely
that macros will be added to Python in the future.
\end{figure}
\begin{figure}[b]\hypertarget{id7}[4]
For a good comparison between Python and Lisp I remind the reader to
the excellent Peter Norvig's article in
\href{http://www.norvig.com/python-lisp.html}{http://www.norvig.com/python-lisp.html}
\end{figure}


%___________________________________________________________________________

\hypertarget{further-thoughts}{}
\section{Further thoughts}

Actually, the principal reasons why I begun studying 
Python was the documentation and the newsgroup: Python has an outstanding 
freely available documentation and an incredibly helpful newsgroup that
make extremely easy to learn the language. If I had found a comparable 
free documentation/newsgroup for C++ or Lisp, I would have studied that
languages instead.

Unfortunately, the enormous development at the software level, had no
correspondence with with an appropriate development of documentation.
As a consequence, the many beatiful, powerful and extremely \emph{useful}
new features of Python 2.2+ object orientation are mostly remained
confined to developers and power users: the average Python programmer
has remained a little a part from the rapid development and she
\emph{wrongly} thinks she has no use for the new features. There have
also been \emph{protestations} of the users against developers of the
kind ``please, stop adding thousands of complicated new extensions
to the language for which we have no use'' !

Extending a language is always a delicate thing to do, for a whole
bunch of reasons:
\newcounter{listcnt4}
\begin{list}{\arabic{listcnt4}.}
{
\usecounter{listcnt4}
\setlength{\rightmargin}{\leftmargin}
}
\item 
once one extension is done, it is there \emph{forever}.

\end{list}

My experience has been the following.

When I first read about metaclasses, in Guido's essay
``Unifying types and classes in Python 2.2'', I thought ``Wow,
classes of classes, cool concept, but how useful is it?  
Are metaclasses really providing some new functionality?  
What can I do with metaclasses that I cannot do without?''

Clearly, in these terms, the question is rather retorical, since in principle
any Turing-complete programming languages contains all the features provided 
by metaclasses. Python metaclasses themselves are implemented in C, that has 
no metaclasses. Therefore, my real question was not ``What can I do 
with metaclasses that I cannot do without?'' but ``How big is the convenience 
provided by metaclasses, with respect to my typical applications?''.

The answer depends on the kind of problem you are considering. For certain
classes of problems it can be \emph{very} large, as I will show in this and in
the next chapters.

I think the biggest advantage of metaclasses is \emph{elegance}. Altough it
is true that most of what you can do with metaclasses, can be done without 
metaclasses, not using metaclasses can result in a much \emph{uglier} solution.

One needs difficult problems in order to appreciate the advantage
of powerful methods.

If all you need is to write few scripts for copying two or three files,
there is no point in learning OOP.On the other hand, if you only
write simple programs where you define only one of two classes, there
is no point in using metaclasses. Metaclasses becomes relevant only
when you have many classes, whole classes of classes with similar
features that you want to modify.

In this sense, metaprogramming is for experts only, i.e. with people
with difficult problems. The point however, is that nowaydays,
many persons have difficult problems.

Finally, let me conclude this preface by recalling the
gist of Python wisdom.
\begin{quote}
\begin{verbatim}>>> import this
The Zen of Python, by Tim Peters
.
Beautiful is better than ugly.
Explicit is better than implicit.
Simple is better than complex.
Complex is better than complicated.
Flat is better than nested.
Sparse is better than dense.
Readability counts.
Special cases aren't special enough to break the rules.
Although practicality beats purity.
Errors should never pass silently.
Unless explicitly silenced.
In the face of ambiguity, refuse the temptation to guess.
There should be one-- and preferably only one --obvious way to do it.
Although that way may not be obvious at first unless you're Dutch.
Now is better than never.
Although never is often better than *right* now.
If the implementation is hard to explain, it's a bad idea.
If the implementation is easy to explain, it may be a good idea.
Namespaces are one honking great idea -- let's do more of those!\end{verbatim}
\end{quote}

\end{document}

