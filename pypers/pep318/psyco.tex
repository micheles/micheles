\documentclass[10pt,english]{article}
\usepackage{babel}
\usepackage{shortvrb}
\usepackage[latin1]{inputenc}
\usepackage{tabularx}
\usepackage{longtable}
\setlength{\extrarowheight}{2pt}
\usepackage{amsmath}
\usepackage{graphicx}
\usepackage{color}
\usepackage{multirow}
\usepackage[colorlinks=true,linkcolor=blue,urlcolor=blue]{hyperref}
\usepackage[a4paper,margin=2cm,nohead]{geometry}
%% generator Docutils: http://docutils.sourceforge.net/
\newlength{\admonitionwidth}
\setlength{\admonitionwidth}{0.9\textwidth}
\newlength{\docinfowidth}
\setlength{\docinfowidth}{0.9\textwidth}
\newcommand{\optionlistlabel}[1]{\bf #1 \hfill}
\newenvironment{optionlist}[1]
{\begin{list}{}
  {\setlength{\labelwidth}{#1}
   \setlength{\rightmargin}{1cm}
   \setlength{\leftmargin}{\rightmargin}
   \addtolength{\leftmargin}{\labelwidth}
   \addtolength{\leftmargin}{\labelsep}
   \renewcommand{\makelabel}{\optionlistlabel}}
}{\end{list}}
% begin: floats for footnotes tweaking.
\setlength{\floatsep}{0.5em}
\setlength{\textfloatsep}{\fill}
\addtolength{\textfloatsep}{3em}
\renewcommand{\textfraction}{0.5}
\renewcommand{\topfraction}{0.5}
\renewcommand{\bottomfraction}{0.5}
\setcounter{totalnumber}{50}
\setcounter{topnumber}{50}
\setcounter{bottomnumber}{50}
% end floats for footnotes
% some commands, that could be overwritten in the style file.
\newcommand{\rubric}[1]{\subsection*{~\hfill {\it #1} \hfill ~}}
% end of "some commands"
% donot indent first line.
\setlength{\parindent}{0pt}
\setlength{\parskip}{5pt plus 2pt minus 1pt}

% sloppy
% ------
% Less strict (opposite to default fussy) space size between words. Therefore
% less hyphenation.
\sloppy

% fonts
% -----
% times for pdf generation, gives smaller pdf files.
%
% But in standard postscript fonts: courier and times/helvetica do not fit.
% Maybe use pslatex.
\usepackage{times} 

% pagestyle
\pagestyle{headings}

\title{Module psyco}
\author{}
\date{}
\hypersetup{
pdftitle={Module psyco}
}
\raggedbottom
\begin{document}
\maketitle


Psyco -- the Python Specializing Compiler.

Typical usage: add the following lines to your application's main module:
\begin{description}
%[visit_definition_list_item]
\item[try::]
%[visit_definition]

import psyco
psyco.profile()

%[depart_definition]
%[depart_definition_list_item]
%[visit_definition_list_item]
\item[except::]
%[visit_definition]

print 'Psyco not found, ignoring it'

%[depart_definition]
%[depart_definition_list_item]
\end{description}


%___________________________________________________________________________

\hypertarget{functions}{}
\section*{Functions}
\pdfbookmark[0]{Functions}{functions}

\texttt{cannotcompile(x):}
\begin{quote}

Instruct Psyco never to compile the given function, method
or code object.
\end{quote}

\texttt{log(logfile='', mode='w', top=10):}
\begin{quote}

Enable logging to the given file.

If the file name is unspecified, a default name is built by appending
a 'log-psyco' extension to the main script name.

Mode is 'a' to append to a possibly existing file or 'w' to overwrite
an existing file. Note that the log file may grow quickly in 'a' mode.
\end{quote}

\texttt{runonly(memory=None, time=None, memorymax=None, timemax=None):}
\begin{quote}

Nonprofiler.

XXX check if this is useful and document.
\end{quote}

\texttt{profile(watermark   = default{\_}watermark,}
\begin{quote}

Turn on profiling.

The 'watermark' parameter controls how easily running functions will
be compiled. The smaller the value, the more functions are compiled.
\end{quote}

\texttt{full(memory=None, time=None, memorymax=None, timemax=None):}
\begin{quote}

Compile as much as possible.

Typical use is for small scripts performing intensive computations
or string handling.
\end{quote}

\texttt{dumpcodebuf():}
\begin{quote}

Write in file psyco.dump a copy of the emitted machine code,
provided Psyco was compiled with a non-zero CODE{\_}DUMP.
See py-utils/httpxam.py to examine psyco.dump.
\end{quote}

\texttt{stop():}
\begin{quote}

Turn off all automatic compilation.  bind() calls remain in effect.
\end{quote}

\texttt{proxy(x, rec=None):}
\begin{quote}

Return a Psyco-enabled copy of the function.

The original function is still available for non-compiled calls.
The optional second argument specifies the number of recursive
compilation levels: all functions called by func are compiled
up to the given depth of indirection.
\end{quote}

\texttt{background(watermark   = default{\_}watermark,}
\begin{quote}

Turn on passive profiling.

This is a very lightweight mode in which only intensively computing
functions can be detected. The smaller the 'watermark', the more functions
are compiled.
\end{quote}

\texttt{unbind(x):}
\begin{quote}

Reverse of bind().
\end{quote}

\texttt{bind(x, rec=None):}
\begin{quote}

Enable compilation of the given function, method, or class object.

If C is a class (or anything with a '{\_}{\_}dict{\_}{\_}' attribute), bind(C) will
rebind all functions and methods found in C.{\_}{\_}dict{\_}{\_} (which means, for
classes, all methods defined in the class but not in its parents).

The optional second argument specifies the number of recursive
compilation levels: all functions called by func are compiled
up to the given depth of indirection.
\end{quote}

\texttt{{\_}getemulframe(depth=0):}
\begin{quote}

As {\_}getframe(), but the returned objects are real Python frame objects
emulating Psyco frames. Some of their attributes can be wrong or missing,
however.
\end{quote}

\texttt{unproxy(proxy):}
\begin{quote}

Return a new copy of the original function of method behind a proxy.
The result behaves like the original function in that calling it
does not trigger compilation nor execution of any compiled code.
\end{quote}

\end{document}
