\documentclass[10pt,english]{article}
\usepackage{babel}
\usepackage{shortvrb}
\usepackage[latin1]{inputenc}
\usepackage{tabularx}
\usepackage{longtable}
\setlength{\extrarowheight}{2pt}
\usepackage{amsmath}
\usepackage{graphicx}
\usepackage{color}
\usepackage{multirow}
\usepackage[colorlinks=true,linkcolor=blue,urlcolor=blue]{hyperref}
\usepackage[a4paper,margin=2cm,nohead]{geometry}
%% generator Docutils: http://docutils.sourceforge.net/
\newlength{\admonitionwidth}
\setlength{\admonitionwidth}{0.9\textwidth}
\newlength{\docinfowidth}
\setlength{\docinfowidth}{0.9\textwidth}
\newcommand{\optionlistlabel}[1]{\bf #1 \hfill}
\newenvironment{optionlist}[1]
{\begin{list}{}
  {\setlength{\labelwidth}{#1}
   \setlength{\rightmargin}{1cm}
   \setlength{\leftmargin}{\rightmargin}
   \addtolength{\leftmargin}{\labelwidth}
   \addtolength{\leftmargin}{\labelsep}
   \renewcommand{\makelabel}{\optionlistlabel}}
}{\end{list}}
% begin: floats for footnotes tweaking.
\setlength{\floatsep}{0.5em}
\setlength{\textfloatsep}{\fill}
\addtolength{\textfloatsep}{3em}
\renewcommand{\textfraction}{0.5}
\renewcommand{\topfraction}{0.5}
\renewcommand{\bottomfraction}{0.5}
\setcounter{totalnumber}{50}
\setcounter{topnumber}{50}
\setcounter{bottomnumber}{50}
% end floats for footnotes
% some commands, that could be overwritten in the style file.
\newcommand{\rubric}[1]{\subsection*{~\hfill {\it #1} \hfill ~}}
% end of "some commands"
% donot indent first line.
\setlength{\parindent}{0pt}
\setlength{\parskip}{5pt plus 2pt minus 1pt}

% sloppy
% ------
% Less strict (opposite to default fussy) space size between words. Therefore
% less hyphenation.
\sloppy

% fonts
% -----
% times for pdf generation, gives smaller pdf files.
%
% But in standard postscript fonts: courier and times/helvetica do not fit.
% Maybe use pslatex.
\usepackage{times} 

% pagestyle
\pagestyle{headings}

\title{Documentation of the oopp module}
\author{}
\date{}
\hypersetup{
pdftitle={Documentation of the oopp module}
}
\raggedbottom
\begin{document}
\maketitle



%___________________________________________________________________________

\hypertarget{documented-metaclasses}{}
\section*{Documented metaclasses}
\pdfbookmark[0]{Documented metaclasses}{documented-metaclasses}

\texttt{Final(type): {\#} better derived from WithCounter,typ}
\begin{quote}

Instances of Final cannot be derived
\end{quote}

\texttt{AutoWrapped(type)}
\begin{quote}

Metaclass that looks at the methods declared in the attributes 
builtinlist and wraplist of its instances and wraps them with
autowrappedmethod.
\end{quote}

\texttt{Wrapped(Customizable,type)}
\begin{quote}

A customizable metaclass to wrap methods with a given wrapper and
a given condition
\begin{quote}

\texttt{Reflective(type)}
\begin{quote}

Cooperative metaclass that defines the private variable {\_}{\_}this in
its instances. {\_}{\_}this contains a reference to the class, therefore
it allows anonymous cooperative super calls in the class.
\end{quote}

\texttt{wrappedmethod(Customizable)}
\begin{quote}

Customizable method factory intended for derivation.
The wrapper method is overridden in the children.
\end{quote}
\end{quote}
\end{quote}

\texttt{Cooperative(BracketCallable,type)}
\begin{quote}

Bracket-callable metaclass implementing cooperative methods. Works
well for plain methods returning None, such as {\_}{\_}init{\_}{\_}

\texttt{coop{\_}method(cls,name,method): {\#} method can be Non}
\begin{quote}

Calls both the superclass method and the class method (if the 
class has an explicit method). Implemented via a closure
\end{quote}
\end{quote}

\texttt{Logged(WithLogger,Reflective)}
\begin{quote}

Metaclass that reuses the features provided by WithLogger. 
In particular the classes created by Logged are Reflective, 
PrettyPrinted and Customizable.
\end{quote}

\texttt{MagicallyTransformed(type)}
\begin{quote}

Metaclass changing the formatstring of its instances
\end{quote}

\texttt{Printable(PrettyPrinted,type)}
\begin{quote}

Apparently does nothing, but actually makes PrettyPrinted acting as
a metaclass.
\end{quote}


%___________________________________________________________________________

\hypertarget{documented-classes}{}
\section*{Documented classes}
\pdfbookmark[0]{Documented classes}{documented-classes}

\texttt{convert2descriptor(object)}
\begin{quote}

To all practical means, this class acts as a function that, given an
object, adds to it a {\_}{\_}get{\_}{\_} method if it is not already there. The 
added {\_}{\_}get{\_}{\_} method is trivial and simply returns the original object, 
independently from obj and cls.

\texttt{{\_}{\_}get{\_}{\_}(self,obj,cls=None)}
\begin{quote}

Returns self.a independently from obj and cls
\end{quote}
\end{quote}

\texttt{Object(object)}
\begin{quote}

A convenient Object class
\end{quote}

\texttt{GeometricFigure(object): {\#}an example of object factor}
\begin{quote}

This class allows to define geometric figures according to their
equation in the cartesian plane. It will be extended later.

\texttt{{\_}{\_}init{\_}{\_}(self,equation,**parameters)}
\begin{quote}

Specify the cartesian equation of the object and its parameters
\end{quote}
\end{quote}

\texttt{UglyDuckling(PrettyPrinted)}
\begin{quote}

A plain, regular class
\end{quote}

\texttt{autowrappedmethod(wrappedmethod)}
\begin{quote}

Makes the method returning cls instances, by wrapping its
output with cls
\end{quote}

\texttt{Clock(object)}
\begin{quote}

Clock with a staticmethod
\end{quote}

\texttt{AccessError(object)}
\begin{quote}

Descriptor raising an AttributeError when the attribute is 
accessed
\end{quote}

\texttt{Frozen(object)}
\begin{quote}

Subclasses of Frozen are frozen, i.e. it is impossibile to add
new attributes to them and their instances
\end{quote}

\texttt{Timer(Clock)}
\begin{quote}

Inherits the get{\_}time staticmethod from Clock
\end{quote}

\texttt{Vector(list)}
\begin{quote}

Implements finite dimensional vectors as lists. Can be instantiated
as Vector([a,b,c,..]) or as Vector(a,b,c ..)
\end{quote}

\texttt{Singleton(WithCounter)}
\begin{quote}

If you inherit from me, you can only have one instance
\end{quote}

\texttt{Homo(PrettyPrinted)}
\begin{quote}

Defines the method 'can', which is intended to be overriden 
in the children classes, and inherits '{\_}{\_}str{\_}{\_}' from PrettyPrinted,
ensuring a nice printing representation for all children.
\end{quote}

\texttt{kwdict(dict)}
\begin{quote}

Keyword dictionary base class

\texttt{pretty(dic)}
\begin{quote}

Returns a nice string representation for the dictionary
\end{quote}
\end{quote}

\texttt{WithLogger(WithCounter,PrettyPrinted)}
\begin{quote}

WithLogger inherits from WithCounter the 'count' class attribute; 
moreover it inherits '{\_}{\_}str{\_}{\_}' from PrettyPrinted
\end{quote}

\texttt{Get(object)}
\begin{quote}

Invoked as Get(cls)(xxx) where xxx = staticmethod, classmethod,
property, plainmethod, plaindata, returns the corresponding 
attributes as a keyword dictionary. It works by internally calling 
the routine inspect.classify{\_}class{\_}attrs. Notice that special data
attributes are not retrieved (this is by design).
\end{quote}

\texttt{Makeobj(object)}
\begin{quote}

A factory of object factories. Makeobj(cls) returns instances
of cls
\end{quote}

\texttt{ExampleBaseClass(PrettyPrinted)}
\begin{quote}

Contains a regular method 'm', a staticmethod 's', a classmethod 
'c', a property 'p' and a data attribute 'd'.
\end{quote}

\texttt{TransformedUglyDuckling(PrettyPrinted)}
\begin{quote}

A class metamagically modified
\end{quote}

\texttt{Customizable(object)}
\begin{quote}

Classes inhering from 'Customizable' have a 'with' method acting as
an object modifier and 'With' classmethod acting as a class factory
\end{quote}

\texttt{WithMultiCounter(WithCounter)}
\begin{quote}

Each time a new subclass is derived, the counter is reset
\end{quote}

\texttt{BracketCallable(object)}
\begin{quote}

Any subclass C(BracketCallable) can be called with the syntax C[t], 
where t is a tuple of arguments stored in bracket{\_}args;  returns the 
class or an instance of it, depending on the flag 'returnclass'.
\end{quote}

\texttt{ClsFactory(BracketCallable)}
\begin{quote}
\begin{description}
%[visit_definition_list_item]
\item[Bracket callable non-cooperative class acting as :]
%[visit_definition]

a factory of class factories.
ClsFactory instances are class factories accepting 0,1,2 or 3 arguments.

%[depart_definition]
%[depart_definition_list_item]
%[visit_definition_list_item]
\item[. They automatically converts functions to static methods :]
%[visit_definition]

if the input object is not a class. If an explicit name is not passed
the name of the created class is obtained by adding an underscore to 
the name of the original object.

%[depart_definition]
%[depart_definition_list_item]
\end{description}

\texttt{{\_}{\_}call{\_}{\_}(self, *args)}
\begin{quote}

Generates a new class using self.meta and avoiding conflicts.
The first metaobject can be a dictionary, an object with a
dictionary (except a class), or a simple name.
\end{quote}
\end{quote}

\texttt{WithCounter(object)}
\begin{quote}

Mixin class counting the total number of its instances and storing 
it in the class attribute counter.
\end{quote}


%___________________________________________________________________________

\hypertarget{documented-functions}{}
\section*{Documented functions}
\pdfbookmark[0]{Documented functions}{documented-functions}

\texttt{mandelbrot(row,col)}
\begin{quote}

Computes the Mandelbrot set in one line
\end{quote}

\texttt{with{\_}tracer(function,namespace='{\_}{\_}main{\_}{\_}',output=sys.stdout, indent=[0])}
\begin{quote}

Closure returning traced functions. It is typically invoked
trough an auxiliary function fixing the parameters of with{\_}tracer.
\end{quote}

\texttt{ancestor(C,S=None)}
\begin{quote}

Returns the ancestors of the first argument with respect to the 
MRO of the second argument. If the second argument is None, then 
returns the MRO of the first argument.
\end{quote}

\texttt{customize(obj,errfile=None,**kw)}
\begin{quote}

Adds attributes to an object, if possible. If not, writes an error
message on 'errfile'. If errfile is None, skips the exception.
\end{quote}

\texttt{loop{\_}overhead(N)}
\begin{quote}

Computes the time spent in empty loop of N iterations
\end{quote}

\texttt{indent(block,n)}
\begin{quote}

Indent a block of code by n spaces
\end{quote}

\texttt{totuple(arg)}
\begin{quote}

Converts the argument to a tuple, if need there is
\end{quote}

\texttt{attributes(obj,condition=lambda n,v: not special(n))}
\begin{quote}

Returns a dictionary containing the accessible attributes of 
an object. By default, returns the non-special attributes only.
\end{quote}

\texttt{generateblocks(regexp,text)}
\begin{quote}

Generator splitting text in blocks according to regexp
\end{quote}

\texttt{wrapfunctions(obj,wrapper,err=None,**options)}
\begin{quote}

Traces the callable objects in an object with a dictionary
\end{quote}

\texttt{wrap(obj,wrapped,condition=lambda k,v: True, err=None)}
\begin{quote}

Retrieves obj's dictionary and wraps it
\end{quote}

\texttt{reflective(*classes)}
\begin{quote}

Reflective classes know themselves, i.e. they own a private
attribute {\_}{\_}this containing a reference to themselves. If the class
name starts with '{\_}', the underscores are stripped. This is needed
to make the mangling mechanism work.
\end{quote}

\texttt{withmemory(f)}
\begin{quote}

This closure invokes the callable object f only if need there is
\end{quote}

\texttt{get{\_}time()}
\begin{quote}

Return the time of the system in the format HH:MM:SS
\end{quote}

\texttt{prn(s)}
\begin{quote}

Given an evaluable string, print its value and its object reference.
Notice that the evaluation is done in the {\_}{\_}main{\_}{\_} dictionary.
\end{quote}

\texttt{isplaindata(a)}
\begin{quote}

A data attribute has no {\_}{\_}get{\_}{\_} or {\_}{\_}set{\_}{\_} attributes, is not
a built-in function, nor a built-in method.
\end{quote}

\texttt{Pizza(toppings,**dic)}
\begin{quote}

This function produces classes inheriting from GenericPizza and 
WithLogger, using a metaclass inferred from Logged
\end{quote}

\texttt{modulesub(s,r,module)}
\begin{quote}

Requires 2.3
\end{quote}

\texttt{memoize(f)}
\begin{quote}

This closure remembers all f invocations
\end{quote}

\texttt{with{\_}timer(func, modulename='{\_}{\_}main{\_}{\_}', n=1, logfile=sys.stdout)}
\begin{quote}

Wraps the function func and executes it n times (default n=1). 
The average time spent in one iteration, express in milliseconds, 
is stored in the attributes func.time and func.CPUtime, and saved 
in a log file which defaults to the standard output.
\end{quote}

\texttt{dedent(text)}
\begin{quote}

dedent(text : string) -{$>$} string
\begin{quote}
\begin{quote}

Remove any whitespace than can be uniformly removed from the left
of every line in text.

This can be used e.g. to make triple-quoted strings line up with
the left edge of screen/whatever, while still presenting it in the
source code in indented form.

For example:
\begin{quote}
\begin{quote}
\begin{description}
%[visit_definition_list_item]
\item[def test()::]
%[visit_definition]

{\#} end first line with to avoid the empty line!
s = '''            hello

Unexpected indentation.

\begin{quote}

world
\end{quote}

Block quote ends without a blank line; unexpected unindent.


'''
print repr(s)          {\#} prints '    hello

%[depart_definition]
%[depart_definition_list_item]
\end{description}
\end{quote}

Block quote ends without a blank line; unexpected unindent.


world
\end{quote}

Block quote ends without a blank line; unexpected unindent.

\begin{description}
%[visit_definition_list_item]
\item[':]
%[visit_definition]

print repr(dedent(s))  {\#} prints 'hello

%[depart_definition]
%[depart_definition_list_item]
\end{description}
\end{quote}

Block quote ends without a blank line; unexpected unindent.


world
\end{quote}

Block quote ends without a blank line; unexpected unindent.


'
\end{quote}

\texttt{makecls(*metas,**options)}
\begin{quote}

Class factory avoiding metatype conflicts. The invocation syntax is
makecls(M1,M2,..,priority=1)(name,bases,dic). If the base classes have 
metaclasses conflicting within themselves or with the given metaclasses,
it automatically generates a compatible metaclass and instantiate it. 
If priority is True, the given metaclasses have priority over the 
bases' metaclasses
\end{quote}

\end{document}
