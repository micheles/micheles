\documentclass[10pt,a4paper,english]{article}
\usepackage{babel}
\usepackage{ae}
\usepackage{aeguill}
\usepackage{shortvrb}
\usepackage[latin1]{inputenc}
\usepackage{tabularx}
\usepackage{longtable}
\setlength{\extrarowheight}{2pt}
\usepackage{amsmath}
\usepackage{graphicx}
\usepackage{color}
\usepackage{multirow}
\usepackage{ifthen}
\usepackage[colorlinks=true,linkcolor=blue,urlcolor=blue]{hyperref}
\usepackage[DIV12]{typearea}
%% generator Docutils: http://docutils.sourceforge.net/
\newlength{\admonitionwidth}
\setlength{\admonitionwidth}{0.9\textwidth}
\newlength{\docinfowidth}
\setlength{\docinfowidth}{0.9\textwidth}
\newlength{\locallinewidth}
\newcommand{\optionlistlabel}[1]{\bf #1 \hfill}
\newenvironment{optionlist}[1]
{\begin{list}{}
  {\setlength{\labelwidth}{#1}
   \setlength{\rightmargin}{1cm}
   \setlength{\leftmargin}{\rightmargin}
   \addtolength{\leftmargin}{\labelwidth}
   \addtolength{\leftmargin}{\labelsep}
   \renewcommand{\makelabel}{\optionlistlabel}}
}{\end{list}}
\newlength{\lineblockindentation}
\setlength{\lineblockindentation}{2.5em}
\newenvironment{lineblock}[1]
{\begin{list}{}
  {\setlength{\partopsep}{\parskip}
   \addtolength{\partopsep}{\baselineskip}
   \topsep0pt\itemsep0.15\baselineskip\parsep0pt
   \leftmargin#1}
 \raggedright}
{\end{list}}
% begin: floats for footnotes tweaking.
\setlength{\floatsep}{0.5em}
\setlength{\textfloatsep}{\fill}
\addtolength{\textfloatsep}{3em}
\renewcommand{\textfraction}{0.5}
\renewcommand{\topfraction}{0.5}
\renewcommand{\bottomfraction}{0.5}
\setcounter{totalnumber}{50}
\setcounter{topnumber}{50}
\setcounter{bottomnumber}{50}
% end floats for footnotes
% some commands, that could be overwritten in the style file.
\newcommand{\rubric}[1]{\subsection*{~\hfill {\it #1} \hfill ~}}
\newcommand{\titlereference}[1]{\textsl{#1}}
% end of "some commands"
% donot indent first line.
\setlength{\parindent}{0pt}
\setlength{\parskip}{5pt plus 2pt minus 1pt}

% sloppy
% ------
% Less strict (opposite to default fussy) space size between words. Therefore
% less hyphenation.
\sloppy

% fonts
% -----
% times for pdf generation, gives smaller pdf files.
%
% But in standard postscript fonts: courier and times/helvetica do not fit.
% Maybe use pslatex.
\usepackage{times} 

% pagestyle
\pagestyle{headings}

\title{The Python 2.3 Method Resolution Order}
\author{}
\date{}
\hypersetup{
pdftitle={The Python 2.3 Method Resolution Order},
pdfauthor={Michele Simionato}
}
\raggedbottom
\begin{document}
\maketitle

%___________________________________________________________________________
\begin{center}
\begin{tabularx}{\docinfowidth}{lX}
\textbf{Version}: &
	1.4 \\
\textbf{Author}: &
	Michele Simionato \\
\textbf{E-mail}: &
	michelesimionato@libero.it \\
\textbf{Address}: &
	{\raggedright
Department of Physics and Astronomy~\\
210 Allen Hall Pittsburgh PA 15260 U.S.A. } \\
\textbf{Home-page}: &
	http://www.phyast.pitt.edu/{\textasciitilde}micheles/ \\
\end{tabularx}
\end{center}

\setlength{\locallinewidth}{\linewidth}


\subsubsection*{~\hfill Abstract\hfill ~}

\emph{This document is intended for Python programmers who want to
understand the C3 Method Resolution Order used in Python 2.3.
Although it is not intended for newbies, it is quite pedagogical with
many worked out examples.  I am not aware of other publicly available
documents with the same scope, therefore it should be useful.}


Disclaimer:
\begin{quote}

I donate this document to the Python Software Foundation, under the
Python 2.3 license.  As usual in these circumstances, I warn the
reader that what follows \emph{should} be correct, but I don't give any
warranty.  Use it at your own risk and peril!
\end{quote}

Acknowledgments:
\begin{quote}

All the people of the Python mailing list who sent me their support.
Paul Foley who pointed out various imprecisions and made me to add the
part on local precedence ordering. David Goodger for help with the
formatting in reStructuredText. David Mertz for help with the editing.
Joan G. Stark for the pythonic pictures. Finally, Guido van Rossum who 
enthusiastically added this document to the official Python 2.3 home-page.
\end{quote}


%___________________________________________________________________________
\hspace*{\fill}\hrulefill\hspace*{\fill}

\begin{quote}
\begin{quote}{\ttfamily \raggedright \noindent
~~~~~~~~~~~~~~~~~~~~~~.-=-.~~~~~~~~~~.-{}-.~\\
~~~~~~~~~~{\_}{\_}~~~~~~~~.'~~~~~'.~~~~~~~/~~"~)~\\
~~{\_}~~~~~.'~~'.~~~~~/~~~.-.~~~{\textbackslash}~~~~~/~~.-'{\textbackslash}~\\
~(~{\textbackslash}~~~/~.-.~~{\textbackslash}~~~/~~~/~~~{\textbackslash}~~~{\textbackslash}~~~/~~/~~~~{\textasciicircum}~\\
~~{\textbackslash}~`-`~/~~~{\textbackslash}~~`-'~~~/~~~~~{\textbackslash}~~~`-`~~/~\\
jgs`-.-`~~~~~'.{\_}{\_}{\_}{\_}.'~~~~~~~`.{\_}{\_}{\_}{\_}.'
}\end{quote}
\end{quote}


%___________________________________________________________________________

\hypertarget{the-beginning}{}
\pdfbookmark[0]{The beginning}{the-beginning}
\section*{The beginning}
\begin{quote}

\emph{Felix qui potuit rerum cognoscere causas} -{}- Virgilius
\end{quote}

Everything started with a post by Samuele Pedroni to the Python
development mailing list [\hyperlink{id4}{1}].  In his post, Samuele showed that the
Python 2.2 method resolution order is not monotonic and he proposed to
replace it with the C3 method resolution order.  Guido agreed with his
arguments and therefore now Python 2.3 uses C3.  The C3 method itself
has nothing to do with Python, since it was invented by people working
on Dylan and it is described in a paper intended for lispers [\hyperlink{id5}{2}].  The
present paper gives a (hopefully) readable discussion of the C3
algorithm for Pythonistas who want to understand the reasons for the
change.

First of all, let me point out that what I am going to say only applies
to the \emph{new style classes} introduced in Python 2.2:  \emph{classic classes}
maintain their old method resolution order, depth first and then left to
right.  Therefore, there is no breaking of old code for classic classes;
and even if in principle there could be breaking of code for Python 2.2
new style classes, in practice the cases in which the C3 resolution
order differs from the Python 2.2 method resolution order are so rare
that no real breaking of code is expected.  Therefore:
\begin{quote}

\emph{Don't be scared!}
\end{quote}

Moreover, unless you make strong use of multiple inheritance and you
have non-trivial hierarchies, you don't need to understand the C3
algorithm, and you can easily skip this paper.  On the other hand, if
you really want to know how multiple inheritance works, then this paper
is for you.  The good news is that things are not as complicated as you
might expect.

Let me begin with some basic definitions.
\newcounter{listcnt1}
\begin{list}{\arabic{listcnt1})}
{
\usecounter{listcnt1}
\setlength{\rightmargin}{\leftmargin}
}
\item {} 
Given a class C in a complicated multiple inheritance hierarchy, it
is a non-trivial task to specify the order in which methods are
overridden, i.e. to specify the order of the ancestors of C.

\item {} 
The list of the ancestors of a class C, including the class itself,
ordered from the nearest ancestor to the furthest, is called the
class precedence list or the \emph{linearization} of C.

\item {} 
The \emph{Method Resolution Order} (MRO) is the set of rules that
construct the linearization.  In the Python literature, the idiom
``the MRO of C'' is also used as a synonymous for the linearization of
the class C.

\item {} 
For instance, in the case of single inheritance hierarchy, if C is a
subclass of C1, and C1 is a subclass of C2, then the linearization of
C is simply the list {[}C, C1 , C2].  However, with multiple
inheritance hierarchies, the construction of the linearization is 
more cumbersome, since it is more difficult to construct a
linearization that respects \emph{local precedence ordering} and
\emph{monotonicity}.

\item {} 
I will discuss the local precedence ordering later, but I can give
the definition of monotonicity here.  A MRO is monotonic when the
following is true:  \emph{if C1 precedes C2 in the linearization of C,
then C1 precedes C2 in the linearization of any subclass of C}.
Otherwise, the innocuous operation of deriving a new class could
change the resolution order of methods, potentially introducing very
subtle bugs.  Examples where this happens will be shown later.

\item {} 
Not all classes admit a linearization.  There are cases, in
complicated hierarchies, where it is not possible to derive a class
such that its linearization respects all the desired properties.

\end{list}

Here I give an example of this situation. Consider the hierarchy
\begin{quote}
\begin{verbatim}>>> O = object
>>> class X(O): pass
>>> class Y(O): pass
>>> class A(X,Y): pass
>>> class B(Y,X): pass\end{verbatim}
\end{quote}

which can be represented with the following inheritance graph, where I
have denoted with O the \texttt{object} class, which is the beginning of any
hierarchy for new style classes:
\begin{quote}
\begin{quote}{\ttfamily \raggedright \noindent
~-{}-{}-{}-{}-{}-{}-{}-{}-{}-{}-~\\
|~~~~~~~~~~~|~\\
|~~~~O~~~~~~|~\\
|~~/~~~{\textbackslash}~~~~|~\\
~-~X~~~~Y~~/~\\
~~~|~~/~|~/~\\
~~~|~/~~|/~\\
~~~A~~~~B~\\
~~~{\textbackslash}~~~/~\\
~~~~~?
}\end{quote}
\end{quote}

In this case, it is not possible to derive a new class C from A and B,
since X precedes Y in A, but Y precedes X in B, therefore the method
resolution order would be ambiguous in C.

Python 2.3 raises an exception in this situation (TypeError:  MRO
conflict among bases Y, X) forbidding the naive programmer from creating
ambiguous hierarchies.  Python 2.2 instead does not raise an exception,
but chooses an \emph{ad hoc} ordering (CABXYO in this case).


%___________________________________________________________________________
\hspace*{\fill}\hrulefill\hspace*{\fill}

\begin{quote}
\begin{quote}{\ttfamily \raggedright \noindent
~~~~{\_}~~~~~~~~~~~~~~~~~~~.-=-.~~~~~~~~~~.-==-.~\\
~~~{\{}~{\}}~~~~~~{\_}{\_}~~~~~~~~.'~O~o~'.~~~~~~~/~~-<'~)~\\
~~~{\{}~{\}}~~~~.'~O'.~~~~~/~o~.-.~O~{\textbackslash}~~~~~/~~.-{}-v`~\\
~~~{\{}~{\}}~~~/~.-.~o{\textbackslash}~~~/O~~/~~~{\textbackslash}~~o{\textbackslash}~~~/O~/~\\
~~~~{\textbackslash}~`-`~/~~~{\textbackslash}~O`-'o~~/~~~~~{\textbackslash}~~O`-`o~/~\\
jgs~~`-.-`~~~~~'.{\_}{\_}{\_}{\_}.'~~~~~~~`.{\_}{\_}{\_}{\_}.'
}\end{quote}
\end{quote}


%___________________________________________________________________________

\hypertarget{the-c3-method-resolution-order}{}
\pdfbookmark[0]{The C3 Method Resolution Order}{the-c3-method-resolution-order}
\section*{The C3 Method Resolution Order}

Let me introduce a few simple notations which will be useful for the
following discussion.  I will use the shortcut notation
\begin{quote}

C1 C2 ... CN
\end{quote}

to indicate the list of classes {[}C1, C2, ... , CN].

The \emph{head} of the list is its first element:
\begin{quote}

head = C1
\end{quote}

whereas the \emph{tail} is the rest of the list:
\begin{quote}

tail = C2 ... CN.
\end{quote}

I shall also use the notation
\begin{quote}

C + (C1 C2 ... CN) = C C1 C2 ... CN
\end{quote}

to denote the sum of the lists {[}C] + {[}C1, C2, ... ,CN].

Now I can explain how the MRO works in Python 2.3.

Consider a class C in a multiple inheritance hierarchy, with C
inheriting from the base classes B1, B2, ...  , BN.  We want to 
compute the linearization L{[}C] of the class C. The rule is the
following:
\begin{quote}

\emph{the linearization of C is the sum of C plus the merge of the
linearizations of the parents and the list of the parents.}
\end{quote}

In symbolic notation:
\begin{quote}

L{[}C(B1 ... BN)] = C + merge(L{[}B1] ... L{[}BN], B1 ... BN)
\end{quote}

In particular, if C is the \texttt{object} class, which has no parents, the
linearization is trivial:
\begin{quote}

L{[}object] = object.
\end{quote}

However, in general one has to compute the merge according to the following 
prescription:
\begin{quote}

\emph{take the head of the first list, i.e L{[}B1]{[}0]; if this head is not in
the tail of any of the other lists, then add it to the linearization
of C and remove it from the lists in the merge, otherwise look at the
head of the next list and take it, if it is a good head.  Then repeat
the operation until all the class are removed or it is impossible to
find good heads.  In this case, it is impossible to construct the
merge, Python 2.3 will refuse to create the class C and will raise an
exception.}
\end{quote}

This prescription ensures that the merge operation \emph{preserves} the
ordering, if the ordering can be preserved.  On the other hand, if the
order cannot be preserved (as in the example of serious order
disagreement discussed above) then the merge cannot be computed.

The computation of the merge is trivial if C has only one parent 
(single inheritance); in this case
\begin{quote}

L{[}C(B)] = C + merge(L{[}B],B) = C + L{[}B]
\end{quote}

However, in the case of multiple inheritance things are more cumbersome
and I don't expect you can understand the rule without a couple of
examples ;-)


%___________________________________________________________________________
\hspace*{\fill}\hrulefill\hspace*{\fill}

\begin{quote}
\begin{quote}{\ttfamily \raggedright \noindent
~~~~~~~~~~.-'-.~\\
~~~~~~~~/'~~~~~`{\textbackslash}~\\
~~~~~~/'~{\_}.-.-.{\_}~`{\textbackslash}~\\
~~~~~|~~(|)~~~(|)~~|~\\
~~~~~|~~~{\textbackslash}{\_}{\_}"{\_}{\_}/~~~|~\\
~~~~~{\textbackslash}~~~~|v.v|~~~~/~\\
~~~~~~{\textbackslash}~~~|~|~|~~~/~\\
~~~~~~~`{\textbackslash}~|={\textasciicircum}-|~/'~\\
~~~~~~~~~`|=-=|'~\\
~~~~~~~~~~|~-~|~\\
~~~~~~~~~~|=~~|~\\
~~~~~~~~~~|-=-|~\\
~~~~{\_}.-=-=|=~-|=-=-.{\_}~\\
~~~(~~~~~~|{\_}{\_}{\_}|~~~~~~)~\\
~~(~`-=-=-=-=-=-=-=-`~)~\\
~~(`-=-=-=-=-=-=-=-=-`)~\\
~~(`-=-=-=-=-=-=-=-=-`)~\\
~~~(`-=-=-=-=-=-=-=-`)~\\
~~~~(`-=-=-=-=-=-=-`)~\\
jgs~~`-=-=-=-=-=-=-`
}\end{quote}
\end{quote}


%___________________________________________________________________________

\hypertarget{examples}{}
\pdfbookmark[0]{Examples}{examples}
\section*{Examples}

First example. Consider the following hierarchy:
\begin{quote}
\begin{verbatim}>>> O = object
>>> class F(O): pass
>>> class E(O): pass
>>> class D(O): pass
>>> class C(D,F): pass
>>> class B(D,E): pass
>>> class A(B,C): pass\end{verbatim}
\end{quote}

In this case the inheritance graph can be drawn as
\begin{quote}
\begin{quote}{\ttfamily \raggedright \noindent
~~~~~~~~~~~~~~~~~~~~~~~~~~6~\\
~~~~~~~~~~~~~~~~~~~~~~~~~-{}-{}-~\\
Level~3~~~~~~~~~~~~~~~~~|~O~|~~~~~~~~~~~~~~~~~~(more~general)~\\
~~~~~~~~~~~~~~~~~~~~~~/~~-{}-{}-~~{\textbackslash}~\\
~~~~~~~~~~~~~~~~~~~~~/~~~~|~~~~{\textbackslash}~~~~~~~~~~~~~~~~~~~~~~|~\\
~~~~~~~~~~~~~~~~~~~~/~~~~~|~~~~~{\textbackslash}~~~~~~~~~~~~~~~~~~~~~|~\\
~~~~~~~~~~~~~~~~~~~/~~~~~~|~~~~~~{\textbackslash}~~~~~~~~~~~~~~~~~~~~|~\\
~~~~~~~~~~~~~~~~~~-{}-{}-~~~~-{}-{}-~~~~-{}-{}-~~~~~~~~~~~~~~~~~~~|~\\
Level~2~~~~~~~~3~|~D~|~4|~E~|~~|~F~|~5~~~~~~~~~~~~~~~~|~\\
~~~~~~~~~~~~~~~~~~-{}-{}-~~~~-{}-{}-~~~~-{}-{}-~~~~~~~~~~~~~~~~~~~|~\\
~~~~~~~~~~~~~~~~~~~{\textbackslash}~~{\textbackslash}~{\_}~/~~~~~~~|~~~~~~~~~~~~~~~~~~~|~\\
~~~~~~~~~~~~~~~~~~~~{\textbackslash}~~~~/~{\textbackslash}~{\_}~~~~|~~~~~~~~~~~~~~~~~~~|~\\
~~~~~~~~~~~~~~~~~~~~~{\textbackslash}~~/~~~~~~{\textbackslash}~~|~~~~~~~~~~~~~~~~~~~|~\\
~~~~~~~~~~~~~~~~~~~~~~-{}-{}-~~~~~~-{}-{}-~~~~~~~~~~~~~~~~~~~~|~\\
Level~1~~~~~~~~~~~~1~|~B~|~~~~|~C~|~2~~~~~~~~~~~~~~~~~|~\\
~~~~~~~~~~~~~~~~~~~~~~-{}-{}-~~~~~~-{}-{}-~~~~~~~~~~~~~~~~~~~~|~\\
~~~~~~~~~~~~~~~~~~~~~~~~{\textbackslash}~~~~~~/~~~~~~~~~~~~~~~~~~~~~~|~\\
~~~~~~~~~~~~~~~~~~~~~~~~~{\textbackslash}~~~~/~~~~~~~~~~~~~~~~~~~~~~{\textbackslash}~/~\\
~~~~~~~~~~~~~~~~~~~~~~~~~~~-{}-{}-~\\
Level~0~~~~~~~~~~~~~~~~~0~|~A~|~~~~~~~~~~~~~~~~(more~specialized)~\\
~~~~~~~~~~~~~~~~~~~~~~~~~~~-{}-{}-
}\end{quote}
\end{quote}

The linearizations of O,D,E and F are trivial:
\begin{quote}
\begin{quote}{\ttfamily \raggedright \noindent
L{[}O]~=~O~\\
L{[}D]~=~D~O~\\
L{[}E]~=~E~O~\\
L{[}F]~=~F~O
}\end{quote}
\end{quote}

The linearization of B can be computed as
\begin{quote}
\begin{quote}{\ttfamily \raggedright \noindent
L{[}B]~=~B~+~merge(DO,~EO,~DE)
}\end{quote}
\end{quote}

We see that D is a good head, therefore we take it and we are reduced to
compute \texttt{merge(O,EO,E)}.  Now O is not a good head, since it is in the
tail of the sequence EO.  In this case the rule says that we have to
skip to the next sequence.  Then we see that E is a good head; we take
it and we are reduced to compute \texttt{merge(O,O)} which gives O. Therefore
\begin{quote}
\begin{quote}{\ttfamily \raggedright \noindent
L{[}B]~=~~B~D~E~O
}\end{quote}
\end{quote}

Using the same procedure one finds:
\begin{quote}
\begin{quote}{\ttfamily \raggedright \noindent
L{[}C]~=~C~+~merge(DO,FO,DF)~\\
~~~~~=~C~+~D~+~merge(O,FO,F)~\\
~~~~~=~C~+~D~+~F~+~merge(O,O)~\\
~~~~~=~C~D~F~O
}\end{quote}
\end{quote}

Now we can compute:
\begin{quote}
\begin{quote}{\ttfamily \raggedright \noindent
L{[}A]~=~A~+~merge(BDEO,CDFO,BC)~\\
~~~~~=~A~+~B~+~merge(DEO,CDFO,C)~\\
~~~~~=~A~+~B~+~C~+~merge(DEO,DFO)~\\
~~~~~=~A~+~B~+~C~+~D~+~merge(EO,FO)~\\
~~~~~=~A~+~B~+~C~+~D~+~E~+~merge(O,FO)~\\
~~~~~=~A~+~B~+~C~+~D~+~E~+~F~+~merge(O,O)~\\
~~~~~=~A~B~C~D~E~F~O
}\end{quote}
\end{quote}

In this example, the linearization is ordered in a pretty nice way
according to the inheritance level, in the sense that lower levels (i.e.
more specialized classes) have higher precedence (see the inheritance
graph).  However, this is not the general case.

I leave as an exercise for the reader to compute the linearization for
my second example:
\begin{quote}
\begin{verbatim}>>> O = object
>>> class F(O): pass
>>> class E(O): pass
>>> class D(O): pass
>>> class C(D,F): pass
>>> class B(E,D): pass
>>> class A(B,C): pass\end{verbatim}
\end{quote}

The only difference with the previous example is the change B(D,E) -{}-{\textgreater}
B(E,D); however even such a little modification completely changes the
ordering of the hierarchy
\begin{quote}
\begin{quote}{\ttfamily \raggedright \noindent
~~~~~~~~~~~~~~~~~~~~~~~~~~~6~\\
~~~~~~~~~~~~~~~~~~~~~~~~~~-{}-{}-~\\
Level~3~~~~~~~~~~~~~~~~~~|~O~|~\\
~~~~~~~~~~~~~~~~~~~~~~~/~~-{}-{}-~~{\textbackslash}~\\
~~~~~~~~~~~~~~~~~~~~~~/~~~~|~~~~{\textbackslash}~\\
~~~~~~~~~~~~~~~~~~~~~/~~~~~|~~~~~{\textbackslash}~\\
~~~~~~~~~~~~~~~~~~~~/~~~~~~|~~~~~~{\textbackslash}~\\
~~~~~~~~~~~~~~~~~~-{}-{}-~~~~~-{}-{}-~~~~-{}-{}-~\\
Level~2~~~~~~~~2~|~E~|~4~|~D~|~~|~F~|~5~\\
~~~~~~~~~~~~~~~~~~-{}-{}-~~~~~-{}-{}-~~~~-{}-{}-~\\
~~~~~~~~~~~~~~~~~~~{\textbackslash}~~~~~~/~{\textbackslash}~~~~~/~\\
~~~~~~~~~~~~~~~~~~~~{\textbackslash}~~~~/~~~{\textbackslash}~~~/~\\
~~~~~~~~~~~~~~~~~~~~~{\textbackslash}~~/~~~~~{\textbackslash}~/~\\
~~~~~~~~~~~~~~~~~~~~~~-{}-{}-~~~~~-{}-{}-~\\
Level~1~~~~~~~~~~~~1~|~B~|~~~|~C~|~3~\\
~~~~~~~~~~~~~~~~~~~~~~-{}-{}-~~~~~-{}-{}-~\\
~~~~~~~~~~~~~~~~~~~~~~~{\textbackslash}~~~~~~~/~\\
~~~~~~~~~~~~~~~~~~~~~~~~{\textbackslash}~~~~~/~\\
~~~~~~~~~~~~~~~~~~~~~~~~~~-{}-{}-~\\
Level~0~~~~~~~~~~~~~~~~0~|~A~|~\\
~~~~~~~~~~~~~~~~~~~~~~~~~~-{}-{}-
}\end{quote}
\end{quote}

Notice that the class E, which is in the second level of the hierarchy,
precedes the class C, which is in the first level of the hierarchy, i.e.
E is more specialized than C, even if it is in a higher level.

A lazy programmer can obtain the MRO directly from Python 2.2, since in
this case it coincides with the Python 2.3 linearization.  It is enough
to invoke the .mro() method of class A:
\begin{quote}
\begin{verbatim}>>> A.mro()
(<class '__main__.A'>, <class '__main__.B'>, <class '__main__.E'>,
<class '__main__.C'>, <class '__main__.D'>, <class '__main__.F'>,
<type 'object'>)\end{verbatim}
\end{quote}

Finally, let me consider the example discussed in the first section,
involving a serious order disagreement.  In this case, it is
straightforward to compute the linearizations of O, X, Y, A and B:
\begin{quote}
\begin{quote}{\ttfamily \raggedright \noindent
L{[}O]~=~0~\\
L{[}X]~=~X~O~\\
L{[}Y]~=~Y~O~\\
L{[}A]~=~A~X~Y~O~\\
L{[}B]~=~B~Y~X~O
}\end{quote}
\end{quote}

However, it is impossible to compute the linearization for a class C
that inherits from A and B:
\begin{quote}
\begin{quote}{\ttfamily \raggedright \noindent
L{[}C]~=~C~+~merge(AXYO,~BYXO,~AB)~\\
~~~~~=~C~+~A~+~merge(XYO,~BYXO,~B)~\\
~~~~~=~C~+~A~+~B~+~merge(XYO,~YXO)
}\end{quote}
\end{quote}

At this point we cannot merge the lists XYO and YXO, since X is in the
tail of YXO whereas Y is in the tail of XYO:  therefore there are no
good heads and the C3 algorithm stops.  Python 2.3 raises an error and
refuses to create the class C.


%___________________________________________________________________________
\hspace*{\fill}\hrulefill\hspace*{\fill}

\begin{quote}
\begin{quote}{\ttfamily \raggedright \noindent
~~~~~~~~~~~~~~~~~~~~~~{\_}{\_}~\\
~~~~({\textbackslash}~~~.-.~~~.-.~~~/{\_}")~\\
~~~~~{\textbackslash}{\textbackslash}{\_}//{\textasciicircum}{\textbackslash}{\textbackslash}{\_}//{\textasciicircum}{\textbackslash}{\textbackslash}{\_}//~\\
jgs~~~`"`~~~`"`~~~`"`
}\end{quote}
\end{quote}


%___________________________________________________________________________

\hypertarget{bad-method-resolution-orders}{}
\pdfbookmark[0]{Bad Method Resolution Orders}{bad-method-resolution-orders}
\section*{Bad Method Resolution Orders}

A MRO is \emph{bad} when it breaks such fundamental properties as local
precedence ordering and monotonicity.  In this section, I will show
that both the MRO for classic classes and the MRO for new style classes
in Python 2.2 are bad.

It is easier to start with the local precedence ordering.  Consider the
following example:
\begin{quote}
\begin{verbatim}>>> F=type('Food',(),{'remember2buy':'spam'})
>>> E=type('Eggs',(F,),{'remember2buy':'eggs'})
>>> G=type('GoodFood',(F,E),{}) # under Python 2.3 this is an error!\end{verbatim}
\end{quote}

with inheritance diagram
\begin{quote}
\begin{quote}{\ttfamily \raggedright \noindent
~~~~~~~~~~~~~O~\\
~~~~~~~~~~~~~|~\\
(buy~spam)~~~F~\\
~~~~~~~~~~~~~|~{\textbackslash}~\\
~~~~~~~~~~~~~|~E~~~(buy~eggs)~\\
~~~~~~~~~~~~~|~/~\\
~~~~~~~~~~~~~G~\\
~\\
~~~~~~(buy~eggs~or~spam~?)
}\end{quote}
\end{quote}

We see that class G inherits from F and E, with F \emph{before} E:  therefore
we would expect the attribute \emph{G.remember2buy} to be inherited by
\emph{F.rembermer2buy} and not by \emph{E.remember2buy}:  nevertheless Python 2.2
gives
\begin{quote}
\begin{verbatim}>>> G.remember2buy
'eggs'\end{verbatim}
\end{quote}

This is a breaking of local precedence ordering since the order in the
local precedence list, i.e. the list of the parents of G, is not
preserved in the Python 2.2 linearization of G:
\begin{quote}
\begin{quote}{\ttfamily \raggedright \noindent
L{[}G,P22]=~G~E~F~object~~~{\#}~F~*follows*~E
}\end{quote}
\end{quote}

One could argue that the reason why F follows E in the Python 2.2
linearization is that F is less specialized than E, since F is the
superclass of E; nevertheless the breaking of local precedence ordering
is quite non-intuitive and error prone.  This is particularly true since
it is a different from old style classes:
\begin{quote}
\begin{verbatim}>>> class F: remember2buy='spam'
>>> class E(F): remember2buy='eggs'
>>> class G(F,E): pass
>>> G.remember2buy
'spam'\end{verbatim}
\end{quote}

In this case the MRO is GFEF and the local precedence ordering is
preserved.

As a general rule, hierarchies such as the previous one should be
avoided, since it is unclear if F should override E or viceversa.
Python 2.3 solves the ambiguity by raising an exception in the creation
of class G, effectively stopping the programmer from generating
ambiguous hierarchies.  The reason for that is that the C3 algorithm
fails when the merge
\begin{quote}
\begin{quote}{\ttfamily \raggedright \noindent
merge(FO,EFO,FE)
}\end{quote}
\end{quote}

cannot be computed, because F is in the tail of EFO and E is in the tail
of FE.

The real solution is to design a non-ambiguous hierarchy, i.e. to derive
G from E and F (the more specific first) and not from F and E; in this
case the MRO is GEF without any doubt.
\begin{quote}
\begin{quote}{\ttfamily \raggedright \noindent
~~~~~~~~~~~O~\\
~~~~~~~~~~~|~\\
~~~~~~~~~~~F~(spam)~\\
~~~~~~~~~/~|~\\
(eggs)~~~E~|~\\
~~~~~~~~~{\textbackslash}~|~\\
~~~~~~~~~~~G~\\
~~~~~~~~~~~~~(eggs,~no~doubt)
}\end{quote}
\end{quote}

Python 2.3 forces the programmer to write good hierarchies (or, at
least, less error-prone ones).

On a related note, let me point out that the Python 2.3 algorithm is
smart enough to recognize obvious mistakes, as the duplication of
classes in the list of parents:
\begin{quote}
\begin{verbatim}>>> class A(object): pass
>>> class C(A,A): pass # error
Traceback (most recent call last):
  File "<stdin>", line 1, in ?
TypeError: duplicate base class A\end{verbatim}
\end{quote}

Python 2.2 (both for classic classes and new style classes) in this
situation, would not raise any exception.

Finally, I would like to point out two lessons we have learned from this
example:
\newcounter{listcnt2}
\begin{list}{\arabic{listcnt2}.}
{
\usecounter{listcnt2}
\setlength{\rightmargin}{\leftmargin}
}
\item {} 
despite the name, the MRO determines the resolution order of
attributes, not only of methods;

\item {} 
the default food for Pythonistas is spam !  (but you already knew
that ;-)

\end{list}


%___________________________________________________________________________
\hspace*{\fill}\hrulefill\hspace*{\fill}

\begin{quote}
\begin{quote}{\ttfamily \raggedright \noindent
~~~~~~~~~~~~~~~~~~~~~~{\_}{\_}~\\
~~~~({\textbackslash}~~~.-.~~~.-.~~~/{\_}")~\\
~~~~~{\textbackslash}{\textbackslash}{\_}//{\textasciicircum}{\textbackslash}{\textbackslash}{\_}//{\textasciicircum}{\textbackslash}{\textbackslash}{\_}//~\\
jgs~~~`"`~~~`"`~~~`"`
}\end{quote}
\end{quote}

Having discussed the issue of local precedence ordering, let me now
consider the issue of monotonicity.  My goal is to show that neither the
MRO for classic classes nor that for Python 2.2 new style classes is
monotonic.

To prove that the MRO for classic classes is non-monotonic is rather
trivial, it is enough to look at the diamond diagram:
\begin{quote}
\begin{quote}{\ttfamily \raggedright \noindent
~~~C~\\
~~/~{\textbackslash}~\\
~/~~~{\textbackslash}~\\
A~~~~~B~\\
~{\textbackslash}~~~/~\\
~~{\textbackslash}~/~\\
~~~D
}\end{quote}
\end{quote}

One easily discerns the inconsistency:
\begin{quote}
\begin{quote}{\ttfamily \raggedright \noindent
L{[}B,P21]~=~B~C~~~~~~~~{\#}~B~precedes~C~:~B's~methods~win~\\
L{[}D,P21]~=~D~A~C~B~C~~{\#}~B~follows~C~~:~C's~methods~win!
}\end{quote}
\end{quote}

On the other hand, there are no problems with the Python 2.2 and 2.3
MROs, they give both
\begin{quote}
\begin{quote}{\ttfamily \raggedright \noindent
L{[}D]~=~D~A~B~C
}\end{quote}
\end{quote}

Guido points out in his essay [\hyperlink{id6}{3}] that the classic MRO is not so bad in
practice, since one can typically avoids diamonds for classic classes.
But all new style classes inherit from \texttt{object}, therefore diamonds are
unavoidable and inconsistencies shows up in every multiple inheritance
graph.

The MRO of Python 2.2 makes breaking monotonicity difficult, but not
impossible.  The following example, originally provided by Samuele
Pedroni, shows that the MRO of Python 2.2 is non-monotonic:
\begin{quote}
\begin{verbatim}>>> class A(object): pass
>>> class B(object): pass
>>> class C(object): pass
>>> class D(object): pass
>>> class E(object): pass
>>> class K1(A,B,C): pass
>>> class K2(D,B,E): pass
>>> class K3(D,A):   pass
>>> class Z(K1,K2,K3): pass\end{verbatim}
\end{quote}

Here are the linearizations according to the C3 MRO (the reader should
verify these linearizations as an exercise and draw the inheritance
diagram ;-)
\begin{quote}
\begin{quote}{\ttfamily \raggedright \noindent
L{[}A]~=~A~O~\\
L{[}B]~=~B~O~\\
L{[}C]~=~C~O~\\
L{[}D]~=~D~O~\\
L{[}E]~=~E~O~\\
L{[}K1]=~K1~A~B~C~O~\\
L{[}K2]=~K2~D~B~E~O~\\
L{[}K3]=~K3~D~A~O~\\
L{[}Z]~=~Z~K1~K2~K3~D~A~B~C~E~O
}\end{quote}
\end{quote}

Python 2.2 gives exactly the same linearizations for A, B, C, D, E, K1,
K2 and K3, but a different linearization for Z:
\begin{quote}
\begin{quote}{\ttfamily \raggedright \noindent
L{[}Z,P22]~=~Z~K1~K3~A~K2~D~B~C~E~O
}\end{quote}
\end{quote}

It is clear that this linearization is \emph{wrong}, since A comes before D
whereas in the linearization of K3 A comes \emph{after} D. In other words, in
K3 methods derived by D override methods derived by A, but in Z, which
still is a subclass of K3, methods derived by A override methods derived
by D!  This is a violation of monotonicity.  Moreover, the Python 2.2
linearization of Z is also inconsistent with local precedence ordering,
since the local precedence list of the class Z is {[}K1, K2, K3] (K2
precedes K3), whereas in the linearization of Z K2 \emph{follows} K3.  These
problems explain why the 2.2 rule has been dismissed in favor of the C3
rule.


%___________________________________________________________________________
\hspace*{\fill}\hrulefill\hspace*{\fill}

\begin{quote}
\begin{quote}{\ttfamily \raggedright \noindent
~~~~~~~~~~~~~~~~~~~~~~~~~~~~~~~~~~~~~~~~~~~~~~~~~~~~~~~~~{\_}{\_}~\\
~~~({\textbackslash}~~~.-.~~~.-.~~~.-.~~~.-.~~~.-.~~~.-.~~~.-.~~~.-.~~~/{\_}")~\\
~~~~{\textbackslash}{\textbackslash}{\_}//{\textasciicircum}{\textbackslash}{\textbackslash}{\_}//{\textasciicircum}{\textbackslash}{\textbackslash}{\_}//{\textasciicircum}{\textbackslash}{\textbackslash}{\_}//{\textasciicircum}{\textbackslash}{\textbackslash}{\_}//{\textasciicircum}{\textbackslash}{\textbackslash}{\_}//{\textasciicircum}{\textbackslash}{\textbackslash}{\_}//{\textasciicircum}{\textbackslash}{\textbackslash}{\_}//{\textasciicircum}{\textbackslash}{\textbackslash}{\_}//~\\
jgs~~`"`~~~`"`~~~`"`~~~`"`~~~`"`~~~`"`~~~`"`~~~`"`~~~`"`
}\end{quote}
\end{quote}


%___________________________________________________________________________

\hypertarget{the-end}{}
\pdfbookmark[0]{The end}{the-end}
\section*{The end}

This section is for the impatient reader, who skipped all the previous
sections and jumped immediately to the end.  This section is for the
lazy programmer too, who didn't want to exercise her/his brain.
Finally, it is for the programmer with some hubris, otherwise s/he would
not be reading a paper on the C3 method resolution order in multiple
inheritance hierarchies ;-) These three virtues taken all together (and
\emph{not} separately) deserve a prize:  the prize is a short Python 2.2
script that allows you to compute the 2.3 MRO without risk to your
brain.  Simply change the last line to play with the various examples I
have discussed in this paper.
\begin{quote}
\begin{quote}{\ttfamily \raggedright \noindent
{\#}<mro.py>~\\
~\\
"{}"{}"C3~algorithm~by~Samuele~Pedroni~(with~readability~enhanced~by~me)."{}"{}"~\\
~\\
class~{\_}{\_}metaclass{\_}{\_}(type):~\\
~~~~"All~classes~are~metamagically~modified~to~be~nicely~printed"~\\
~~~~{\_}{\_}repr{\_}{\_}~=~lambda~cls:~cls.{\_}{\_}name{\_}{\_}~\\
~\\
class~ex{\_}2:~\\
~~~~"Serious~order~disagreement"~{\#}From~Guido~\\
~~~~class~O:~pass~\\
~~~~class~X(O):~pass~\\
~~~~class~Y(O):~pass~\\
~~~~class~A(X,Y):~pass~\\
~~~~class~B(Y,X):~pass~\\
~~~~try:~\\
~~~~~~~~class~Z(A,B):~pass~{\#}creates~Z(A,B)~in~Python~2.2~\\
~~~~except~TypeError:~\\
~~~~~~~~pass~{\#}~Z(A,B)~cannot~be~created~in~Python~2.3~\\
~\\
class~ex{\_}5:~\\
~~~~"My~first~example"~\\
~~~~class~O:~pass~\\
~~~~class~F(O):~pass~\\
~~~~class~E(O):~pass~\\
~~~~class~D(O):~pass~\\
~~~~class~C(D,F):~pass~\\
~~~~class~B(D,E):~pass~\\
~~~~class~A(B,C):~pass~\\
~\\
class~ex{\_}6:~\\
~~~~"My~second~example"~\\
~~~~class~O:~pass~\\
~~~~class~F(O):~pass~\\
~~~~class~E(O):~pass~\\
~~~~class~D(O):~pass~\\
~~~~class~C(D,F):~pass~\\
~~~~class~B(E,D):~pass~\\
~~~~class~A(B,C):~pass~\\
~\\
class~ex{\_}9:~\\
~~~~"Difference~between~Python~2.2~MRO~and~C3"~{\#}From~Samuele~\\
~~~~class~O:~pass~\\
~~~~class~A(O):~pass~\\
~~~~class~B(O):~pass~\\
~~~~class~C(O):~pass~\\
~~~~class~D(O):~pass~\\
~~~~class~E(O):~pass~\\
~~~~class~K1(A,B,C):~pass~\\
~~~~class~K2(D,B,E):~pass~\\
~~~~class~K3(D,A):~pass~\\
~~~~class~Z(K1,K2,K3):~pass~\\
~\\
def~merge(seqs):~\\
~~~~print~'{\textbackslash}n{\textbackslash}nCPL{[}{\%}s]={\%}s'~{\%}~(seqs{[}0]{[}0],seqs),~\\
~~~~res~=~{[}];~i=0~\\
~~~~while~1:~\\
~~~~~~nonemptyseqs={[}seq~for~seq~in~seqs~if~seq]~\\
~~~~~~if~not~nonemptyseqs:~return~res~\\
~~~~~~i+=1;~print~'{\textbackslash}n',i,'round:~candidates...',~\\
~~~~~~for~seq~in~nonemptyseqs:~{\#}~find~merge~candidates~among~seq~heads~\\
~~~~~~~~~~cand~=~seq{[}0];~print~'~',cand,~\\
~~~~~~~~~~nothead={[}s~for~s~in~nonemptyseqs~if~cand~in~s{[}1:]]~\\
~~~~~~~~~~if~nothead:~cand=None~{\#}reject~candidate~\\
~~~~~~~~~~else:~break~\\
~~~~~~if~not~cand:~raise~"Inconsistent~hierarchy"~\\
~~~~~~res.append(cand)~\\
~~~~~~for~seq~in~nonemptyseqs:~{\#}~remove~cand~\\
~~~~~~~~~~if~seq{[}0]~==~cand:~del~seq{[}0]~\\
~\\
def~mro(C):~\\
~~~~"Compute~the~class~precedence~list~(mro)~according~to~C3"~\\
~~~~return~merge({[}{[}C]]+map(mro,C.{\_}{\_}bases{\_}{\_})+{[}list(C.{\_}{\_}bases{\_}{\_})])~\\
~\\
def~print{\_}mro(C):~\\
~~~~print~'{\textbackslash}nMRO{[}{\%}s]={\%}s'~{\%}~(C,mro(C))~\\
~~~~print~'{\textbackslash}nP22~MRO{[}{\%}s]={\%}s'~{\%}~(C,C.mro())~\\
~\\
print{\_}mro(ex{\_}9.Z)~\\
~\\
{\#}</mro.py>
}\end{quote}
\end{quote}

That's all folks,
\begin{quote}

enjoy !
\end{quote}


%___________________________________________________________________________
\hspace*{\fill}\hrulefill\hspace*{\fill}

\begin{quote}
\begin{quote}{\ttfamily \raggedright \noindent
~~~~{\_}{\_}~\\
~~~("{\_}{\textbackslash}~~~.-.~~~.-.~~~.-.~~~.-.~~~.-.~~~.-.~~~.-.~~~.-.~~~/)~\\
~~~~~~{\textbackslash}{\textbackslash}{\_}//{\textasciicircum}{\textbackslash}{\textbackslash}{\_}//{\textasciicircum}{\textbackslash}{\textbackslash}{\_}//{\textasciicircum}{\textbackslash}{\textbackslash}{\_}//{\textasciicircum}{\textbackslash}{\textbackslash}{\_}//{\textasciicircum}{\textbackslash}{\textbackslash}{\_}//{\textasciicircum}{\textbackslash}{\textbackslash}{\_}//{\textasciicircum}{\textbackslash}{\textbackslash}{\_}//{\textasciicircum}{\textbackslash}{\textbackslash}{\_}//~\\
jgs~~~~`"`~~~`"`~~~`"`~~~`"`~~~`"`~~~`"`~~~`"`~~~`"`~~~`"`
}\end{quote}
\end{quote}


%___________________________________________________________________________

\hypertarget{resources}{}
\pdfbookmark[0]{Resources}{resources}
\section*{Resources}
\begin{figure}[b]\hypertarget{id4}[1]
The thread on python-dev started by Samuele Pedroni:
\href{http://mail.python.org/pipermail/python-dev/2002-October/029035.html}{http://mail.python.org/pipermail/python-dev/2002-October/029035.html}
\end{figure}
\begin{figure}[b]\hypertarget{id5}[2]
The paper \emph{A Monotonic Superclass Linearization for Dylan}:
\href{http://www.webcom.com/haahr/dylan/linearization-oopsla96.html}{http://www.webcom.com/haahr/dylan/linearization-oopsla96.html}
\end{figure}
\begin{figure}[b]\hypertarget{id6}[3]
Guido van Rossum's essay, \emph{Unifying types and classes in Python 2.2}:
\href{http://www.python.org/2.2.2/descrintro.html}{http://www.python.org/2.2.2/descrintro.html}
\end{figure}

\end{document}

