\documentclass[10pt,a4paper,english]{article}
\usepackage{babel}
\usepackage{ae}
\usepackage{aeguill}
\usepackage{shortvrb}
\usepackage[latin1]{inputenc}
\usepackage{tabularx}
\usepackage{longtable}
\setlength{\extrarowheight}{2pt}
\usepackage{amsmath}
\usepackage{graphicx}
\usepackage{color}
\usepackage{multirow}
\usepackage{ifthen}
\usepackage[colorlinks=true,linkcolor=blue,urlcolor=blue]{hyperref}
\usepackage[DIV12]{typearea}
%% generator Docutils: http://docutils.sourceforge.net/
\newlength{\admonitionwidth}
\setlength{\admonitionwidth}{0.9\textwidth}
\newlength{\docinfowidth}
\setlength{\docinfowidth}{0.9\textwidth}
\newlength{\locallinewidth}
\newcommand{\optionlistlabel}[1]{\bf #1 \hfill}
\newenvironment{optionlist}[1]
{\begin{list}{}
  {\setlength{\labelwidth}{#1}
   \setlength{\rightmargin}{1cm}
   \setlength{\leftmargin}{\rightmargin}
   \addtolength{\leftmargin}{\labelwidth}
   \addtolength{\leftmargin}{\labelsep}
   \renewcommand{\makelabel}{\optionlistlabel}}
}{\end{list}}
\newlength{\lineblockindentation}
\setlength{\lineblockindentation}{2.5em}
\newenvironment{lineblock}[1]
{\begin{list}{}
  {\setlength{\partopsep}{\parskip}
   \addtolength{\partopsep}{\baselineskip}
   \topsep0pt\itemsep0.15\baselineskip\parsep0pt
   \leftmargin#1}
 \raggedright}
{\end{list}}
% begin: floats for footnotes tweaking.
\setlength{\floatsep}{0.5em}
\setlength{\textfloatsep}{\fill}
\addtolength{\textfloatsep}{3em}
\renewcommand{\textfraction}{0.5}
\renewcommand{\topfraction}{0.5}
\renewcommand{\bottomfraction}{0.5}
\setcounter{totalnumber}{50}
\setcounter{topnumber}{50}
\setcounter{bottomnumber}{50}
% end floats for footnotes
% some commands, that could be overwritten in the style file.
\newcommand{\rubric}[1]{\subsection*{~\hfill {\it #1} \hfill ~}}
\newcommand{\titlereference}[1]{\textsl{#1}}
% end of "some commands"
% donot indent first line.
\setlength{\parindent}{0pt}
\setlength{\parskip}{5pt plus 2pt minus 1pt}

% sloppy
% ------
% Less strict (opposite to default fussy) space size between words. Therefore
% less hyphenation.
\sloppy

% fonts
% -----
% times for pdf generation, gives smaller pdf files.
%
% But in standard postscript fonts: courier and times/helvetica do not fit.
% Maybe use pslatex.
\usepackage{times} 

% pagestyle
\pagestyle{headings}

\title{Lectures on Advanced Python Programming}
\author{}
\date{}
\hypersetup{
pdftitle={Lectures on Advanced Python Programming}
}
\raggedbottom
\begin{document}
\maketitle


\setlength{\locallinewidth}{\linewidth}

\includegraphics{accu2005.ps}
\begin{quote}
\begin{description}
\item [Author:]
Michele Simionato


\item [Given:]
19 April 2005


\item [Revised:]
7 September 2005


\end{description}
\end{quote}
\hypertarget{contents}{}
\pdfbookmark[0]{Contents}{contents}
\subsubsection*{~\hfill Contents\hfill ~}
\begin{list}{}{}
\item {} \href{\#lecture-1-loops-i-e-iterators-generators}{Lecture 1: Loops (i.e. iterators {\&} generators)}
\begin{list}{}{}
\item {} \href{\#part-i-iterators}{Part I: iterators}
\begin{list}{}{}
\item {} \href{\#iterators-are-everywhere}{Iterators are everywhere}

\item {} \href{\#iterables-and-iterators}{Iterables and iterators}

\item {} \href{\#simpler-way-to-get-an-iterator}{Simpler way to get an iterator}

\item {} \href{\#sentinel-syntax-iter-callable-sentinel}{Sentinel syntax iter(callable, sentinel)}

\item {} \href{\#second-simpler-way-to-get-an-iterator-generator-expressions}{Second simpler way to get an iterator: generator-expressions}

\item {} \href{\#iteration-caveats}{Iteration caveats}

\end{list}

\item {} \href{\#part-ii-generators}{Part II: generators}
\begin{list}{}{}
\item {} \href{\#a-simple-recipe-skip-redundant}{A simple recipe: skip redundant}

\item {} \href{\#another-real-life-example-working-with-nested-structures}{Another real life example: working with nested structures}

\item {} \href{\#another-typical-use-case-for-generators-parsers}{Another typical use case for generators: parsers}

\item {} \href{\#other-kinds-of-iterables}{Other kinds of iterables}

\item {} \href{\#the-itertools-module}{The itertools module}

\item {} \href{\#anytrue}{anyTrue}

\item {} \href{\#chop}{chop}

\item {} \href{\#tee}{tee}

\item {} \href{\#grouping-and-sorting}{Grouping and sorting}

\end{list}

\end{list}

\item {} \href{\#lecture-2-objects-delegation-inheritance}{Lecture 2: Objects (delegation {\&} inheritance)}
\begin{list}{}{}
\item {} \href{\#part-i-delegation}{Part I: delegation}
\begin{list}{}{}
\item {} \href{\#accessing-simple-attributes}{Accessing simple attributes}

\item {} \href{\#accessing-methods}{Accessing methods}

\item {} \href{\#converting-functions-into-methods}{Converting functions into methods}

\item {} \href{\#hack-a-very-slick-adder}{Hack: a very slick adder}

\item {} \href{\#descriptor-protocol}{Descriptor Protocol}

\item {} \href{\#multilingual-attribute}{Multilingual attribute}

\item {} \href{\#another-use-case-for-properties-storing-users}{Another use case for properties: storing users}

\item {} \href{\#low-level-delegation-via-getattribute}{Low-level delegation via {\_}{\_}getattribute{\_}{\_}}

\item {} \href{\#traditional-delegation-via-getattr}{Traditional delegation via {\_}{\_}getattr{\_}{\_}}

\item {} \href{\#keyword-dictionaries-with-getattr-setattr}{Keyword dictionaries with {\_}{\_}getattr{\_}{\_}/{\_}{\_}setattr{\_}{\_}}

\item {} \href{\#delegation-to-special-methods-caveat}{Delegation to special methods caveat}

\end{list}

\item {} \href{\#part-ii-inheritance}{Part II: Inheritance}
\begin{list}{}{}
\item {} \href{\#why-you-need-to-know-about-mi-even-if-you-do-not-use-it}{Why you need to know about MI even if you do not use it}

\item {} \href{\#a-few-details-about-super-not-the-whole-truth}{A few details about \texttt{super} (not the whole truth)}

\item {} \href{\#subclassing-built-in-types-new-vs-init}{Subclassing built-in types; {\_}{\_}new{\_}{\_} vs. {\_}{\_}init{\_}{\_}}

\item {} \href{\#be-careful-when-using-new-with-mutable-types}{Be careful when using {\_}{\_}new{\_}{\_} with mutable types}

\end{list}

\end{list}

\item {} \href{\#lecture-3-magic-i-e-decorators-and-metaclasses}{Lecture 3: Magic (i.e. decorators and metaclasses)}
\begin{list}{}{}
\item {} \href{\#part-i-decorators}{Part I: decorators}
\begin{list}{}{}
\item {} \href{\#a-typical-decorator-traced}{A typical decorator: traced}

\item {} \href{\#a-decorator-factory-timed}{A decorator factory: Timed}

\item {} \href{\#a-powerful-decorator-pattern}{A powerful decorator pattern}

\item {} \href{\#a-deferred-decorator}{A deferred decorator}

\end{list}

\item {} \href{\#part-ii-metaclasses}{Part II: metaclasses}
\begin{list}{}{}
\item {} \href{\#rejuvenating-old-style-classes}{Rejuvenating old-style classes}

\item {} \href{\#a-typical-metaclass-example-metatracer}{A typical metaclass example: MetaTracer}

\item {} \href{\#real-life-example-check-overriding}{Real life example: check overriding}

\item {} \href{\#logfile}{LogFile}

\item {} \href{\#cooperative-hierarchies}{Cooperative hierarchies}

\item {} \href{\#metaclass-enhanced-modules}{Metaclass-enhanced modules}

\item {} \href{\#magic-properties}{Magic properties}

\item {} \href{\#hack-evil-properties}{Hack: evil properties}

\item {} \href{\#why-i-suggest-not-to-use-metaclasses-in-production-code}{Why I suggest \emph{not} to use metaclasses in production code}

\item {} \href{\#is-there-an-automatic-way-of-solving-the-conflict}{Is there an automatic way of solving the conflict?}

\end{list}

\end{list}

\end{list}



%___________________________________________________________________________

\hypertarget{lecture-1-loops-i-e-iterators-generators}{}
\pdfbookmark[0]{Lecture 1: Loops (i.e. iterators {\&} generators)}{lecture-1-loops-i-e-iterators-generators}
\section*{Lecture 1: Loops (i.e. iterators {\&} generators)}


%___________________________________________________________________________

\hypertarget{part-i-iterators}{}
\pdfbookmark[1]{Part I: iterators}{part-i-iterators}
\subsection*{Part I: iterators}


%___________________________________________________________________________

\hypertarget{iterators-are-everywhere}{}
\pdfbookmark[2]{Iterators are everywhere}{iterators-are-everywhere}
\subsubsection*{Iterators are everywhere}
\begin{verbatim}>>> for i in 1, 2, 3:
...     print i
1
2
3\end{verbatim}

The 'for' loop is using \emph{iterators} internally:
\begin{quote}{\ttfamily \raggedright \noindent
it~=~iter((1,2,3))~\\
while~True:~\\
~~~~try:~\\
~~~~~~~~print~it.next()~\\
~~~~except~StopIteration:~\\
~~~~~~~~break
}\end{quote}


%___________________________________________________________________________

\hypertarget{iterables-and-iterators}{}
\pdfbookmark[2]{Iterables and iterators}{iterables-and-iterators}
\subsubsection*{Iterables and iterators}

\emph{Iterable} = anything you can loop over = any sequence + any object with an {\_}{\_}iter{\_}{\_} method;

Not every sequence has an {\_}{\_}iter{\_}{\_} method:
\begin{verbatim}>>> "hello".__iter__()
Traceback (most recent call last):
  ...
AttributeError: 'str' object has no attribute '__iter__'\end{verbatim}

\emph{Iterator} = any object with a .next method and an {\_}{\_}iter{\_}{\_} method returning self


%___________________________________________________________________________

\hypertarget{simpler-way-to-get-an-iterator}{}
\pdfbookmark[2]{Simpler way to get an iterator}{simpler-way-to-get-an-iterator}
\subsubsection*{Simpler way to get an iterator}
\begin{verbatim}>>> it = iter("hello")
>>> it.next()
'h'
>>> it.next()
'e'
>>> it.next()
'l'
>>> it.next()
'l'
>>> it.next()
'o'
>>> it.next()
Traceback (most recent call last):
  ...
StopIteration\end{verbatim}


%___________________________________________________________________________

\hypertarget{sentinel-syntax-iter-callable-sentinel}{}
\pdfbookmark[2]{Sentinel syntax iter(callable, sentinel)}{sentinel-syntax-iter-callable-sentinel}
\subsubsection*{Sentinel syntax iter(callable, sentinel)}

Example:
\begin{quote}{\ttfamily \raggedright \noindent
{\$}~echo~-e~"value1{\textbackslash}nvalue2{\textbackslash}nEND{\textbackslash}n"~>~data.txt~\\
{\$}~python~-c~"print~list(iter(file('data.txt').readline,~'END{\textbackslash}n'))"~\\
{[}'value1{\textbackslash}n',~'value2{\textbackslash}n']
}\end{quote}

Beware of infinite iterators:
\begin{verbatim}>>> repeat = iter(lambda : "some value", "")
>>> repeat.next()
'some value'\end{verbatim}


%___________________________________________________________________________

\hypertarget{second-simpler-way-to-get-an-iterator-generator-expressions}{}
\pdfbookmark[2]{Second simpler way to get an iterator: generator-expressions}{second-simpler-way-to-get-an-iterator-generator-expressions}
\subsubsection*{Second simpler way to get an iterator: generator-expressions}
\begin{verbatim}>>> squares = (i*i for i in range(1,11)) 
>>> list(squares)
[1, 4, 9, 16, 25, 36, 49, 64, 81, 100]\end{verbatim}

Excessive parenthesis can be skipped, so use
\begin{verbatim}>>> dict((i, i*i) for i in range(1,11))
{1: 1, 2: 4, 3: 9, 4: 16, 5: 25, 6: 36, 7: 49, 8: 64, 9: 81, 10: 100}\end{verbatim}

instead of
\begin{verbatim}>>> dict([(i, i*i) for i in range(1,11)])
{1: 1, 2: 4, 3: 9, 4: 16, 5: 25, 6: 36, 7: 49, 8: 64, 9: 81, 10: 100}\end{verbatim}

(as usual, the most elegant version is the most efficient).


%___________________________________________________________________________

\hypertarget{iteration-caveats}{}
\pdfbookmark[2]{Iteration caveats}{iteration-caveats}
\subsubsection*{Iteration caveats}
\begin{verbatim}>>> ls = [i for i in (1,2,3)]
>>> i 
3\end{verbatim}
\begin{verbatim}>>> it = (j for j in (1,2,3))
>>> j 
Traceback (most recent call last):
  ...
NameError: name 'j' is not defined\end{verbatim}

A subtler example:
\begin{verbatim}>>> ls = [lambda :i for i in (1,2,3)]
>>> ls[0]()
3\end{verbatim}

instead
\begin{verbatim}>>> it = (lambda :i for i in (1,2,3))
>>> it.next()()
1
>>> it.next()()
2
>>> it.next()()
3\end{verbatim}

\emph{seems} to be working but it is not really the case:
\begin{verbatim}>>> it = (lambda :i for i in (1,2,3))
>>> f1 = it.next()
>>> f2 = it.next()
>>> f3 = it.next()
>>> f1()
3\end{verbatim}

The reason is that Python does LATE binding \emph{always}. The solution is ugly:
\begin{verbatim}>>> it = list(lambda i=i:i for i in (1,2,3))
>>> it[0]()
1
>>> it[1]()
2
>>> it[2]()
3\end{verbatim}


%___________________________________________________________________________

\hypertarget{part-ii-generators}{}
\pdfbookmark[1]{Part II: generators}{part-ii-generators}
\subsection*{Part II: generators}

Trivial example:
\begin{verbatim}>>> def gen123(): # "function" which returns an iterator over the values 1, 2, 3
...     yield 1
...     yield 2
...     yield 3
...
>>> it = gen123()
>>> it.next()
1
>>> it.next()
2
>>> it.next()
3
>>> it.next()
Traceback (most recent call last):
  ...
StopIteration\end{verbatim}

Real life example: using generators to generate HTML tables
\begin{quote}{\ttfamily \raggedright \noindent
{\#}<htmltable.py>~\\
~\\
def~HTMLTablegen(table):~\\
~~~~yield~"<table>"~\\
~~~~for~row~in~table:~\\
~~~~~~~yield~"<tr>"~\\
~~~~~~~for~col~in~row:~\\
~~~~~~~~~~~yield~"<td>{\%}s</td>"~{\%}~col~\\
~~~~~~~yield~"</tr>"~\\
~~~~yield~"</table>"~\\
~\\
def~test():~\\
~~~~return~"{\textbackslash}n".join(HTMLTablegen({[}{[}"Row",~"City"],~~\\
~~~~~~~~~~~~~~~~~~~~~~~{[}1,'London'],~{[}2,~'Oxford']]))~\\
~\\
if~{\_}{\_}name{\_}{\_}~==~"{\_}{\_}main{\_}{\_}":~{\#}~example~\\
~~~~print~test()~\\
~\\
{\#}</htmltable.py>
}\end{quote}
\begin{verbatim}>>> from htmltable import test
>>> print test()
<table>
<tr>
<td>Row</td>
<td>City</td>
</tr>
<tr>
<td>1</td>
<td>London</td>
</tr>
<tr>
<td>2</td>
<td>Oxford</td>
</tr>
</table>\end{verbatim}


%___________________________________________________________________________

\hypertarget{a-simple-recipe-skip-redundant}{}
\pdfbookmark[2]{A simple recipe: skip redundant}{a-simple-recipe-skip-redundant}
\subsubsection*{A simple recipe: skip redundant}

How to remove duplicates by keeping the order:
\begin{quote}{\ttfamily \raggedright \noindent
{\#}<skip{\_}redundant.py>~\\
~\\
def~skip{\_}redundant(iterable,~skipset=None):~\\
~~~"Redundant~items~are~repeated~items~or~items~in~the~original~skipset."~\\
~~~if~skipset~is~None:~skipset~=~set()~\\
~~~for~item~in~iterable:~\\
~~~~~~~if~item~not~in~skipset:~\\
~~~~~~~~~~~skipset.add(item)~\\
~~~~~~~~~~~yield~item~\\
~~~~~~~~~~\\
{\#}</skip{\_}redundant.py>
}\end{quote}
\begin{verbatim}>>> from skip_redundant import skip_redundant
>>> print list(skip_redundant("<hello, world>", skipset=set("<>")))
['h', 'e', 'l', 'o', ',', ' ', 'w', 'r', 'd']\end{verbatim}


%___________________________________________________________________________

\hypertarget{another-real-life-example-working-with-nested-structures}{}
\pdfbookmark[2]{Another real life example: working with nested structures}{another-real-life-example-working-with-nested-structures}
\subsubsection*{Another real life example: working with nested structures}
\begin{quote}{\ttfamily \raggedright \noindent
{\#}<walk.py>~\\
~\\
def~walk(iterable,~level=0):~\\
~~~~for~obj~in~iterable:~\\
~~~~~~~~if~not~hasattr(obj,~"{\_}{\_}iter{\_}{\_}"):~{\#}~atomic~object~\\
~~~~~~~~~~~~yield~obj,~level~\\
~~~~~~~~else:~{\#}~composed~object:~iterate~again~\\
~~~~~~~~~~~~for~subobj,~lvl~in~walk(obj,~level~+~1):~\\
~~~~~~~~~~~~~~~~yield~subobj,~lvl~\\
~\\
def~flatten(iterable):~\\
~~~~return~(obj~for~obj,~level~in~walk(iterable))~\\
~~~~~~~~\\
def~pprint(iterable):~\\
~~~~for~obj,~level~in~walk(iterable):~\\
~~~~~~~~print~"~"*level,~obj~\\
~~~~~~~~\\
{\#}</walk.py>
}\end{quote}
\begin{verbatim}>>> from walk import flatten, pprint
>>> nested_ls = [1,[2,[3,[[[4,5],6]]]],7]
>>> pprint(nested_ls)     
 1
  2
   3
      4
      5
     6
 7
>>> pprint(flatten(nested_ls))
 1
 2
 3
 4
 5
 6
 7\end{verbatim}


%___________________________________________________________________________

\hypertarget{another-typical-use-case-for-generators-parsers}{}
\pdfbookmark[2]{Another typical use case for generators: parsers}{another-typical-use-case-for-generators-parsers}
\subsubsection*{Another typical use case for generators: parsers}

A very stripped down parser for nested expressions
\begin{quote}{\ttfamily \raggedright \noindent
{\#}<sexpr2indent.py>~\\
"{}"{}"A~simple~s-expression~formatter."{}"{}"~\\
~\\
import~re~\\
~\\
def~parse(sexpr):~\\
~~~~position~=~0~\\
~~~~nesting{\_}level~=~0~\\
~~~~paren~=~re.compile(r"(?P<paren{\_}beg>{\textbackslash}()|(?P<paren{\_}end>{\textbackslash}))")~\\
~~~~while~True:~\\
~~~~~~~~match~=~paren.search(sexpr,~position)~\\
~~~~~~~~if~match:~\\
~~~~~~~~~~~~yield~nesting{\_}level,~sexpr{[}position:~match.start()]~\\
~~~~~~~~~~~~if~match.lastgroup~==~"paren{\_}beg":~\\
~~~~~~~~~~~~~~~~nesting{\_}level~+=~1~\\
~~~~~~~~~~~~elif~match.lastgroup~==~"paren{\_}end":~\\
~~~~~~~~~~~~~~~~nesting{\_}level~-=~1~\\
~~~~~~~~~~~~position~=~match.end()~\\
~~~~~~~~else:~\\
~~~~~~~~~~~~break~\\
~\\
def~sexpr{\_}indent(sexpr):~\\
~~~~for~nesting,~text~in~parse(sexpr.replace("{\textbackslash}n",~"{}")):~\\
~~~~~~~~if~text.strip():~~print~"~"*nesting,~text~\\
~\\
{\#}</sexpr2indent.py>
}\end{quote}
\begin{verbatim}>>> from sexpr2indent import sexpr_indent
>>> sexpr_indent("""\
... (html (head (title Example)) (body (h1 s-expr formatter example)
... (a (@ (href http://www.example.com)) A link)))""")
... #doctest: +NORMALIZE_WHITESPACE
  html
   head
    title Example
    body
    h1 s-expr formatter example
    a
     @
      href http://www.example.com
     A link\end{verbatim}


%___________________________________________________________________________

\hypertarget{other-kinds-of-iterables}{}
\pdfbookmark[2]{Other kinds of iterables}{other-kinds-of-iterables}
\subsubsection*{Other kinds of iterables}

The following class generates iterable which are not iterators:
\begin{quote}{\ttfamily \raggedright \noindent
{\#}<reiterable.py>~\\
~\\
class~ReIter(object):~\\
~~~~"A~re-iterable~object."~\\
~~~~def~{\_}{\_}iter{\_}{\_}(self):~\\
~~~~~~~~yield~1~\\
~~~~~~~~yield~2~\\
~~~~~~~~yield~3~\\
~\\
{\#}</reiterable.py>
}\end{quote}
\begin{verbatim}>>> from reiterable import ReIter
>>> rit = ReIter()
>>> list(rit)
[1, 2, 3]
>>> list(rit) # it is reiterable!
[1, 2, 3]\end{verbatim}


%___________________________________________________________________________

\hypertarget{the-itertools-module}{}
\pdfbookmark[2]{The itertools module}{the-itertools-module}
\subsubsection*{The itertools module}
\begin{quote}
\begin{itemize}
\item {} 
count({[}n]) -{}-{\textgreater} n, n+1, n+2, ...

\item {} 
cycle(p) -{}-{\textgreater} p0, p1, ... plast, p0, p1, ...

\item {} 
repeat(elem {[},n]) -{}-{\textgreater} elem, elem, elem, ... endlessly or up to n times

\item {} 
izip(p, q, ...) -{}-{\textgreater} (p{[}0], q{[}0]), (p{[}1], q{[}1]), ...

\item {} 
ifilter(pred, seq) -{}-{\textgreater} elements of seq where pred(elem) is True

\item {} 
ifilterfalse(pred, seq) -{}-{\textgreater} elements of seq where pred(elem) is False

\item {} 
islice(seq, {[}start,] stop {[}, step]) -{}-{\textgreater} elements from seq{[}start:stop:step]

\item {} 
imap(fun, p, q, ...) -{}-{\textgreater} fun(p0, q0), fun(p1, q1), ...

\item {} 
starmap(fun, seq) -{}-{\textgreater} fun(*seq{[}0]), fun(*seq{[}1]), ...

\item {} 
tee(it, n=2) -{}-{\textgreater} (it1, it2 , ... itn) splits one iterator into n

\item {} 
chain(p, q, ...) -{}-{\textgreater} p0, p1, ... plast, q0, q1, ...

\item {} 
takewhile(pred, seq) -{}-{\textgreater} seq{[}0], seq{[}1], until pred fails

\item {} 
dropwhile(pred, seq) -{}-{\textgreater} seq{[}n], seq{[}n+1], starting when pred fails

\item {} 
groupby(iterable{[}, keyfunc]) -{}-{\textgreater} sub-iterators grouped by value of keyfunc(v)

\end{itemize}
\end{quote}


%___________________________________________________________________________

\hypertarget{anytrue}{}
\pdfbookmark[2]{anyTrue}{anytrue}
\subsubsection*{anyTrue}
\begin{verbatim}>>> import itertools
>>> def anyTrue(predicate, iterable): 
...     return True in itertools.imap(predicate, iterable)
...
>>> fname = "picture.ps"
>>> anyTrue(fname.endswith, ".jpg .ps .ps".split())
True\end{verbatim}

AnyTrue does \emph{short-circuit}:
\begin{verbatim}>>> def is3(i):
...     print "i=%s" % i
...     return i == 3\end{verbatim}
\begin{verbatim}>>> anyTrue(is3, range(10))
i=0
i=1
i=2
i=3
True\end{verbatim}


%___________________________________________________________________________

\hypertarget{chop}{}
\pdfbookmark[2]{chop}{chop}
\subsubsection*{chop}

You want to chop an iterable in batches of a given size:
\begin{verbatim}>>> from chop import chop
>>> list(chop([1, 2, 3, 4], 2))
[[1, 2], [3, 4]]
>>> list(chop([1, 2, 3, 4, 5, 6, 7],3))
[[1, 2, 3], [4, 5, 6], [7]]\end{verbatim}

Here is a possible implementation:
\begin{quote}{\ttfamily \raggedright \noindent
{\#}<chop.py>~\\
~\\
{\#}~see~also~http://aspn.activestate.com/ASPN/Cookbook/Python/Recipe/303279~\\
~\\
import~itertools~\\
~\\
def~chop(iterable,~batchsize):~\\
~~~~it~=~iter(iterable)~\\
~~~~while~True:~\\
~~~~~~~~batch~=~list(itertools.islice(it,~batchsize))~\\
~~~~~~~~if~batch:~yield~batch~\\
~~~~~~~~else:~break~\\
~\\
{\#}</chop.py>
}\end{quote}

For people thinking Python is too readable, here is a one-liner:
\begin{verbatim}>>> chop = lambda it, n : itertools.izip(*(iter(it),)*n)
...
>>> list(chop([1,2,3,4], 2))
[(1, 2), (3, 4)]\end{verbatim}


%___________________________________________________________________________

\hypertarget{tee}{}
\pdfbookmark[2]{tee}{tee}
\subsubsection*{tee}

To make copies of iterables; typically used in parsers:
\begin{verbatim}>>> from itertools import tee, chain, izip
>>> chars, prevs = tee("abc")
>>> prevs = chain([None], prevs)
>>> for char, prev in izip(chars, prevs):
...     print char, prev
...
a None
b a
c b\end{verbatim}


%___________________________________________________________________________

\hypertarget{grouping-and-sorting}{}
\pdfbookmark[2]{Grouping and sorting}{grouping-and-sorting}
\subsubsection*{Grouping and sorting}
\begin{verbatim}>>> from itertools import groupby
>>> from operator import itemgetter\end{verbatim}
\begin{verbatim}>>> NAME, AGE = 0, 1
>>> query_result = ("Smith", 34), ("Donaldson", 34), ("Lee", 22), ("Orr", 22)\end{verbatim}

Grouping together people of the same age:
\begin{verbatim}>>> for k, g in groupby(query_result, key=itemgetter(AGE)):
...     print k, list(g)
...
34 [('Smith', 34), ('Donaldson', 34)]
22 [('Lee', 22), ('Orr', 22)]\end{verbatim}

Sorting by name:
\begin{verbatim}>>> for tup in sorted(query_result, key=itemgetter(NAME)):
...     print tup
('Donaldson', 34)
('Lee', 22)
('Orr', 22)
('Smith', 34)\end{verbatim}


%___________________________________________________________________________

\hypertarget{lecture-2-objects-delegation-inheritance}{}
\pdfbookmark[0]{Lecture 2: Objects (delegation {\&} inheritance)}{lecture-2-objects-delegation-inheritance}
\section*{Lecture 2: Objects (delegation {\&} inheritance)}


%___________________________________________________________________________

\hypertarget{part-i-delegation}{}
\pdfbookmark[1]{Part I: delegation}{part-i-delegation}
\subsection*{Part I: delegation}

Understanding how attribute access works: internal delegation via \emph{descriptors}


%___________________________________________________________________________

\hypertarget{accessing-simple-attributes}{}
\pdfbookmark[2]{Accessing simple attributes}{accessing-simple-attributes}
\subsubsection*{Accessing simple attributes}
\begin{verbatim}>>> class C(object):
...     a = 2
...     def __init__(self, x):
...        self.x = x
...\end{verbatim}
\begin{verbatim}>>> c = C(1)
>>> c.x
1
>>> c.a
2\end{verbatim}

We are retrieving
\begin{verbatim}>>> c.__dict__["x"]
1\end{verbatim}

If there is nothing in c.{\_}{\_}dict{\_}{\_}, Python looks at C.{\_}{\_}dict{\_}{\_}:
\begin{verbatim}>>> print c.__dict__.get("a")
None\end{verbatim}
\begin{verbatim}>>> C.__dict__["a"]
2\end{verbatim}

If there is nothing in C.{\_}{\_}dict{\_}{\_}, Python looks at the superclasses according
to the MRO (see part II).


%___________________________________________________________________________

\hypertarget{accessing-methods}{}
\pdfbookmark[2]{Accessing methods}{accessing-methods}
\subsubsection*{Accessing methods}
\begin{verbatim}>>> c.__init__ #doctest: +ELLIPSIS
<bound method C.__init__ of <__main__.C object at 0x...>>\end{verbatim}

since {\_}{\_}init{\_}{\_} is not in c.{\_}{\_}dict{\_}{\_} Python looks in the class dictionary 
and finds
\begin{verbatim}>>> C.__dict__["__init__"] #doctest: +ELLIPSIS
<function __init__ at 0x...>\end{verbatim}

Then it magically converts the function into a method bound to the instance
``c''.

NOTE: this mechanism works for new-style classes only.

The old-style mechanism is less consistent and the attribute lookup of special
methods is special: (*)
\begin{verbatim}>>> class C(object): pass
>>> c = C()
>>> c.__str__ = lambda : "hello!"
>>> print c #doctest: +ELLIPSIS
<__main__.C object at ...>\end{verbatim}

whereas for old-style classes
\begin{verbatim}>>> class C: pass
>>> c = C()
>>> c.__str__ = lambda : "hello!"
>>> print c 
hello!\end{verbatim}

the special method is looked for in the instance dictionary too.

(*) modulo a very subtle difference for {\_}{\_}getattr{\_}{\_}-delegated special methods,
see later.


%___________________________________________________________________________

\hypertarget{converting-functions-into-methods}{}
\pdfbookmark[2]{Converting functions into methods}{converting-functions-into-methods}
\subsubsection*{Converting functions into methods}

It is possible to convert a function into a bound or unbound method
by invoking the \texttt{{\_}{\_}get{\_}{\_}} special method:
\begin{verbatim}>>> def f(x): pass
>>> f.__get__ #doctest: +ELLIPSIS
<method-wrapper object at 0x...>\end{verbatim}
\begin{verbatim}>>> class C(object): pass
...\end{verbatim}
\begin{verbatim}>>> def f(self): pass
...
>>> f.__get__(C(), C) #doctest: +ELLIPSIS
<bound method C.f of <__main__.C object at 0x...>>\end{verbatim}
\begin{verbatim}>>> f.__get__(None, C)
<unbound method C.f>\end{verbatim}

Functions are the simplest example of \emph{descriptors}.

Access to methods works since internally Python transforms
\begin{quote}

\texttt{c.{\_}{\_}init{\_}{\_} -> type(c).{\_}{\_}dict{\_}{\_}{[}'{\_}{\_}init{\_}{\_}'].{\_}{\_}get{\_}{\_}(c, type(c))}
\end{quote}

Note: not \emph{all} functions are descriptors:
\begin{verbatim}>>> from operator import add
>>> add.__get__
Traceback (most recent call last):
  ...
AttributeError: 'builtin_function_or_method' object has no attribute '__get__'\end{verbatim}


%___________________________________________________________________________

\hypertarget{hack-a-very-slick-adder}{}
\pdfbookmark[2]{Hack: a very slick adder}{hack-a-very-slick-adder}
\subsubsection*{Hack: a very slick adder}

The descriptor protocol can be (ab)used as a way to avoid the late binding
issue in for loops:
\begin{verbatim}>>> def add(x,y):
...     return x + y
>>> closures = [add.__get__(i) for i in range(10)]
>>> closures[5](1000)
1005\end{verbatim}

Notice: operator.add will not work.


%___________________________________________________________________________

\hypertarget{descriptor-protocol}{}
\pdfbookmark[2]{Descriptor Protocol}{descriptor-protocol}
\subsubsection*{Descriptor Protocol}

Everything at \href{http://users.rcn.com/python/download/Descriptor.htm}{http://users.rcn.com/python/download/Descriptor.htm}

Formally:
\begin{quote}{\ttfamily \raggedright \noindent
descr.{\_}{\_}get{\_}{\_}(self,~obj,~type=None)~-{}->~value~\\
descr.{\_}{\_}set{\_}{\_}(self,~obj,~value)~-{}->~None~\\
descr.{\_}{\_}delete{\_}{\_}(self,~obj)~-{}->~None
}\end{quote}

Examples of custom descriptors:
\begin{quote}{\ttfamily \raggedright \noindent
{\#}<descriptor.py>~\\
~\\
~\\
class~AttributeDescriptor(object):~\\
~~~def~{\_}{\_}get{\_}{\_}(self,~obj,~cls=None):~\\
~~~~~~~if~obj~is~None~and~cls~is~None:~\\
~~~~~~~~~~~raise~TypeError("{\_}{\_}get{\_}{\_}(None,~None)~is~invalid")~\\
~~~~~~~elif~obj~is~None:~\\
~~~~~~~~~~~return~self.get{\_}from{\_}class(cls)~\\
~~~~~~~else:~\\
~~~~~~~~~~~return~self.get{\_}from{\_}obj(obj)~\\
~~~def~get{\_}from{\_}class(self,~cls):~\\
~~~~~~~print~"Getting~{\%}s~from~{\%}s"~{\%}~(self,~cls)~\\
~~~def~get{\_}from{\_}obj(self,~obj):~\\
~~~~~~~print~"Getting~{\%}s~from~{\%}s"~{\%}~(self,~obj)~\\
~\\
~\\
class~Staticmethod(AttributeDescriptor):~\\
~~~def~{\_}{\_}init{\_}{\_}(self,~func):~\\
~~~~~~~self.func~=~func~\\
~~~def~get{\_}from{\_}class(self,~cls):~\\
~~~~~~~return~self.func~\\
~~~get{\_}from{\_}obj~=~get{\_}from{\_}class~\\
~\\
~\\
class~Classmethod(AttributeDescriptor):~\\
~~~def~{\_}{\_}init{\_}{\_}(self,~func):~\\
~~~~~~~self.func~=~func~\\
~~~def~get{\_}from{\_}class(self,~cls):~\\
~~~~~~~return~self.func.{\_}{\_}get{\_}{\_}(cls,~type(cls))~\\
~~~def~get{\_}from{\_}obj(self,~obj):~\\
~~~~~~~return~self.get{\_}from{\_}class(obj.{\_}{\_}class{\_}{\_})~\\
~\\
class~C(object):~\\
~~~s~=~Staticmethod(lambda~:~1)~\\
~~~c~=~Classmethod(lambda~cls~:~cls.{\_}{\_}name{\_}{\_})~\\
~\\
c~=~C()~\\
~\\
assert~C.s()~==~c.s()~==~1~\\
assert~C.c()~==~c.c()~==~"C"~\\
~\\
{\#}</descriptor.py>
}\end{quote}


%___________________________________________________________________________

\hypertarget{multilingual-attribute}{}
\pdfbookmark[2]{Multilingual attribute}{multilingual-attribute}
\subsubsection*{Multilingual attribute}

Inspirated by a question in the Italian Newsgroup:
\begin{quote}{\ttfamily \raggedright \noindent
{\#}<multilingual.py>~\\
~\\
import~sys~\\
from~descriptor~import~AttributeDescriptor~\\
~\\
class~MultilingualAttribute(AttributeDescriptor):~\\
~~~~"{}"{}"When~a~MultilingualAttribute~is~accessed,~you~get~the~translation~~\\
~~~~corresponding~to~the~currently~selected~language.~\\
~~~~"{}"{}"~\\
~~~~def~{\_}{\_}init{\_}{\_}(self,~**translations):~\\
~~~~~~~~self.trans~=~translations~\\
~~~~def~get{\_}from{\_}class(self,~cls):~\\
~~~~~~~~return~self.trans{[}getattr(cls,~"language",~None)~or~\\
~~~~~~~~~~~~~~~~~~~~~~~~~sys.modules{[}cls.{\_}{\_}module{\_}{\_}].language]~\\
~~~~def~get{\_}from{\_}obj(self,~obj):~\\
~~~~~~~~return~self.trans{[}getattr(obj,~"language",~None)~or~\\
~~~~~~~~~~~~~~~~~~~~~~~~~sys.modules{[}obj.{\_}{\_}class{\_}{\_}.{\_}{\_}module{\_}{\_}].language]~\\
~~~~~~\\
~\\
language~=~"en"~\\
~\\
{\#}~a~dummy~User~class~\\
class~DefaultUser(object):~\\
~~~~def~has{\_}permission(self):~\\
~~~~~~~~return~False~\\
~~~~\\
class~WebApplication(object):~\\
~~~~error{\_}msg~=~MultilingualAttribute(~\\
~~~~~~~~en="You~cannot~access~this~page",~\\
~~~~~~~~it="Questa~pagina~non~e'~accessibile",~\\
~~~~~~~~fr="Vous~ne~pouvez~pas~acceder~cette~page",)~\\
~~~~user~=~DefaultUser()~\\
~~~~def~{\_}{\_}init{\_}{\_}(self,~language=None):~\\
~~~~~~~~self.language~=~language~or~getattr(self.{\_}{\_}class{\_}{\_},~"language",~None)~\\
~~~~def~show{\_}page(self):~\\
~~~~~~~~if~not~self.user.has{\_}permission():~\\
~~~~~~~~~~~~return~self.error{\_}msg~\\
~\\
~\\
app~=~WebApplication()~\\
assert~app.show{\_}page()~==~"You~cannot~access~this~page"~\\
~\\
app.language~=~"fr"~\\
assert~app.show{\_}page()~==~"Vous~ne~pouvez~pas~acceder~cette~page"~\\
~\\
app.language~=~"it"~\\
assert~app.show{\_}page()~==~"Questa~pagina~non~e'~accessibile"~\\
~\\
app.language~=~"en"~\\
assert~app.show{\_}page()~==~"You~cannot~access~this~page"~\\
~\\
{\#}</multilingual.py>
}\end{quote}

The same can be done with properties:
\begin{quote}{\ttfamily \raggedright \noindent
{\#}<multilingualprop.py>~\\
~\\
language~=~"en"~\\
~\\
{\#}~a~dummy~User~class~\\
class~DefaultUser(object):~\\
~~~~def~has{\_}permission(self):~\\
~~~~~~~~return~False~\\
~~~~\\
def~multilingualProperty(**trans):~\\
~~~~def~get(self):~\\
~~~~~~~~return~trans{[}self.language]~\\
~~~~def~set(self,~value):~\\
~~~~~~~~trans{[}self.language]~=~value~~\\
~~~~return~property(get,~set)~\\
~\\
class~WebApplication(object):~\\
~~~~language~=~language~\\
~~~~error{\_}msg~=~multilingualProperty(~\\
~~~~~~~~en="You~cannot~access~this~page",~\\
~~~~~~~~it="Questa~pagina~non~e'~accessibile",~\\
~~~~~~~~fr="Vous~ne~pouvez~pas~acceder~cette~page",)~\\
~~~~user~=~DefaultUser()~\\
~~~~def~{\_}{\_}init{\_}{\_}(self,~language=None):~\\
~~~~~~~~if~language:~self.language~=~self.language~\\
~~~~def~show{\_}page(self):~\\
~~~~~~~~if~not~self.user.has{\_}permission():~\\
~~~~~~~~~~~~return~self.error{\_}msg~\\
~\\
{\#}</multilingualprop.py>
}\end{quote}

This also gives the possibility to set the error messages.

The difference with the descriptor approach
\begin{verbatim}>>> from multilingual import WebApplication
>>> app = WebApplication()
>>> print app.error_msg
You cannot access this page
>>> print WebApplication.error_msg
You cannot access this page\end{verbatim}

is that with properties there is no nice access from the class:
\begin{verbatim}>>> from multilingualprop import WebApplication
>>> WebApplication.error_msg #doctest: +ELLIPSIS
<property object at ...>\end{verbatim}


%___________________________________________________________________________

\hypertarget{another-use-case-for-properties-storing-users}{}
\pdfbookmark[2]{Another use case for properties: storing users}{another-use-case-for-properties-storing-users}
\subsubsection*{Another use case for properties: storing users}

Consider a library providing a simple User class:
\begin{quote}{\ttfamily \raggedright \noindent
{\#}<crypt{\_}user.py>~\\
~\\
class~User(object):~\\
~~~~def~{\_}{\_}init{\_}{\_}(self,~username,~password):~\\
~~~~~~~~self.username,~self.password~=~username,~password~\\
~\\
{\#}</crypt{\_}user.py>
}\end{quote}

The User objects are stored in a database as they are.
For security purpose, in a second version of the library it is
decided to crypt the password, so that only crypted passwords
are stored in the database. With properties, it is possible to
implement this functionality without changing the source code for 
the User class:
\begin{quote}{\ttfamily \raggedright \noindent
{\#}<crypt{\_}user.py>~\\
~\\
from~crypt~import~crypt~\\
~\\
def~cryptedAttribute(seed="x"):~\\
~~~~def~get(self):~\\
~~~~~~~~return~getattr(self,~"{\_}pw",~None)~\\
~~~~def~set(self,~value):~\\
~~~~~~~~self.{\_}pw~=~crypt(value,~seed)~\\
~~~~return~property(get,~set)~\\
~~~~\\
User.password~=~cryptedAttribute()
}\end{quote}

{\#}{\textless}/crypt{\_}user.py{\textgreater}
\begin{verbatim}>>> from crypt_user import User
>>> u = User("michele", "secret")
>>> print u.password
xxZREZpkHZpkI\end{verbatim}

Notice the property factory approach used here.


%___________________________________________________________________________

\hypertarget{low-level-delegation-via-getattribute}{}
\pdfbookmark[2]{Low-level delegation via {\_}{\_}getattribute{\_}{\_}}{low-level-delegation-via-getattribute}
\subsubsection*{Low-level delegation via {\_}{\_}getattribute{\_}{\_}}

Attribute access is managed by the{\_}{\_}getattribute{\_}{\_} special method:
\begin{quote}{\ttfamily \raggedright \noindent
{\#}<tracedaccess.py>~\\
~\\
class~TracedAccess(object):~\\
~~~~def~{\_}{\_}getattribute{\_}{\_}(self,~name):~\\
~~~~~~~~print~"Accessing~{\%}s"~{\%}~name~\\
~~~~~~~~return~object.{\_}{\_}getattribute{\_}{\_}(self,~name)~\\
~\\
~\\
class~C(TracedAccess):~\\
~~~~s~=~staticmethod(lambda~:~'staticmethod')~\\
~~~~c~=~classmethod(lambda~cls:~'classmethod')~\\
~~~~m~=~lambda~self:~'method'~\\
~~~~a~=~"hello"~\\
~\\
{\#}</tracedaccess.py>
}\end{quote}
\begin{verbatim}>>> from tracedaccess import C
>>> c = C()
>>> print c.s()
Accessing s
staticmethod
>>> print c.c()
Accessing c
classmethod
>>> print c.m()
Accessing m
method
>>> print c.a
Accessing a
hello
>>> print c.__init__ #doctest: +ELLIPSIS
Accessing __init__
<method-wrapper object at 0x...>
>>> try: c.x
... except AttributeError, e: print e
...
Accessing x
'C' object has no attribute 'x'\end{verbatim}
\begin{verbatim}>>> c.y = 'y'
>>> c.y
Accessing y
'y'\end{verbatim}

You are probably familiar with \texttt{{\_}{\_}getattr{\_}{\_}} which is similar 
to \texttt{{\_}{\_}getattribute{\_}{\_}}, but it is called \emph{only for missing attributes}.


%___________________________________________________________________________

\hypertarget{traditional-delegation-via-getattr}{}
\pdfbookmark[2]{Traditional delegation via {\_}{\_}getattr{\_}{\_}}{traditional-delegation-via-getattr}
\subsubsection*{Traditional delegation via {\_}{\_}getattr{\_}{\_}}

Realistic use case in ``object publishing'':
\begin{quote}{\ttfamily \raggedright \noindent
{\#}<webapp.py>~\\
~\\
class~WebApplication(object):~\\
~~~~def~{\_}{\_}getattr{\_}{\_}(self,~name):~\\
~~~~~~~~return~name.capitalize()~\\
~\\
~\\
app~=~WebApplication()~\\
~\\
assert~app.page1~==~'Page1'~\\
assert~app.page2~==~'Page2'~\\
~\\
{\#}</webapp.py>
}\end{quote}

Here is another use case in HTML generation:
\begin{quote}{\ttfamily \raggedright \noindent
{\#}<XMLtag.py>~\\
~\\
def~makeattr(dict{\_}or{\_}list{\_}of{\_}pairs):~\\
~~~~dic~=~dict(dict{\_}or{\_}list{\_}of{\_}pairs)~~\\
~~~~return~"~".join('{\%}s="{\%}s"'~{\%}~(k,~dic{[}k])~for~k~in~dic)~{\#}~simplistic~\\
~\\
class~XMLTag(object):~\\
~~~~def~{\_}{\_}getattr{\_}{\_}(self,~name):~\\
~~~~~~~~def~tag(value,~**attr):~\\
~~~~~~~~~~~~"{}"{}"value~can~be~a~string~or~a~sequence~of~strings."{}"{}"~\\
~~~~~~~~~~~~if~hasattr(value,~"{\_}{\_}iter{\_}{\_}"):~{\#}~is~iterable~\\
~~~~~~~~~~~~~~~~value~=~"~".join(value)~\\
~~~~~~~~~~~~return~"<{\%}s~{\%}s>{\%}s</{\%}s>"~{\%}~(name,~makeattr(attr),~value,~name)~\\
~~~~~~~~return~tag~\\
~\\
class~XMLShortTag(object):~\\
~~~~def~{\_}{\_}getattr{\_}{\_}(self,~name):~\\
~~~~~~~~def~tag(**attr):~\\
~~~~~~~~~~~~return~"<{\%}s~{\%}s~/>"~{\%}~(name,~makeattr(attr))~\\
~~~~~~~~return~tag~\\
~\\
tag~=~XMLTag()~\\
tg~=~XMLShortTag()~\\
~\\
{\#}</XMLtag.py>
}\end{quote}
\begin{verbatim}>>> from XMLtag import tag, tg
>>> print tag.a("example.com", href="http://www.example.com")
<a href="http://www.example.com">example.com</a>
>>> print tg.br(**{'class':"br_style"})
<br class="br_style" />\end{verbatim}


%___________________________________________________________________________

\hypertarget{keyword-dictionaries-with-getattr-setattr}{}
\pdfbookmark[2]{Keyword dictionaries with {\_}{\_}getattr{\_}{\_}/{\_}{\_}setattr{\_}{\_}}{keyword-dictionaries-with-getattr-setattr}
\subsubsection*{Keyword dictionaries with {\_}{\_}getattr{\_}{\_}/{\_}{\_}setattr{\_}{\_}}
\begin{quote}{\ttfamily \raggedright \noindent
{\#}<kwdict.py>~\\
~\\
class~kwdict(dict):~{\#}~or~UserDict,~to~make~it~to~work~with~Zope~\\
~~~~"{}"{}"A~typing~shortcut~used~in~place~of~a~keyword~dictionary."{}"{}"~\\
~~~~def~{\_}{\_}getattr{\_}{\_}(self,~name):~\\
~~~~~~~~return~self{[}name]~\\
~~~~def~{\_}{\_}setattr{\_}{\_}(self,~name,~value):~\\
~~~~~~~~self{[}name]~=~value~\\
~\\
{\#}</kwdict.py>
}\end{quote}

And now for a completely different solution:
\begin{quote}{\ttfamily \raggedright \noindent
{\#}<dictwrapper.py>~\\
~\\
class~DictWrapper(object):~~\\
~~~~def~{\_}{\_}init{\_}{\_}(self,~**kw):~\\
~~~~~~~~self.{\_}{\_}dict{\_}{\_}.update(kw)~\\
~\\
{\#}</dictwrapper.py>
}\end{quote}


%___________________________________________________________________________

\hypertarget{delegation-to-special-methods-caveat}{}
\pdfbookmark[2]{Delegation to special methods caveat}{delegation-to-special-methods-caveat}
\subsubsection*{Delegation to special methods caveat}
\begin{verbatim}>>> class ListWrapper(object):
...     def __init__(self, ls):
...         self._list = ls
...     def __getattr__(self, name):
...         if name == "__getitem__": # special method
...             return self._list.__getitem__
...         elif name == "reverse": # regular method
...             return self._list.reverse
...         else:
...             raise AttributeError("%r is not defined" % name)
... 
>>> lw = ListWrapper([0,1,2])
>>> print lw.x
Traceback (most recent call last):
  ...
AttributeError: 'x' is not defined\end{verbatim}
\begin{verbatim}>>> lw.reverse()
>>> print lw.__getitem__(0)
2
>>> print lw.__getitem__(1)
1
>>> print lw.__getitem__(2)
0
>>> print lw[0]
Traceback (most recent call last):
  ...
TypeError: unindexable object\end{verbatim}


%___________________________________________________________________________

\hypertarget{part-ii-inheritance}{}
\pdfbookmark[1]{Part II: Inheritance}{part-ii-inheritance}
\subsection*{Part II: Inheritance}

The major changes in inheritance from Python 2.1 to 2.2+ are:
\newcounter{listcnt1}
\begin{list}{\arabic{listcnt1}.}
{
\usecounter{listcnt1}
\setlength{\rightmargin}{\leftmargin}
}
\item {} 
you can subclass built-in types (as a consequence the constructor{\_}{\_}new{\_}{\_} 
has been exposed to the user, to help subclassing immutable types);

\item {} 
the Method Resolution Order (MRO) has changed;

\item {} 
now Python allows \emph{cooperative method calls}, i.e. we have \emph{super}.

\end{list}


%___________________________________________________________________________

\hypertarget{why-you-need-to-know-about-mi-even-if-you-do-not-use-it}{}
\pdfbookmark[2]{Why you need to know about MI even if you do not use it}{why-you-need-to-know-about-mi-even-if-you-do-not-use-it}
\subsubsection*{Why you need to know about MI even if you do not use it}

In principle, the last two changes are relevant only if you use multiple 
inheritance. If you use single inheritance only, you don't need \texttt{super}:
you can just name the superclass.
However, somebody else may want to use your class in a MI hierarchy,
and you would make her life difficult if you don't use \texttt{super}.

My SI hierarchy:
\begin{quote}{\ttfamily \raggedright \noindent
{\#}<why{\_}super.py>~\\
~\\
class~Base(object):~\\
~~~~def~{\_}{\_}init{\_}{\_}(self):~\\
~~~~~~~~print~"B.{\_}{\_}init{\_}{\_}"~\\
~\\
class~MyClass(Base):~\\
~~~~"I~do~not~cooperate~with~others"~\\
~~~~def~{\_}{\_}init{\_}{\_}(self):~\\
~~~~~~~~print~"MyClass.{\_}{\_}init{\_}{\_}"~\\
~~~~~~~~Base.{\_}{\_}init{\_}{\_}(self)~~{\#}instead~of~super(MyClass,~self).{\_}{\_}init{\_}{\_}()~\\
~\\
{\#}</why{\_}super.py>
}\end{quote}

Her MI hierarchy:
\begin{quote}{\ttfamily \raggedright \noindent
{\#}<why{\_}super.py>~\\
~\\
class~Mixin(Base):~\\
~~~~"I~am~cooperative~with~others"~\\
~~~~def~{\_}{\_}init{\_}{\_}(self):~\\
~~~~~~~~print~"Mixin.{\_}{\_}init{\_}{\_}"~\\
~~~~~~~~super(Mixin,~self).{\_}{\_}init{\_}{\_}()~\\
~\\
class~HerClass(MyClass,~Mixin):~\\
~~~~"I~am~supposed~to~be~cooperative~too"~\\
~~~~def~{\_}{\_}init{\_}{\_}(self):~\\
~~~~~~~~print~"HerClass.{\_}{\_}init{\_}{\_}"~\\
~~~~~~~~super(HerClass,~self).{\_}{\_}init{\_}{\_}()~\\
~\\
{\#}</why{\_}super.py>
}\end{quote}
\begin{verbatim}>>> from why_super import HerClass
>>> h = HerClass() # Mixin.__init__ is not called!
HerClass.__init__
MyClass.__init__
B.__init__\end{verbatim}
\begin{quote}
\begin{quote}{\ttfamily \raggedright \noindent
~~~~~~4~object~~~\\
~~~~~~~~~~|~\\
~~~~~~~3~Base~\\
~~~~~~/~~~~~~{\textbackslash}~\\
1~MyClass~~2~Mixin~\\
~~~~~~~{\textbackslash}~~~~~/~\\
~~~~~0~HerClass
}\end{quote}
\end{quote}
\begin{verbatim}>>> [ancestor.__name__ for ancestor in HerClass.mro()]
['HerClass', 'MyClass', 'Mixin', 'Base', 'object']\end{verbatim}

In order to be polite versus your future users, you should use \texttt{super} 
always. This adds a cognitive burden even for people not using MI :-(

Notice that there is no good comprehensive reference on \texttt{super} (yet)
Your best bet is still \href{http://www.python.org/2.2.3/descrintro.html\#cooperation}{http://www.python.org/2.2.3/descrintro.html{\#}cooperation}

The MRO instead is explained here: \href{http://www.python.org/2.3/mro.html}{http://www.python.org/2.3/mro.html}

Notice that I DO NOT recommand Multiple Inheritance.

More often than not you are better off using composition/delegation/wrapping, 
etc.

See Zope 2 -{\textgreater} Zope 3 experience.


%___________________________________________________________________________

\hypertarget{a-few-details-about-super-not-the-whole-truth}{}
\pdfbookmark[2]{A few details about super (not the whole truth)}{a-few-details-about-super-not-the-whole-truth}
\subsubsection*{A few details about \texttt{super} (not the whole truth)}
\begin{verbatim}>>> class B(object):
...     def __init__(self): print "B.__init__"
...
>>> class C(B):
...     def __init__(self): print "C.__init__"
...
>>> c = C()
C.__init__\end{verbatim}

\texttt{super(cls, instance)}, where \texttt{instance} is an instance of \texttt{cls} or of
a subclass of \texttt{cls}, retrieves the right method in the MRO:
\begin{verbatim}>>> super(C, c).__init__ #doctest: +ELLIPSIS
<bound method C.__init__ of <__main__.C object at 0x...>>\end{verbatim}
\begin{verbatim}>>> super(C, c).__init__.im_func is B.__init__.im_func
True\end{verbatim}
\begin{verbatim}>>> super(C, c).__init__()
B.__init__\end{verbatim}

\texttt{super(cls, subclass)} works for unbound methods:
\begin{verbatim}>>> super(C, C).__init__
<unbound method C.__init__>\end{verbatim}
\begin{verbatim}>>> super(C, C).__init__.im_func is B.__init__.im_func
True
>>> super(C, C).__init__(c)
B.__init__\end{verbatim}

\texttt{super(cls, subclass)} is also necessary for classmethods and staticmethods. 
Properties and custom descriptorsw works too:
\begin{quote}{\ttfamily \raggedright \noindent
{\#}<super{\_}ex.py>~\\
~\\
from~descriptor~import~AttributeDescriptor~\\
~\\
class~B(object):~\\
~~~@staticmethod~\\
~~~def~sm():~return~"staticmethod"~\\
~\\
~~~@classmethod~\\
~~~def~cm(cls):~return~cls.{\_}{\_}name{\_}{\_}~\\
~\\
~~~p~=~property()~\\
~~~a~=~AttributeDescriptor()~\\
~\\
class~C(B):~pass~\\
~\\
{\#}</super{\_}ex.py>
}\end{quote}
\begin{verbatim}>>> from super_ex import C\end{verbatim}

Staticmethod usage:
\begin{verbatim}>>> super(C, C).sm #doctest: +ELLIPSIS
<function sm at 0x...>
>>> super(C, C).sm()
'staticmethod'\end{verbatim}

Classmethod usage:
\begin{verbatim}>>> super(C, C()).cm
<bound method type.cm of <class 'super_ex.C'>>
>>> super(C, C).cm() # C is automatically passed
'C'\end{verbatim}

Property usage:
\begin{verbatim}>>> print super(C, C).p #doctest: +ELLIPSIS
<property object at 0x...>
>>> super(C, C).a #doctest: +ELLIPSIS
Getting <descriptor.AttributeDescriptor object at 0x...> from <class 'super_ex.C'>\end{verbatim}

\texttt{super} does not work with old-style classes, however you can use the
following trick:
\begin{quote}{\ttfamily \raggedright \noindent
{\#}<super{\_}old{\_}new.py>~\\
class~O:~\\
~~~~def~{\_}{\_}init{\_}{\_}(self):~\\
~~~~~~~~print~"O.{\_}{\_}init{\_}{\_}"~\\
~\\
class~N(O,~object):~\\
~~~~def~{\_}{\_}init{\_}{\_}(self):~\\
~~~~~~~~print~"N.{\_}{\_}init{\_}{\_}"~\\
~~~~~~~~super(N,~self).{\_}{\_}init{\_}{\_}()~\\
~\\
{\#}</super{\_}old{\_}new.py>
}\end{quote}
\begin{verbatim}>>> from super_old_new import N
>>> new = N()
N.__init__
O.__init__\end{verbatim}

There are dozens of tricky points concerning \texttt{super}, be warned!


%___________________________________________________________________________

\hypertarget{subclassing-built-in-types-new-vs-init}{}
\pdfbookmark[2]{Subclassing built-in types; {\_}{\_}new{\_}{\_} vs. {\_}{\_}init{\_}{\_}}{subclassing-built-in-types-new-vs-init}
\subsubsection*{Subclassing built-in types; {\_}{\_}new{\_}{\_} vs. {\_}{\_}init{\_}{\_}}
\begin{quote}{\ttfamily \raggedright \noindent
{\#}<point.py>~\\
~\\
class~NotWorkingPoint(tuple):~\\
~~~~def~{\_}{\_}init{\_}{\_}(self,~x,~y):~\\
~~~~~~~~super(NotWorkingPoint,~self).{\_}{\_}init{\_}{\_}((x,y))~\\
~~~~~~~~self.x,~self.y~=~x,~y~\\
~\\
{\#}</point.py>
}\end{quote}
\begin{verbatim}>>> from point import NotWorkingPoint
>>> p = NotWorkingPoint(2,3)
Traceback (most recent call last):
  ...
TypeError: tuple() takes at most 1 argument (2 given)\end{verbatim}
\begin{quote}{\ttfamily \raggedright \noindent
{\#}<point.py>~\\
~\\
class~Point(tuple):~\\
~~~~def~{\_}{\_}new{\_}{\_}(cls,~x,~y):~\\
~~~~~~~~return~super(Point,~cls).{\_}{\_}new{\_}{\_}(cls,~(x,y))~\\
~~~~def~{\_}{\_}init{\_}{\_}(self,~x,~y):~\\
~~~~~~~~super(Point,~self).{\_}{\_}init{\_}{\_}((x,~y))~\\
~~~~~~~~self.x,~self.y~=~x,~y~\\
~\\
{\#}</point.py>
}\end{quote}

Notice that{\_}{\_}new{\_}{\_} is a staticmethod, not a classmethod, so one needs
to pass the class explicitely.
\begin{verbatim}>>> from point import Point
>>> p = Point(2,3)
>>> print p, p.x, p.y
(2, 3) 2 3\end{verbatim}


%___________________________________________________________________________

\hypertarget{be-careful-when-using-new-with-mutable-types}{}
\pdfbookmark[2]{Be careful when using {\_}{\_}new{\_}{\_} with mutable types}{be-careful-when-using-new-with-mutable-types}
\subsubsection*{Be careful when using {\_}{\_}new{\_}{\_} with mutable types}
\begin{verbatim}>>> class ListWithDefault(list):
...     def __new__(cls):
...         return super(ListWithDefault, cls).__new__(cls, ["hello"])
...
>>> print ListWithDefault() # beware! NOT ["hello"]!
[]\end{verbatim}

Reason: lists are re-initialized to empty lists in list.{\_}{\_}init{\_}{\_}!

Instead
\begin{verbatim}>>> class ListWithDefault(list):
...     def __init__(self):
...         super(ListWithDefault, self).__init__(["hello"])
...
>>> print ListWithDefault() # works!
['hello']\end{verbatim}


%___________________________________________________________________________

\hypertarget{lecture-3-magic-i-e-decorators-and-metaclasses}{}
\pdfbookmark[0]{Lecture 3: Magic (i.e. decorators and metaclasses)}{lecture-3-magic-i-e-decorators-and-metaclasses}
\section*{Lecture 3: Magic (i.e. decorators and metaclasses)}


%___________________________________________________________________________

\hypertarget{part-i-decorators}{}
\pdfbookmark[1]{Part I: decorators}{part-i-decorators}
\subsection*{Part I: decorators}

Decorators are just sugar: their functionality was already in the language
\begin{verbatim}>>> def s(): pass
>>> s = staticmethod(s)\end{verbatim}
\begin{verbatim}>>> @staticmethod
... def s(): pass
...\end{verbatim}

However sugar \emph{does} matter.


%___________________________________________________________________________

\hypertarget{a-typical-decorator-traced}{}
\pdfbookmark[2]{A typical decorator: traced}{a-typical-decorator-traced}
\subsubsection*{A typical decorator: traced}
\begin{quote}{\ttfamily \raggedright \noindent
{\#}<traced.py>~\\
~\\
def~traced(func):~\\
~~~~def~tracedfunc(*args,~**kw):~\\
~~~~~~~~print~"calling~{\%}s.{\%}s"~{\%}~(func.{\_}{\_}module{\_}{\_},~func.{\_}{\_}name{\_}{\_})~\\
~~~~~~~~return~func(*args,~**kw)~\\
~~~~tracedfunc.{\_}{\_}name{\_}{\_}~=~func.{\_}{\_}name{\_}{\_}~\\
~~~~return~tracedfunc~\\
~\\
@traced~\\
def~f():~pass~\\
~\\
{\#}</traced.py>
}\end{quote}
\begin{verbatim}>>> from traced import f
>>> f()
calling traced.f\end{verbatim}


%___________________________________________________________________________

\hypertarget{a-decorator-factory-timed}{}
\pdfbookmark[2]{A decorator factory: Timed}{a-decorator-factory-timed}
\subsubsection*{A decorator factory: Timed}
\begin{quote}{\ttfamily \raggedright \noindent
{\#}<timed.py>~\\
~\\
import~sys,~time~\\
~\\
class~Timed(object):~\\
~~~~"{}"{}"Decorator~factory:~each~decorator~object~wraps~a~function~and~~\\
~~~~executes~it~many~times~(default~100~times).~\\
~~~~The~average~time~spent~in~one~iteration,~expressed~in~milliseconds,~~\\
~~~~is~stored~in~the~attributes~wrappedfunc.time~and~wrappedfunc.clocktime,~\\
~~~~and~displayed~into~a~log~file~which~defaults~to~stdout.~\\
~~~~"{}"{}"~\\
~~~~def~{\_}{\_}init{\_}{\_}(self,~repeat=100,~logfile=sys.stdout):~\\
~~~~~~~~self.repeat~=~repeat~\\
~~~~~~~~self.logfile~=~logfile~\\
~~~~def~{\_}{\_}call{\_}{\_}(self,~func):~\\
~~~~~~~~def~wrappedfunc(*args,~**kw):~\\
~~~~~~~~~~~~fullname~=~"{\%}s.{\%}s~..."~{\%}~(func.{\_}{\_}module{\_}{\_},~func.func{\_}name)~\\
~~~~~~~~~~~~print~>{}>~self.logfile,~'Executing~{\%}s'~{\%}~fullname.ljust(30),~\\
~~~~~~~~~~~~time1~=~time.time()~\\
~~~~~~~~~~~~clocktime1~=~time.clock()~\\
~~~~~~~~~~~~for~i~in~xrange(self.repeat):~\\
~~~~~~~~~~~~~~~~res~=~func(*args,**kw)~{\#}~executes~func~self.repeat~times~\\
~~~~~~~~~~~~time2~=~time.time()~\\
~~~~~~~~~~~~clocktime2~=~time.clock()~\\
~~~~~~~~~~~~wrappedfunc.time~=~1000*(time2-time1)/self.repeat~\\
~~~~~~~~~~~~wrappedfunc.clocktime~=~1000*(clocktime2~-~clocktime1)/self.repeat~\\
~~~~~~~~~~~~print~>{}>~self.logfile,~{\textbackslash}~\\
~~~~~~~~~~~~~~~~~~'Real~time:~{\%}s~ms;'~{\%}~self.rounding(wrappedfunc.time),~\\
~~~~~~~~~~~~print~>{}>~self.logfile,~{\textbackslash}~\\
~~~~~~~~~~~~~~~~~~'Clock~time:~{\%}s~ms'~{\%}~self.rounding(wrappedfunc.clocktime)~\\
~~~~~~~~~~~~return~res~\\
~~~~~~~~wrappedfunc.func{\_}name~=~func.func{\_}name~\\
~~~~~~~~wrappedfunc.{\_}{\_}module{\_}{\_}~=~func.{\_}{\_}module{\_}{\_}~\\
~~~~~~~~return~wrappedfunc~\\
~~~~@staticmethod~\\
~~~~def~rounding(float{\_}):~\\
~~~~~~~~"Three~digits~rounding~for~small~numbers,~1~digit~rounding~otherwise."~\\
~~~~~~~~if~float{\_}~<~10.:~\\
~~~~~~~~~~~~return~"{\%}5.3f"~{\%}~float{\_}~\\
~~~~~~~~else:~\\
~~~~~~~~~~~~return~"{\%}5.1f"~{\%}~float{\_}~\\
~~~~\\
{\#}</timed.py>
}\end{quote}
\begin{verbatim}>>> from timed import Timed
>>> from random import sample
>>> example_ls = sample(xrange(1000000), 1000)
>>> @Timed()
... def list_sort(ls):
...     ls.sort()
... 
>>> list_sort(example_ls) #doctest: +ELLIPSIS
Executing __main__.list_sort ... Real time: 0... ms; Clock time: 0... ms\end{verbatim}


%___________________________________________________________________________

\hypertarget{a-powerful-decorator-pattern}{}
\pdfbookmark[2]{A powerful decorator pattern}{a-powerful-decorator-pattern}
\subsubsection*{A powerful decorator pattern}
\begin{quote}{\ttfamily \raggedright \noindent
{\#}<traced{\_}function2.py>~\\
~\\
from~decorators~import~decorator~\\
~\\
def~trace(f,~*args,~**kw):~\\
~~~~print~"calling~{\%}s~with~args~{\%}s,~{\%}s"~{\%}~(f.func{\_}name,~args,~kw)~~\\
~~~~return~f(*args,~**kw)~\\
~\\
traced{\_}function~=~decorator(trace)~\\
~\\
@traced{\_}function~\\
def~f1(x):~\\
~~~~pass~\\
~\\
@traced{\_}function~\\
def~f2(x,~y):~\\
~~~~pass~\\
~\\
{\#}</traced{\_}function2.py>
}\end{quote}
\begin{verbatim}>>> from traced_function2 import traced_function, f1, f2
>>> f1(1)
calling f1 with args (1,), {}
>>> f2(1,2)
calling f2 with args (1, 2), {}\end{verbatim}

works with pydoc:
\begin{quote}{\ttfamily \raggedright \noindent
{\$}~pydoc2.4~traced{\_}function2.f2~~~\\
Help~on~function~f1~in~traced{\_}function2:~\\
~\\
traced{\_}function2.f1~=~f1(x)~\\
~\\
{\$}~pydoc2.4~traced{\_}function2.f2~~~\\
Help~on~function~f2~in~traced{\_}function2:~\\
~\\
traced{\_}function2.f2~=~f2(x,~y)
}\end{quote}

Here is the source code:
\begin{quote}{\ttfamily \raggedright \noindent
{\#}<decorators.py>~\\
~\\
def~getinfo(func):~\\
~~~~"{}"{}"Return~an~info~dictionary~containing:~\\
~~~~-~name~(the~name~of~the~function~:~str)~\\
~~~~-~argnames~(the~names~of~the~arguments~:~list)~\\
~~~~-~defarg~(the~values~of~the~default~arguments~:~list)~\\
~~~~-~fullsign~(the~full~signature~:~str)~\\
~~~~-~shortsign~(the~short~signature~:~str)~\\
~~~~-~arg0~...~argn~(shortcuts~for~the~names~of~the~arguments)~\\
~\\
~~~~>{}>{}>~def~f(self,~x=1,~y=2,~*args,~**kw):~pass~\\
~\\
~~~~>{}>{}>~info~=~getinfo(f)~\\
~\\
~~~~>{}>{}>~info{[}"name"]~\\
~~~~'f'~\\
~~~~>{}>{}>~info{[}"argnames"]~\\
~~~~{[}'self',~'x',~'y',~'args',~'kw']~\\
~~~~~\\
~~~~>{}>{}>~info{[}"defarg"]~\\
~~~~(1,~2)~\\
~\\
~~~~>{}>{}>~info{[}"shortsign"]~\\
~~~~'self,~x,~y,~*args,~**kw'~\\
~~~~~\\
~~~~>{}>{}>~info{[}"fullsign"]~\\
~~~~'self,~x=defarg{[}0],~y=defarg{[}1],~*args,~**kw'~\\
~\\
~~~~>{}>{}>~info{[}"arg0"],~info{[}"arg1"],~info{[}"arg2"],~info{[}"arg3"],~info{[}"arg4"]~\\
~~~~('self',~'x',~'y',~'args',~'kw')~\\
~~~~"{}"{}"~\\
~~~~assert~inspect.ismethod(func)~or~inspect.isfunction(func)~\\
~~~~regargs,~varargs,~varkwargs,~defaults~=~inspect.getargspec(func)~\\
~~~~argnames~=~list(regargs)~\\
~~~~if~varargs:~argnames.append(varargs)~\\
~~~~if~varkwargs:~argnames.append(varkwargs)~\\
~~~~counter~=~itertools.count()~\\
~~~~fullsign~=~inspect.formatargspec(~\\
~~~~~~~~regargs,~varargs,~varkwargs,~defaults,~\\
~~~~~~~~formatvalue=lambda~value:~"=defarg{[}{\%}i]"~{\%}~counter.next()){[}1:-1]~\\
~~~~shortsign~=~inspect.formatargspec(~\\
~~~~~~~~regargs,~varargs,~varkwargs,~defaults,~\\
~~~~~~~~formatvalue=lambda~value:~"{}"){[}1:-1]~\\
~~~~dic~=~dict(("arg{\%}s"~{\%}~n,~name)~for~n,~name~in~enumerate(argnames))~\\
~~~~dic.update(name=func.{\_}{\_}name{\_}{\_},~argnames=argnames,~shortsign=shortsign,~\\
~~~~~~~~fullsign~=~fullsign,~defarg~=~func.func{\_}defaults~or~())~\\
~~~~return~dic~\\
~\\
def~{\_}contains{\_}reserved{\_}names(dic):~{\#}~helper~\\
~~~~return~"{\_}call{\_}"~in~dic~or~"{\_}func{\_}"~in~dic~\\
~\\
def~{\_}decorate(func,~caller):~\\
~~~~"{}"{}"Takes~a~function~and~a~caller~and~returns~the~function~\\
~~~~decorated~with~that~caller.~The~decorated~function~is~obtained~\\
~~~~by~evaluating~a~lambda~function~with~the~correct~signature.~\\
~~~~"{}"{}"~\\
~~~~infodict~=~getinfo(func)~\\
~~~~assert~not~{\_}contains{\_}reserved{\_}names(infodict{[}"argnames"]),~{\textbackslash}~\\
~~~~~~~~~~~"You~cannot~use~{\_}call{\_}~or~{\_}func{\_}~as~argument~names!"~\\
~~~~execdict=dict({\_}func{\_}=func,~{\_}call{\_}=caller,~defarg=func.func{\_}defaults~or~())~\\
~~~~if~func.{\_}{\_}name{\_}{\_}~==~"<lambda>":~\\
~~~~~~~~lambda{\_}src~=~"lambda~{\%}(fullsign)s:~{\_}call{\_}({\_}func{\_},~{\%}(shortsign)s)"~{\textbackslash}~\\
~~~~~~~~~~~~~~~~~~~~~{\%}~infodict~\\
~~~~~~~~dec{\_}func~=~eval(lambda{\_}src,~execdict)~\\
~~~~else:~\\
~~~~~~~~func{\_}src~=~"{}"{}"def~{\%}(name)s({\%}(fullsign)s):~\\
~~~~~~~~return~{\_}call{\_}({\_}func{\_},~{\%}(shortsign)s)"{}"{}"~{\%}~infodict~\\
~~~~~~~~exec~func{\_}src~in~execdict~~\\
~~~~~~~~dec{\_}func~=~execdict{[}func.{\_}{\_}name{\_}{\_}]~\\
~~~~{\#}import~sys;~print~>{}>~sys.stderr,~func{\_}src~{\#}~for~debugging~~\\
~~~~dec{\_}func.{\_}{\_}doc{\_}{\_}~=~func.{\_}{\_}doc{\_}{\_}~\\
~~~~dec{\_}func.{\_}{\_}dict{\_}{\_}~=~func.{\_}{\_}dict{\_}{\_}~\\
~~~~return~dec{\_}func~\\
~\\
class~decorator(object):~\\
~~~~"{}"{}"General~purpose~decorator~factory:~takes~a~caller~function~as~\\
input~and~returns~a~decorator.~A~caller~function~is~any~function~like~this::~\\
~\\
~~~~def~caller(func,~*args,~**kw):~\\
~~~~~~~~{\#}~do~something~\\
~~~~~~~~return~func(*args,~**kw)~\\
~~~~~\\
Here~is~an~example~of~usage:~\\
~\\
~~~~>{}>{}>~@decorator~\\
~~~~...~def~chatty(f,~*args,~**kw):~\\
~~~~...~~~~~print~"Calling~{\%}r"~{\%}~f.{\_}{\_}name{\_}{\_}~\\
~~~~...~~~~~return~f(*args,~**kw)~\\
~~~~~\\
~~~~>{}>{}>~@chatty~\\
~~~~...~def~f():~pass~\\
~~~~...~\\
~~~~>{}>{}>~f()~\\
~~~~Calling~'f'~\\
~~~~"{}"{}"~\\
~~~~def~{\_}{\_}init{\_}{\_}(self,~caller):~\\
~~~~~~~~self.caller~=~caller~\\
~~~~def~{\_}{\_}call{\_}{\_}(self,~func):~\\
~~~~~~~~return~{\_}decorate(func,~self.caller)~\\
~\\
~\\
{\#}</decorators.py>
}\end{quote}

The possibilities of this pattern are endless.


%___________________________________________________________________________

\hypertarget{a-deferred-decorator}{}
\pdfbookmark[2]{A deferred decorator}{a-deferred-decorator}
\subsubsection*{A deferred decorator}

You want to execute a procedure only after a certain time delay (for instance
for use within an asyncronous Web framework):
\begin{quote}{\ttfamily \raggedright \noindent
{\#}<deferred.py>~\\
"Deferring~the~execution~of~a~procedure~(function~returning~None)"~\\
~\\
import~threading~\\
from~decorators~import~decorator~\\
~\\
def~deferred(nsec):~\\
~~~~def~call{\_}later(func,~*args,~**kw):~\\
~~~~~~~~return~threading.Timer(nsec,~func,~args,~kw).start()~\\
~~~~return~decorator(call{\_}later)~\\
~\\
@deferred(2)~\\
def~hello():~\\
~~~~print~"hello"~\\
~\\
if~{\_}{\_}name{\_}{\_}~==~"{\_}{\_}main{\_}{\_}":~\\
~~~~hello()~~~~~\\
~~~~print~"Calling~hello()~..."~\\
~~~~\\
~\\
{\#}</deferred.py>~\\
~\\
{\$}~python~deferred.py
}\end{quote}

Show an example of an experimental decorator based web framework
(doctester{\_}frontend).


%___________________________________________________________________________

\hypertarget{part-ii-metaclasses}{}
\pdfbookmark[1]{Part II: metaclasses}{part-ii-metaclasses}
\subsection*{Part II: metaclasses}

Metaclasses are there! Consider this example from a recent post on c.l.py:
\begin{quote}{\ttfamily \raggedright \noindent
{\#}<BaseClass.py>~\\
~\\
class~BaseClass(object):~\\
~~~"Do~something"~\\
~\\
{\#}</BaseClass.py>
}\end{quote}
\begin{verbatim}>>> import BaseClass # instead of 'from BaseClass import BaseClass'
>>> class C(BaseClass): pass
...
Traceback (most recent call last):
  ...
TypeError: Error when calling the metaclass bases
    module.__init__() takes at most 2 arguments (3 given)\end{verbatim}

The reason for the error is that class \texttt{C(BaseClass): pass} is
actually calling the \texttt{type} metaclass with three arguments:
\begin{quote}{\ttfamily \raggedright \noindent
C~=~type("C",~(BaseClass,),~{\{}{\}})
}\end{quote}

\texttt{type.{\_}{\_}new{\_}{\_}} tries to use \texttt{type(BaseClass)} as metaclass,
but since BaseClass here is a module,  and \texttt{ModuleType} is not
a metaclass, it cannot work. The error message reflects a conflict with 
the signature of ModuleType which requires two parameters and not three.

So even if you don't use them, you may want to know they exist.


%___________________________________________________________________________

\hypertarget{rejuvenating-old-style-classes}{}
\pdfbookmark[2]{Rejuvenating old-style classes}{rejuvenating-old-style-classes}
\subsubsection*{Rejuvenating old-style classes}
\begin{verbatim}>>> class Old: pass
>>> print type(Old)
<type 'classobj'>\end{verbatim}
\begin{verbatim}>>> __metaclass__ = type # to rejuvenate class
>>> class NotOld: pass
...
>>> print NotOld.__class__
<type 'type'>\end{verbatim}


%___________________________________________________________________________

\hypertarget{a-typical-metaclass-example-metatracer}{}
\pdfbookmark[2]{A typical metaclass example: MetaTracer}{a-typical-metaclass-example-metatracer}
\subsubsection*{A typical metaclass example: MetaTracer}
\begin{quote}{\ttfamily \raggedright \noindent
{\#}<metatracer.py>~\\
~\\
import~inspect~\\
from~decorators~import~decorator~\\
~\\
@decorator~\\
def~traced(meth,~*args,~**kw):~\\
~~~~cls~=~meth.{\_}{\_}cls{\_}{\_}~\\
~~~~modname~=~meth.{\_}{\_}module{\_}{\_}~or~cls.{\_}{\_}module{\_}{\_}~\\
~~~~print~"calling~{\%}s.{\%}s.{\%}s"~{\%}~(modname,~cls.{\_}{\_}name{\_}{\_},~meth.{\_}{\_}name{\_}{\_})~\\
~~~~return~meth(*args,~**kw)~\\
~\\
class~MetaTracer(type):~~~~~~~~~~~~~\\
~~~~def~{\_}{\_}init{\_}{\_}(cls,~name,~bases,~dic):~\\
~~~~~~~~super(MetaTracer,~cls).{\_}{\_}init{\_}{\_}(name,~bases,~dic)~\\
~~~~~~~~for~k,~v~in~dic.iteritems():~\\
~~~~~~~~~~~~if~inspect.isfunction(v):~\\
~~~~~~~~~~~~~~~~v.{\_}{\_}cls{\_}{\_}~=~cls~{\#}~so~we~know~in~which~class~v~was~defined~\\
~~~~~~~~~~~~~~~~setattr(cls,~k,~traced(v))~\\
~\\
{\#}</metatracer.py>
}\end{quote}

Usage: exploring classes in the standard library
\begin{quote}{\ttfamily \raggedright \noindent
{\#}<dictmixin.py>~\\
~\\
from~metatracer~import~MetaTracer~\\
from~UserDict~import~DictMixin~\\
~\\
class~TracedDM(DictMixin,~object):~\\
~~~~{\_}{\_}metaclass{\_}{\_}~=~MetaTracer~\\
~~~~def~{\_}{\_}getitem{\_}{\_}(self,~item):~\\
~~~~~~~~return~item~\\
~~~~def~keys(self):~~\\
~~~~~~~~return~{[}1,2,3]~\\
~\\
{\#}</dictmixin.py>
}\end{quote}
\begin{verbatim}>>> from dictmixin import TracedDM
>>> print TracedDM()
calling dictmixin.TracedDM.keys
calling dictmixin.TracedDM.__getitem__
calling dictmixin.TracedDM.__getitem__
calling dictmixin.TracedDM.__getitem__
{1: 1, 2: 2, 3: 3}\end{verbatim}


%___________________________________________________________________________

\hypertarget{real-life-example-check-overriding}{}
\pdfbookmark[2]{Real life example: check overriding}{real-life-example-check-overriding}
\subsubsection*{Real life example: check overriding}
\begin{quote}{\ttfamily \raggedright \noindent
{\#}<check{\_}overriding.py>~\\
~\\
class~Base(object):~\\
~~~~a~=~0~\\
~\\
class~CheckOverriding(type):~\\
~~~~"Prints~a~message~if~we~are~overriding~a~name."~\\
~~~~def~{\_}{\_}new{\_}{\_}(mcl,~name,~bases,~dic):~\\
~~~~~~~~for~name,~val~in~dic.iteritems():~\\
~~~~~~~~~~~~if~name.startswith("{\_}{\_}")~and~name.endswith("{\_}{\_}"):~~\\
~~~~~~~~~~~~~~~~continue~{\#}~ignore~special~names~\\
~~~~~~~~~~~~a{\_}base{\_}has{\_}name~=~True~in~(hasattr(base,~name)~for~base~in~bases)~\\
~~~~~~~~~~~~if~a{\_}base{\_}has{\_}name:~\\
~~~~~~~~~~~~~~~~print~"AlreadyDefinedNameWarning:~"~+~name~\\
~~~~~~~~return~super(CheckOverriding,~mcl).{\_}{\_}new{\_}{\_}(mcl,~name,~bases,~dic)~\\
~\\
class~MyClass(Base):~\\
~~~~{\_}{\_}metaclass{\_}{\_}~=~CheckOverriding~\\
~~~~a~=~1~\\
~\\
class~ChildClass(MyClass):~\\
~~~~a~=~2
}\end{quote}

{\#}{\textless}/check{\_}overriding.py{\textgreater}
\begin{verbatim}>>> import check_overriding
AlreadyDefinedNameWarning: a
AlreadyDefinedNameWarning: a\end{verbatim}


%___________________________________________________________________________

\hypertarget{logfile}{}
\pdfbookmark[2]{LogFile}{logfile}
\subsubsection*{LogFile}
\begin{quote}{\ttfamily \raggedright \noindent
{\#}<logfile.py>~\\
~\\
import~subprocess~\\
~\\
def~memoize(func):~\\
~~~~memoize{\_}dic~=~{\{}{\}}~\\
~~~~def~wrapped{\_}func(*args):~\\
~~~~~~~~if~args~in~memoize{\_}dic:~\\
~~~~~~~~~~~~return~memoize{\_}dic{[}args]~\\
~~~~~~~~else:~\\
~~~~~~~~~~~~result~=~func(*args)~\\
~~~~~~~~~~~~memoize{\_}dic{[}args]~=~result~\\
~~~~~~~~~~~~return~result~\\
~~~~wrapped{\_}func.{\_}{\_}name{\_}{\_}~=~func.{\_}{\_}name{\_}{\_}~\\
~~~~wrapped{\_}func.{\_}{\_}doc{\_}{\_}~=~func.{\_}{\_}doc{\_}{\_}~\\
~~~~wrapped{\_}func.{\_}{\_}dict{\_}{\_}~=~func.{\_}{\_}dict{\_}{\_}~\\
~~~~return~wrapped{\_}func~\\
~\\
class~Memoize(type):~{\#}~Singleton~is~a~special~case~of~Memoize~\\
~~~~@memoize~\\
~~~~def~{\_}{\_}call{\_}{\_}(cls,~*args):~\\
~~~~~~~~return~super(Memoize,~cls).{\_}{\_}call{\_}{\_}(*args)~\\
~\\
class~LogFile(file):~\\
~~~~"{}"{}"Open~a~file~for~append."{}"{}"~\\
~~~~{\_}{\_}metaclass{\_}{\_}~=~Memoize~\\
~~~~def~{\_}{\_}init{\_}{\_}(self,~name~=~"/tmp/err.log"):~\\
~~~~~~~~self.viewer{\_}cmd~=~'xterm~-e~less'.split()~\\
~~~~~~~~super(LogFile,~self).{\_}{\_}init{\_}{\_}(name,~"a")~\\
~\\
~~~~def~display(self,~*ls):~\\
~~~~~~~~"Use~'less'~to~display~the~log~file~in~a~separate~xterm."~\\
~~~~~~~~print~>{}>~self,~"{\textbackslash}n".join(map(str,~ls));~self.flush()~\\
~~~~~~~~subprocess.call(self.viewer{\_}cmd~+~{[}self.name])~\\
~\\
~~~~def~reset(self):~\\
~~~~~~~~"Erase~the~log~file."~\\
~~~~~~~~print~>{}>~file(self.name,~"w")~\\
~\\
if~{\_}{\_}name{\_}{\_}~==~"{\_}{\_}main{\_}{\_}":~{\#}~test~\\
~~~~print~>{}>~LogFile(),~"hello"~\\
~~~~print~>{}>~LogFile(),~"world"~\\
~~~~LogFile().display()~\\
~\\
{\#}</logfile.py>~\\
~\\
{\$}~python~logfile.py
}\end{quote}


%___________________________________________________________________________

\hypertarget{cooperative-hierarchies}{}
\pdfbookmark[2]{Cooperative hierarchies}{cooperative-hierarchies}
\subsubsection*{Cooperative hierarchies}
\begin{quote}{\ttfamily \raggedright \noindent
{\#}<cooperative{\_}init.py>~\\
~\\
"{}"{}"Given~a~hierarchy,~makes~{\_}{\_}init{\_}{\_}~cooperative.~\\
The~only~change~needed~is~to~add~a~line~\\
~\\
~~~{\_}{\_}metaclass{\_}{\_}~=~CooperativeInit~\\
~\\
to~the~base~class~of~your~hierarchy."{}"{}"~\\
~\\
from~decorators~import~decorator~~\\
~\\
class~CooperativeInit(type):~\\
~~~~def~{\_}{\_}init{\_}{\_}(cls,~name,~bases,~dic):~\\
~\\
~~~~~~~~@decorator~\\
~~~~~~~~def~make{\_}cooperative({\_}{\_}init{\_}{\_},~self,~*args,~**kw):~\\
~~~~~~~~~~~~super(cls,~self).{\_}{\_}init{\_}{\_}(*args,~**kw)~\\
~~~~~~~~~~~~{\_}{\_}init{\_}{\_}(self,~*args,~**kw)~\\
~\\
~~~~~~~~{\_}{\_}init{\_}{\_}~=~dic.get("{\_}{\_}init{\_}{\_}")~\\
~~~~~~~~if~{\_}{\_}init{\_}{\_}:~\\
~~~~~~~~~~~~cls.{\_}{\_}init{\_}{\_}~=~make{\_}cooperative({\_}{\_}init{\_}{\_})~\\
~\\
class~Base:~\\
~~~~{\_}{\_}metaclass{\_}{\_}~=~CooperativeInit~\\
~~~~def~{\_}{\_}init{\_}{\_}(self):~\\
~~~~~~~~print~"B.{\_}{\_}init{\_}{\_}"~\\
~\\
class~C1(Base):~\\
~~~~def~{\_}{\_}init{\_}{\_}(self):~\\
~~~~~~~~print~"C1.{\_}{\_}init{\_}{\_}"~\\
~\\
class~C2(Base):~\\
~~~def~{\_}{\_}init{\_}{\_}(self):~\\
~~~~~~~print~"C2.{\_}{\_}init{\_}{\_}"~\\
~\\
class~D(C1,~C2):~\\
~~~~def~{\_}{\_}init{\_}{\_}(self):~\\
~~~~~~~~print~"D.{\_}{\_}init{\_}{\_}"~\\
~\\
{\#}</cooperative{\_}init.py>
}\end{quote}
\begin{verbatim}>>> from cooperative_init import D
>>> d = D()
B.__init__
C2.__init__
C1.__init__
D.__init__\end{verbatim}


%___________________________________________________________________________

\hypertarget{metaclass-enhanced-modules}{}
\pdfbookmark[2]{Metaclass-enhanced modules}{metaclass-enhanced-modules}
\subsubsection*{Metaclass-enhanced modules}
\begin{quote}{\ttfamily \raggedright \noindent
{\#}<import{\_}with{\_}metaclass.py>~\\
"{}"{}"~\\
`{}`import{\_}with{\_}metaclass(metaclass,~modulepath)`{}`~generates~\\
a~new~module~from~and~old~module,~by~enhancing~all~of~its~classes.~\\
This~is~not~perfect,~but~it~should~give~you~a~start."{}"{}"~\\
~\\
import~os,~sys,~inspect,~types~\\
~\\
def~import{\_}with{\_}metaclass(metaclass,~modulepath):~\\
~~~~modname~=~os.path.basename(modulepath){[}:-3]~{\#}~simplistic~\\
~~~~mod~=~types.ModuleType(modname)~\\
~~~~locs~=~dict(~\\
~~~~~~~~{\_}{\_}module{\_}{\_}~=~modname,~\\
~~~~~~~~{\_}{\_}metaclass{\_}{\_}~=~metaclass,~\\
~~~~~~~~object~=~metaclass("object",~(),~{\{}{\}}))~\\
~~~~execfile(modulepath,~locs)~\\
~~~~for~k,~v~in~locs.iteritems():~\\
~~~~~~~~if~inspect.isclass(v):~{\#}~otherwise~it~would~be~"{\_}{\_}builtin{\_}{\_}"~\\
~~~~~~~~~~~~v.{\_}{\_}module{\_}{\_}~=~"{\_}{\_}dynamic{\_}{\_}"~\\
~~~~~~~~setattr(mod,~k,~v)~\\
~~~~return~mod
}\end{quote}

{\#}{\textless}/import{\_}with{\_}metaclass.py{\textgreater}
\begin{verbatim}>>> from import_with_metaclass import import_with_metaclass
>>> from metatracer import MetaTracer
>>> traced_optparse = import_with_metaclass(MetaTracer, 
...     "/usr/lib/python2.4/optparse.py")
>>> op = traced_optparse.OptionParser()
calling __dynamic__.OptionParser.__init__
calling __dynamic__.OptionContainer.__init__
calling __dynamic__.OptionParser._create_option_list
calling __dynamic__.OptionContainer._create_option_mappings
calling __dynamic__.OptionContainer.set_conflict_handler
calling __dynamic__.OptionContainer.set_description
calling __dynamic__.OptionParser.set_usage
calling __dynamic__.IndentedHelpFormatter.__init__
calling __dynamic__.HelpFormatter.__init__
calling __dynamic__.HelpFormatter.set_parser
calling __dynamic__.OptionParser._populate_option_list
calling __dynamic__.OptionParser._add_help_option
calling __dynamic__.OptionContainer.add_option
calling __dynamic__.Option.__init__
calling __dynamic__.Option._check_opt_strings
calling __dynamic__.Option._set_opt_strings
calling __dynamic__.Option._set_attrs
calling __dynamic__.OptionContainer._check_conflict
calling __dynamic__.OptionParser._init_parsing_state\end{verbatim}

traced{\_}optparse is a dynamically generated module not leaving in the
file system.


%___________________________________________________________________________

\hypertarget{magic-properties}{}
\pdfbookmark[2]{Magic properties}{magic-properties}
\subsubsection*{Magic properties}
\begin{quote}{\ttfamily \raggedright \noindent
{\#}<magicprop.py>~\\
~\\
class~MagicProperties(type):~\\
~~~~def~{\_}{\_}init{\_}{\_}(cls,~name,~bases,~dic):~\\
~~~~~~~~prop{\_}names~=~set(name{[}3:]~for~name~in~dic~\\
~~~~~~~~~~~~~~~~~~~~~~~~~if~name.startswith("get")~\\
~~~~~~~~~~~~~~~~~~~~~~~~~or~name.startswith("set"))~\\
~~~~~~~~for~name~in~prop{\_}names:~\\
~~~~~~~~~~~~getter~=~getattr(cls,~"get"~+~name,~None)~\\
~~~~~~~~~~~~setter~=~getattr(cls,~"set"~+~name,~None)~\\
~~~~~~~~~~~~setattr(cls,~name,~property(getter,~setter))~\\
~\\
class~Base(object):~\\
~~~~{\_}{\_}metaclass{\_}{\_}~=~MagicProperties~\\
~~~~def~getx(self):~\\
~~~~~~~~return~self.{\_}x~\\
~~~~def~setx(self,~value):~\\
~~~~~~~~self.{\_}x~=~value~\\
~\\
class~Child(Base):~\\
~~~~def~getx(self):~\\
~~~~~~~~print~"getting~x"~\\
~~~~~~~~return~super(Child,~self).getx()~~\\
~~~~def~setx(self,~value):~\\
~~~~~~~~print~"setting~x"~\\
~~~~~~~~super(Child,~self).setx(value)~~\\
~\\
{\#}</magicprop.py>
}\end{quote}
\begin{verbatim}>>> from magicprop import Child
>>> c = Child()
>>> c.x = 1
setting x
>>> print c.x
getting x
1\end{verbatim}


%___________________________________________________________________________

\hypertarget{hack-evil-properties}{}
\pdfbookmark[2]{Hack: evil properties}{hack-evil-properties}
\subsubsection*{Hack: evil properties}
\begin{quote}{\ttfamily \raggedright \noindent
{\#}<evilprop.py>~\\
~\\
def~convert2property(name,~bases,~d):~\\
~~~~return~property(d.get('get'),~d.get('set'),~\\
~~~~~~~~~~~~~~~~~~~~d.get('del'),d.get('{\_}{\_}doc{\_}{\_}'))~\\
~\\
class~C(object):~\\
~~~~class~x:~\\
~~~~~~~~"{}"{}"An~evil~test~property"{}"{}"~\\
~~~~~~~~{\_}{\_}metaclass{\_}{\_}~=~convert2property~\\
~~~~~~~~def~get(self):~\\
~~~~~~~~~~~~print~'Getting~{\%}s'~{\%}~self.{\_}x~\\
~~~~~~~~~~~~return~self.{\_}x~\\
~~~~~~~~def~set(self,~value):~\\
~~~~~~~~~~~~self.{\_}x~=~value~\\
~~~~~~~~~~~~print~'Setting~to',~value~\\
~\\
{\#}</evilprop.py>
}\end{quote}
\begin{verbatim}>>> from evilprop import C
>>> c = C()
>>> c.x = 5
Setting to 5
>>> c.x
Getting 5
5
>>> print C.x.__doc__
An evil test property\end{verbatim}


%___________________________________________________________________________

\hypertarget{why-i-suggest-not-to-use-metaclasses-in-production-code}{}
\pdfbookmark[2]{Why I suggest not to use metaclasses in production code}{why-i-suggest-not-to-use-metaclasses-in-production-code}
\subsubsection*{Why I suggest \emph{not} to use metaclasses in production code}
\begin{quote}
\begin{itemize}
\item {} 
there are very few good use case for metaclasses in production code
(i.e. 99{\%} of time you don't need them)

\item {} 
they put a cognitive burden on the developer;

\item {} 
a design without metaclasses is less magic and likely more robust;

\item {} 
a design with metaclasses makes it difficult to use other metaclasses 
for debugging.

\end{itemize}
\end{quote}

As far as I know, string.Template is the only metaclass-enhanced class
in the standard library; the metaclass is used to give the possibility to
change the defaults:
\begin{quote}{\ttfamily \raggedright \noindent
delimiter~=~'{\$}'~\\
idpattern~=~r'{[}{\_}a-z]{[}{\_}a-z0-9]*'
}\end{quote}

in subclasses of Template.
\begin{verbatim}>>> from string import Template
>>> from metatracer import MetaTracer
>>> class TracedTemplate(Template):
...     __metaclass__ = MetaTracer
...
Traceback (most recent call last):
  ...
TypeError: Error when calling the metaclass bases
    metaclass conflict: the metaclass of a derived class must be a (non-strict) subclass of the metaclasses of all its bases\end{verbatim}

Solution: use a consistent metaclass
\begin{verbatim}>>> class GoodMeta(MetaTracer, type(Template)): pass
...
>>> class TracedTemplate(Template):
...     __metaclass__ = GoodMeta\end{verbatim}


%___________________________________________________________________________

\hypertarget{is-there-an-automatic-way-of-solving-the-conflict}{}
\pdfbookmark[2]{Is there an automatic way of solving the conflict?}{is-there-an-automatic-way-of-solving-the-conflict}
\subsubsection*{Is there an automatic way of solving the conflict?}

Yes, but you really need to be a metaclass wizard.

\href{http://aspn.activestate.com/ASPN/Cookbook/Python/Recipe/204197}{http://aspn.activestate.com/ASPN/Cookbook/Python/Recipe/204197}
\begin{verbatim}>>> from noconflict import classmaker
>>> class TracedTemplate(Template):
...     __metaclass__ = classmaker((MetaTracer,))
>>> print type(TracedTemplate)
<class 'noconflict._MetaTracer_TemplateMetaclass'>\end{verbatim}
\begin{quote}{\ttfamily \raggedright \noindent
{\#}<noconflict.py>~\\
~\\
import~inspect,~types,~{\_}{\_}builtin{\_}{\_}~\\
from~skip{\_}redundant~import~skip{\_}redundant~\\
~\\
memoized{\_}metaclasses{\_}map~=~{\{}{\}}~\\
~\\
{\#}~utility~function~\\
def~remove{\_}redundant(metaclasses):~\\
~~~skipset~=~set({[}types.ClassType])~\\
~~~for~meta~in~metaclasses:~{\#}~determines~the~metaclasses~to~be~skipped~\\
~~~~~~~skipset.update(inspect.getmro(meta){[}1:])~\\
~~~return~tuple(skip{\_}redundant(metaclasses,~skipset))~\\
~\\
{\#}{\#}{\#}{\#}{\#}{\#}{\#}{\#}{\#}{\#}{\#}{\#}{\#}{\#}{\#}{\#}{\#}{\#}{\#}{\#}{\#}{\#}{\#}{\#}{\#}{\#}{\#}{\#}{\#}{\#}{\#}{\#}{\#}{\#}{\#}{\#}{\#}{\#}{\#}{\#}{\#}{\#}{\#}{\#}{\#}{\#}{\#}{\#}{\#}{\#}{\#}{\#}{\#}{\#}{\#}{\#}{\#}{\#}{\#}{\#}{\#}{\#}{\#}{\#}{\#}{\#}~\\
{\#}{\#}~now~the~core~of~the~module:~two~mutually~recursive~functions~{\#}{\#}~\\
{\#}{\#}{\#}{\#}{\#}{\#}{\#}{\#}{\#}{\#}{\#}{\#}{\#}{\#}{\#}{\#}{\#}{\#}{\#}{\#}{\#}{\#}{\#}{\#}{\#}{\#}{\#}{\#}{\#}{\#}{\#}{\#}{\#}{\#}{\#}{\#}{\#}{\#}{\#}{\#}{\#}{\#}{\#}{\#}{\#}{\#}{\#}{\#}{\#}{\#}{\#}{\#}{\#}{\#}{\#}{\#}{\#}{\#}{\#}{\#}{\#}{\#}{\#}{\#}{\#}{\#}~\\
~\\
def~get{\_}noconflict{\_}metaclass(bases,~left{\_}metas,~right{\_}metas):~\\
~~~~"{}"{}"Not~intended~to~be~used~outside~of~this~module,~unless~you~know~\\
~~~~what~you~are~doing."{}"{}"~\\
~~~~{\#}~make~tuple~of~needed~metaclasses~in~specified~priority~order~\\
~~~~metas~=~left{\_}metas~+~tuple(map(type,~bases))~+~right{\_}metas~\\
~~~~needed{\_}metas~=~remove{\_}redundant(metas)~\\
~\\
~~~~{\#}~return~existing~confict-solving~meta,~if~any~\\
~~~~if~needed{\_}metas~in~memoized{\_}metaclasses{\_}map:~\\
~~~~~~return~memoized{\_}metaclasses{\_}map{[}needed{\_}metas]~\\
~~~~{\#}~nope:~compute,~memoize~and~return~needed~conflict-solving~meta~\\
~~~~elif~not~needed{\_}metas:~~~~~~~~~{\#}~wee,~a~trivial~case,~happy~us~\\
~~~~~~~~meta~=~type~\\
~~~~elif~len(needed{\_}metas)~==~1:~{\#}~another~trivial~case~\\
~~~~~~~meta~=~needed{\_}metas{[}0]~\\
~~~~{\#}~check~for~recursion,~can~happen~i.e.~for~Zope~ExtensionClasses~\\
~~~~elif~needed{\_}metas~==~bases:~~\\
~~~~~~~~raise~TypeError("Incompatible~root~metatypes",~needed{\_}metas)~\\
~~~~else:~{\#}~gotta~work~...~\\
~~~~~~~~metaname~=~'{\_}'~+~'{}'.join({[}m.{\_}{\_}name{\_}{\_}~for~m~in~needed{\_}metas])~\\
~~~~~~~~meta~=~classmaker()(metaname,~needed{\_}metas,~{\{}{\}})~\\
~~~~memoized{\_}metaclasses{\_}map{[}needed{\_}metas]~=~meta~\\
~~~~return~meta~\\
~\\
def~classmaker(left{\_}metas=(),~right{\_}metas=()):~\\
~~~def~make{\_}class(name,~bases,~adict):~\\
~~~~~~~metaclass~=~get{\_}noconflict{\_}metaclass(bases,~left{\_}metas,~right{\_}metas)~\\
~~~~~~~return~metaclass(name,~bases,~adict)~\\
~~~return~make{\_}class~\\
~\\
{\#}{\#}{\#}{\#}{\#}{\#}{\#}{\#}{\#}{\#}{\#}{\#}{\#}{\#}{\#}{\#}{\#}{\#}{\#}{\#}{\#}{\#}{\#}{\#}{\#}{\#}{\#}{\#}{\#}{\#}{\#}{\#}{\#}{\#}{\#}{\#}{\#}{\#}{\#}{\#}{\#}{\#}{\#}{\#}{\#}{\#}{\#}{\#}{\#}{\#}{\#}{\#}{\#}{\#}{\#}{\#}{\#}{\#}{\#}{\#}{\#}{\#}{\#}{\#}{\#}~\\
{\#}{\#}~and~now~a~conflict-safe~replacement~for~'type'~~~~~~~~~~~~~~{\#}{\#}~~\\
{\#}{\#}{\#}{\#}{\#}{\#}{\#}{\#}{\#}{\#}{\#}{\#}{\#}{\#}{\#}{\#}{\#}{\#}{\#}{\#}{\#}{\#}{\#}{\#}{\#}{\#}{\#}{\#}{\#}{\#}{\#}{\#}{\#}{\#}{\#}{\#}{\#}{\#}{\#}{\#}{\#}{\#}{\#}{\#}{\#}{\#}{\#}{\#}{\#}{\#}{\#}{\#}{\#}{\#}{\#}{\#}{\#}{\#}{\#}{\#}{\#}{\#}{\#}{\#}{\#}~\\
~\\
{\_}{\_}type{\_}{\_}={\_}{\_}builtin{\_}{\_}.type~{\#}~the~aboriginal~'type'~\\
{\#}~left~available~in~case~you~decide~to~rebind~{\_}{\_}builtin{\_}{\_}.type~\\
~\\
class~safetype({\_}{\_}type{\_}{\_}):~\\
~~~~{\#}~this~is~REALLY~DEEP~MAGIC~\\
~~~~"{}"{}"Overrides~the~`{}`{\_}{\_}new{\_}{\_}`{}`~method~of~the~`{}`type`{}`~metaclass,~making~the~\\
~~~~generation~of~classes~conflict-proof."{}"{}"~\\
~~~~def~{\_}{\_}new{\_}{\_}(mcl,~*args):~\\
~~~~~~~~nargs~=~len(args)~\\
~~~~~~~~if~nargs~==~1:~{\#}~works~as~{\_}{\_}builtin{\_}{\_}.type~\\
~~~~~~~~~~~~return~{\_}{\_}type{\_}{\_}(args{[}0])~~\\
~~~~~~~~elif~nargs~==~3:~{\#}~creates~the~class~using~the~appropriate~metaclass~\\
~~~~~~~~~~~~n,~b,~d~=~args~{\#}~name,~bases~and~dictionary~\\
~~~~~~~~~~~~meta~=~get{\_}noconflict{\_}metaclass(b,~(mcl,),~())~~\\
~~~~~~~~~~~~if~meta~is~mcl:~{\#}~meta~is~trivial,~dispatch~to~the~default~{\_}{\_}new{\_}{\_}~\\
~~~~~~~~~~~~~~~~return~super(safetype,~mcl).{\_}{\_}new{\_}{\_}(mcl,~n,~b,~d)~\\
~~~~~~~~~~~~else:~{\#}~non-trivial~metaclass,~dispatch~to~the~right~{\_}{\_}new{\_}{\_}~\\
~~~~~~~~~~~~~~~~{\#}~(it~will~take~a~second~round)~{\#}~print~mcl,~meta~\\
~~~~~~~~~~~~~~~~return~super(mcl,~meta).{\_}{\_}new{\_}{\_}(meta,~n,~b,~d)~\\
~~~~~~~~else:~\\
~~~~~~~~~~~~raise~TypeError('{\%}s()~takes~1~or~3~arguments'~{\%}~mcl.{\_}{\_}name{\_}{\_})~\\
~\\
{\#}</noconflict.py>
}\end{quote}

\end{document}

