\documentclass[10pt,a4paper,english]{article}
\usepackage{babel}
\usepackage{ae}
\usepackage{aeguill}
\usepackage{shortvrb}
\usepackage[latin1]{inputenc}
\usepackage{tabularx}
\usepackage{longtable}
\setlength{\extrarowheight}{2pt}
\usepackage{amsmath}
\usepackage{graphicx}
\usepackage{color}
\usepackage{multirow}
\usepackage{ifthen}
\usepackage[colorlinks=true,linkcolor=blue,urlcolor=blue]{hyperref}
\usepackage[DIV12]{typearea}
%% generator Docutils: http://docutils.sourceforge.net/
\newlength{\admonitionwidth}
\setlength{\admonitionwidth}{0.9\textwidth}
\newlength{\docinfowidth}
\setlength{\docinfowidth}{0.9\textwidth}
\newlength{\locallinewidth}
\newcommand{\optionlistlabel}[1]{\bf #1 \hfill}
\newenvironment{optionlist}[1]
{\begin{list}{}
  {\setlength{\labelwidth}{#1}
   \setlength{\rightmargin}{1cm}
   \setlength{\leftmargin}{\rightmargin}
   \addtolength{\leftmargin}{\labelwidth}
   \addtolength{\leftmargin}{\labelsep}
   \renewcommand{\makelabel}{\optionlistlabel}}
}{\end{list}}
\newlength{\lineblockindentation}
\setlength{\lineblockindentation}{2.5em}
\newenvironment{lineblock}[1]
{\begin{list}{}
  {\setlength{\partopsep}{\parskip}
   \addtolength{\partopsep}{\baselineskip}
   \topsep0pt\itemsep0.15\baselineskip\parsep0pt
   \leftmargin#1}
 \raggedright}
{\end{list}}
% begin: floats for footnotes tweaking.
\setlength{\floatsep}{0.5em}
\setlength{\textfloatsep}{\fill}
\addtolength{\textfloatsep}{3em}
\renewcommand{\textfraction}{0.5}
\renewcommand{\topfraction}{0.5}
\renewcommand{\bottomfraction}{0.5}
\setcounter{totalnumber}{50}
\setcounter{topnumber}{50}
\setcounter{bottomnumber}{50}
% end floats for footnotes
% some commands, that could be overwritten in the style file.
\newcommand{\rubric}[1]{\subsection*{~\hfill {\it #1} \hfill ~}}
\newcommand{\titlereference}[1]{\textsl{#1}}
% end of "some commands"
% donot indent first line.
\setlength{\parindent}{0pt}
\setlength{\parskip}{5pt plus 2pt minus 1pt}

% sloppy
% ------
% Less strict (opposite to default fussy) space size between words. Therefore
% less hyphenation.
\sloppy

% fonts
% -----
% times for pdf generation, gives smaller pdf files.
%
% But in standard postscript fonts: courier and times/helvetica do not fit.
% Maybe use pslatex.
\usepackage{times} 

% pagestyle
\pagestyle{headings}

\title{Corso di Python avanzato per la Magneti Marelli}
\author{}
\date{}
\hypersetup{
pdftitle={Corso di Python avanzato per la Magneti Marelli}
}
\raggedbottom
\begin{document}
\maketitle


\setlength{\locallinewidth}{\linewidth}

Daniele Matteucci mi ha chiesto di stilare una breve nota per una proposta
di un corso di Python avanzato da tenere presso la Magneti Marelli nel 
Settembre 2005.

Il corso si svolgerebbe in 3 giorni di full immersion. Il mio compenso
ordinario per questo tipologia di corsi � di 400 Euro al giorno pi� le 
spese (diciamo 150 Euro al giorno per vitto e alloggio). La cifra precisa 
andr� poi definita in base al programma da svolgere. Un programma indicativo 
potrebbe essere il seguente.
\begin{itemize}
\item {} 
Primo giorno: testing automatico di applicazioni in Python.

\end{itemize}

Discuter� i frameworks pi� usati (doctest, py.test, unittest) con
esercitazioni pratiche.
\begin{itemize}
\item {} 
Secondo giorno: refactoring a buone pratiche di programmazione.

\end{itemize}

Esempi pratici di come convertire cattivo codice in buon codice e
discussione dei pi� tipici errori di programmazione.
\begin{itemize}
\item {} 
Terzo giorno: argomenti specifici di programmazione avanzata.

\end{itemize}

Questa parte del programma � da concordare: argomenti trattabili
potrebbero essere la programmazione multithreading, programmazione
ad oggetti avanzata, iteratori e generatori, etc.

Ho gi� tenuto corsi di Python di questo tipo (quest'anno a Bolzano
e ad Oxford); il programma del corso di Oxford si pu� trovare a
qui:

\href{https://www.accu.org/conference/events_2005_04_19.html\#55}{https://www.accu.org/conference/events{\_}2005{\_}04{\_}19.html{\#}55}

Preventivo indicativo per tre giorni:
\begin{quote}

\begin{longtable}[c]{|p{0.11\locallinewidth}|p{0.10\locallinewidth}|p{0.10\locallinewidth}|}
\hline

Compenso
 & 
400 x 3
 & 
1200E
 \\
\hline

IVA
 & 
20{\%}
 & 
240E
 \\
\hline

Spese
 & 
150 x 3
 & 
450E
 \\
\hline

Viaggio
 &  & 
60E
 \\
\hline
\multicolumn{2}{|l|}{
Totale
} & 
1950E
 \\
\hline
\end{longtable}
\end{quote}

Roma, 23 giugno 2005
\begin{quote}

\emph{Michele Simionato}, \href{mailto:michele.simionato@gmail.com}{michele.simionato@gmail.com}
\end{quote}

\end{document}

