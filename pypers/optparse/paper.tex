\documentclass[10pt,english]{article}
\usepackage{babel}
\usepackage{shortvrb}
\usepackage[latin1]{inputenc}
\usepackage{tabularx}
\usepackage{longtable}
\setlength{\extrarowheight}{2pt}
\usepackage{amsmath}
\usepackage{graphicx}
\usepackage{color}
\usepackage{multirow}
\usepackage[colorlinks=true,linkcolor=blue,urlcolor=blue]{hyperref}
\usepackage[a4paper]{geometry}
%% generator Docutils: http://docutils.sourceforge.net/
\newlength{\admonitionwidth}
\setlength{\admonitionwidth}{0.9\textwidth}
\newlength{\docinfowidth}
\setlength{\docinfowidth}{0.9\textwidth}
\newlength{\locallinewidth}
\newcommand{\optionlistlabel}[1]{\bf #1 \hfill}
\newenvironment{optionlist}[1]
{\begin{list}{}
  {\setlength{\labelwidth}{#1}
   \setlength{\rightmargin}{1cm}
   \setlength{\leftmargin}{\rightmargin}
   \addtolength{\leftmargin}{\labelwidth}
   \addtolength{\leftmargin}{\labelsep}
   \renewcommand{\makelabel}{\optionlistlabel}}
}{\end{list}}
% begin: floats for footnotes tweaking.
\setlength{\floatsep}{0.5em}
\setlength{\textfloatsep}{\fill}
\addtolength{\textfloatsep}{3em}
\renewcommand{\textfraction}{0.5}
\renewcommand{\topfraction}{0.5}
\renewcommand{\bottomfraction}{0.5}
\setcounter{totalnumber}{50}
\setcounter{topnumber}{50}
\setcounter{bottomnumber}{50}
% end floats for footnotes
% some commands, that could be overwritten in the style file.
\newcommand{\rubric}[1]{\subsection*{~\hfill {\it #1} \hfill ~}}
% end of "some commands"
% donot indent first line.
\setlength{\parindent}{0pt}
\setlength{\parskip}{5pt plus 2pt minus 1pt}

% sloppy
% ------
% Less strict (opposite to default fussy) space size between words. Therefore
% less hyphenation.
\sloppy

% fonts
% -----
% times for pdf generation, gives smaller pdf files.
%
% But in standard postscript fonts: courier and times/helvetica do not fit.
% Maybe use pslatex.
\usepackage{times} 

% pagestyle
\pagestyle{headings}

\title{The optparse module: writing command-line tools the easy way}
\author{}
\date{}
\hypersetup{
pdftitle={The optparse module: writing command-line tools the easy way}
}
\raggedbottom
\begin{document}
\maketitle


\setlength{\locallinewidth}{\linewidth}
\begin{quote}
\begin{quote}
\begin{description}
\item [Status:]
Draft


\item [Author:]
Michele Simionato


\item [E-mail:]
\href{mailto:michele.simionato@partecs.com}{michele.simionato@partecs.com}


\item [Date:]
April 2004


\end{description}
\end{quote}
\end{quote}

\emph{The optparse module is a powerful, flexible, extensible, easy-to-use 
command-line parsing library for Python. Using optparse, you can add 
intelligent, sophisticated handling of command-line options to your 
scripts with very little overhead.} -- Greg Ward, optparse author


%___________________________________________________________________________

\hypertarget{introduction}{}
\pdfbookmark[0]{Introduction}{introduction}
\section*{Introduction}

Once upon a time, when graphic interfaces were still to be dreamed
about, command-line tools were the body and the soul of all programming
tools. Many years have passed since then, but some things have not 
changed: command-line tools are still fast, efficient, portable, easy 
to use and - more importantly - reliable. You can count on them.
You can expect command-line scripts to work in any situation, 
during the installation phase, in a situation of disaster recovery, when 
your window manager breaks down and even in systems with severe 
memory/hardware constraints. When you really need them, command-line 
tools are always there.

Hence, it is important for a programming language - especially
one that wants to be called a ``scripting'' language - to provide 
facilities to help the programmer in the task of writing command-line 
tools. For a long time Python support for this kind of tasks has 
been provided by the (old fashioned) \texttt{getopt} module. I have never 
been particularly fond of \texttt{getopt}, since it required the programmer 
to do a sensible amount of boring job even for the parsing of simple 
argument lines. However, with the coming of Python 2.3 the situation 
has changed: thanks to the great job of Greg Ward (the author of 
\texttt{optparse} a.k.a. \texttt{Optik}) now the Python programmer 
has at her disposal (in the standard library and not as an 
add-on module) a fully fledged Object Oriented API for 
command-line arguments parsing, which makes writing Unix-style 
command-line tools easy, efficient and fast.

The only disadvantage of \texttt{optparse} is that it is  
sophisticated tool, which requires some time to be fully mastered. 
The purpose of this paper is to help the reader to rapidly get the 
10{\%} of the features of \texttt{optparse} that she will use in the 90{\%} of 
the cases. Taking as an example a real life application - a search and 
replace tool -  I will guide the reader through (some of) the wonders 
of \texttt{optparse}, showing how easy is to use it. Also, I will 
show some trick that will make your life with \texttt{optparse} much happier. 
This paper is intended for both Unix and
Windows programmers - actually I will argue that Windows programmers 
need \texttt{optparse} even more than Unix programmers - and does not
require any particular expertise to be fully appreciated.


%___________________________________________________________________________

\hypertarget{a-simple-example}{}
\pdfbookmark[0]{A simple example}{a-simple-example}
\section*{A simple example}

I will take as pedagogical example a little tool I wrote some time ago,
a multiple files search and replace tool. I needed it because I am 
not always working under Unix, and not always I have Emacs or 
another powerful editor installed, so it made sense to have this 
little Python script in my toolbox. It is only few lines long,
it can always be modified and extended with a minimal effort,
works on every platform (including my PDA) and has the advantage 
of being completely command-line driven: 
it does not require to have any graphics library installed
and I can use it when I work on a remote machine via ssh.

The tool takes a bunch of files, look for a given regular expression 
and replace it everywhere in-place; moreover, it saves a backup copy of
the original un-modified files and give the option to recover 
them when I want to.  Of course, all of this can be done more efficiently 
in the Unix world with \texttt{sed} or other tools, but those tools are written 
in C and they are not as easily customizable as a Python script, that 
you may change in real time to suit your needs. So, it makes sense 
to write this kind of utilities in Python, and actually many people 
(including myself) are actively replacing some Unix commands and bash 
scripts with Python scripts. In real life, I have hacked quite a lot 
the minimal tool that I describe here, and I continue to tune it as the 
need raises.

As a final note, let me notice that I find \texttt{optparse} 
to be much more useful in the Windows world than in the Unix/Linux/Mac OS X
world. The reason is that the pletora 
of pretty good command-line tools which are available under Unix are 
missing in a Windows environment, or do not have a satisfactory 
equivalent. Therefore, 
it makes sense to write a personal collection of command-line scripts 
for your more common task, if you need to work on many platforms and
portability is an important requirement. 
Using Python and \texttt{optparse}, you may write your own scripts 
once and having them to run on every platform running Python, 
which means in practice any traditional platform and increasingly
more of the non-traditional ones - Python is spreading into the 
embedded market too, including PDA's, cellular phones, and more.


%___________________________________________________________________________

\hypertarget{the-unix-philosophy-for-command-line-arguments}{}
\pdfbookmark[0]{The Unix philosophy for command-line arguments}{the-unix-philosophy-for-command-line-arguments}
\section*{The Unix philosophy for command-line arguments}

In order to understand how \texttt{optparse} works, it is essential
to understand the Unix philosophy about command-lines arguments. 
As Greg Ward puts it:

\emph{The purpose of optparse is to make it very easy to provide the 
most standard, obvious, straightforward, and user-friendly user 
interface for Unix command-line programs. The optparse philosophy 
is heavily influenced by the Unix and GNU toolkits ...}

So, I think my Windows readers will be best served if I put here
a brief summary of the Unix terminology. Old time Unix programmers may safely 
skip this section. Let me just notice that \texttt{optparse} could easily be 
extended to implement other kinds of conventions for optional argument 
parsing, but very likely you \emph{don't want} to do that.

Here is optparse/Unix/GNU terminology:
the arguments given to a command-line script - \emph{i.e.}  the arguments
that Python stores in the list \texttt{sys.argv[1:]} - are classified in
three groups: options, option arguments and positional arguments. 
Options can be distinguished since they are prefixed by a dash
or a double dash; options can have option arguments or not 
(there is at most an option argument right after each option); 
options without arguments are called flags. Positional arguments 
are what it is left in the command-line after you remove options 
and option arguments.

In the example of the search/replace tool I was
talking about, I will need two options with an argument - since I want 
to pass to the script a regular expression and a replacement string - 
and I will need a flag specifying if a backup of the original files has 
to be performed or not. Finally, I will need a number of positional
arguments to store the names of the files on which the search and
replace will act.

Consider - for the sake of the example - the following situations:
you have a bunch of text files in the current directory containing dates 
in the European format DD-MM-YYYY, and that I want to convert them in
the American format MM-DD-YYYY. If I am sure that all my dates
are in the correct format, I can match them with a simple regular
expression such as \texttt{({\textbackslash}d{\textbackslash}d)-({\textbackslash}d{\textbackslash}d)-({\textbackslash}d{\textbackslash}d{\textbackslash}d{\textbackslash}d)} . Notice that
this regular expression can be complicated at will, or can be 
written differently, but it is enough for the purpose of this 
paper and has the advantage of being simple enough to be
understood even by readers with very little familiarity with 
regexes (it essentially looks for strings composed of the
groups of digits separated by dashes, with the first and
second group composed by two digits and the last group
composed by four digits).

In this particular example it is not so important to make a backup
copy of the original files, since to reverts to the original
format it is enough to run again the script: then the position
of the month will be switched again back to the European 
convention. So the syntax to use would be something like
\begin{quote}
\begin{ttfamily}\begin{flushleft}
\mbox{{\$}>~replace.py~--nobackup~--regx="({\textbackslash}d{\textbackslash}d)-({\textbackslash}d{\textbackslash}d)-({\textbackslash}d{\textbackslash}d{\textbackslash}d{\textbackslash}d)"~{\textbackslash}}\\
\mbox{~~~~~~~~~~~~~~~~~~~~~~~~~--repl="{\textbackslash}2-{\textbackslash}1-{\textbackslash}3"~*.txt}
\end{flushleft}\end{ttfamily}
\end{quote}

In order to emphasize the portability, I have used a generic 
\texttt{{\$}>} promtp, meaning that these examples work equally well on
both Unix and Windows (of course on Unix I could do the same 
job with sed or awk, but these tools are not as scalable as
a Python script).

The double slash syntax has the advantage of being
quite clear, but the disadvantage of being quite verbose, and it is
handier to use abbreviations for the name of the options. For instance, 
sensible abbreviations can be \texttt{-x} for \texttt{--regx}, \texttt{-r} for \texttt{--repl} 
and \texttt{-n} for \texttt{--nobackup}; moreover, the \texttt{=} sign can safely be
removed. Then the previous command reads
\begin{quote}
\begin{ttfamily}\begin{flushleft}
\mbox{{\$}>~replace.py~-n~-x"({\textbackslash}dd)-({\textbackslash}dd)-({\textbackslash}d{\textbackslash}d{\textbackslash}d{\textbackslash}d)"~-r"{\textbackslash}2-{\textbackslash}1-{\textbackslash}3"~*.txt}
\end{flushleft}\end{ttfamily}
\end{quote}

You see here the Unix convention at work: one-letter options
(a.k.a. short options) are prefixed with a single dash, whereas 
long options are prefixed with a double dash. The advantage of the 
convention is that short options can be composed: for instance
\begin{quote}
\begin{ttfamily}\begin{flushleft}
\mbox{{\$}>~replace.py~-nx~"({\textbackslash}dd)-({\textbackslash}dd)-({\textbackslash}d{\textbackslash}d{\textbackslash}d{\textbackslash}d)"~-r~"{\textbackslash}2-{\textbackslash}1-{\textbackslash}3"~*.txt}
\end{flushleft}\end{ttfamily}
\end{quote}

means the same as the previous line, i.e. \texttt{-nx} is parsed as
\texttt{-n -x}.  You can also freely exchange the order of the options
(provided option arguments are kept right 
after their respective options):
\begin{quote}
\begin{ttfamily}\begin{flushleft}
\mbox{{\$}>~replace.py~-nr~"{\textbackslash}2-{\textbackslash}1-{\textbackslash}3"~*.txt~-x~"({\textbackslash}dd)-({\textbackslash}dd)-({\textbackslash}d{\textbackslash}d{\textbackslash}d{\textbackslash}d)"}
\end{flushleft}\end{ttfamily}
\end{quote}

This command will be parsed exactly as before, since options and option 
arguments are not positional.


%___________________________________________________________________________

\hypertarget{how-does-it-work-in-practice}{}
\pdfbookmark[0]{How does it work in practice?}{how-does-it-work-in-practice}
\section*{How does it work in practice?}

Having stated the requirements, we may start implementing our 
search and replace tool. The first step, and the most important
one, is to write down the documentation string, even if you 
will have to wait until the last section to understand
why the docstring is the most important part of this script ;)
\begin{quote}
\begin{ttfamily}\begin{flushleft}
\mbox{{\#}!/usr/bin/env~python}\\
\mbox{"""}\\
\mbox{Given~a~sequence~of~text~files,~replaces~everywhere}\\
\mbox{a~regular~expression~x~with~a~replacement~string~s.}\\
\mbox{}\\
\mbox{~~usage:~{\%}prog~files~[options]}\\
\mbox{~~-x,~--regx=REGX:~regular~expression}\\
\mbox{~~-r,~--repl=REPL:~replacement~string}\\
\mbox{~~-n,~--nobackup:~do~not~make~backup~copies}\\
\mbox{"""}
\end{flushleft}\end{ttfamily}
\end{quote}

For the sake of my Windows readers here I notice
that the first line is not needed if you work on
Windows only, where it is just a comment, but in
the Unix world it is important since it allows the
shell to recognize the script as a python script.
So, it is a good habit to use it, it is harmless in Windows
and helpful in Unix.

The next step is to write down a simple search and replace routine:
\begin{quote}
\begin{ttfamily}\begin{flushleft}
\mbox{import~optparse,~re}\\
\mbox{}\\
\mbox{def~replace(regx,repl,files,backup{\_}option=True):}\\
\mbox{~~~~rx=re.compile(regx)}\\
\mbox{~~~~for~fname~in~files:}\\
\mbox{~~~~~~~~txt=file(fname,"U").read()}\\
\mbox{~~~~~~~~if~backup{\_}option:}\\
\mbox{~~~~~~~~~~~~print~>>~file(fname+".bak",~"w"),~txt,}\\
\mbox{~~~~~~~~print~>>~file(fname,"w"),~rx.sub(repl,txt),}
\end{flushleft}\end{ttfamily}
\end{quote}

This replace routine is entirely unsurprising, the only thing you
may notice is the usage of the ``U'' option in the line
\begin{quote}
\begin{ttfamily}\begin{flushleft}
\mbox{txt=file(fname,"U").read()}
\end{flushleft}\end{ttfamily}
\end{quote}

This is a new feature of Python 2.3. Text files open with the ``U''
option are read in ``Universal'' mode: this means that Python takes
care for you of the newline pain, i.e. this script will work 
correctly everywhere, independently by the newline
conventions of your operating system. The script works by reading 
the whole file in memory: this is bad practice, and here I am assuming 
that you will use this script only on short files that will fit in 
your memory, otherwise you should ``massage'' a bit the code.
Also, a fully fledget script would check if the file exists 
and can be read, and would do something in the case it is not,
but I think you will forbid me for skipping on these points,
since the thing I am really interested in is the \texttt{optparse}
module that, as I am sure you noticed, I have already imported 
at the top.

So, how does it work? It is quite simple, really. 
First you need to instantiate an argument line parser from
the \texttt{OptionParser} class provided by \texttt{optparse}:
\begin{quote}
\begin{ttfamily}\begin{flushleft}
\mbox{parser~=~optparse.OptionParser("usage:~{\%}prog~files~[options]")}
\end{flushleft}\end{ttfamily}
\end{quote}

The string \texttt{"usage: {\%}prog files [options]"} will be used to
print a customized usage message,  where \texttt{{\%}prog} will be replaced
by the name of the script - in this case - \texttt{replace.py}. You
may safely omit it and \texttt{optparse} will use a default 
\texttt{"usage: {\%}prog [options]"} string.

Then, you tell the parser informations about which options
it must recognize:
\begin{quote}
\begin{ttfamily}\begin{flushleft}
\mbox{parser.add{\_}option("-x",~"--regx",}\\
\mbox{~~~~~~~~~~~~~~~~help="regular~expression")}\\
\mbox{parser.add{\_}option("-r",~"--repl",}\\
\mbox{~~~~~~~~~~~~~~~~help="replacement~string")}\\
\mbox{parser.add{\_}option("-n",~"--nobackup",}\\
\mbox{~~~~~~~~~~~~~~~~action="store{\_}true",}\\
\mbox{~~~~~~~~~~~~~~~~help="do~not~make~backup~copies")}
\end{flushleft}\end{ttfamily}
\end{quote}

The \texttt{help} keyword argument is intended to document the
intent of the given option; it is also used by \texttt{optparse} in the 
usage message. The \texttt{action=store{\_}true} keyword argument is
used to distinguish flags from options with arguments, it tells
\texttt{optparse} to set the flag \texttt{nobackup} to \texttt{True} if \texttt{-n}
or \texttt{--nobackup} is given in the command line.

Finally, you tell the parse to do its job and to parse the command line:
\begin{quote}
\begin{ttfamily}\begin{flushleft}
\mbox{option,~files~=~parser.parse{\_}args()}
\end{flushleft}\end{ttfamily}
\end{quote}

The \texttt{.parse{\_}args()} method returns two values: an object \texttt{option}, 
which is an instance of the \texttt{optparse.Option} class, and a list
of positional arguments
The \texttt{option} object has attributes - called \emph{destionations} in 
\texttt{optparse} terminology - corresponding to the given options.
In our example, \texttt{option} will have the attributes \texttt{option.regx}, 
\texttt{option.repl} and  \texttt{option.nobackup}.

If no options are passed to the command line, all these attributes
are initialized to \texttt{None}, otherwise they are initialized to 
the argument option. In particular flag options are initialized to 
\texttt{True} if they are given, to``None`` otherwise. So, in our example 
\texttt{option.nobackup} is \texttt{True} if the flag \texttt{-n} or \texttt{--nobackup} 
is given.
The list \texttt{files} in our example contains the files passed
to the command line that must be replaced (assuming you passed
the names of accessible files in your system).

At this point, we may dispatch to the right routine according to
the given command line with a simple \texttt{if} statement such as the
following:
\begin{quote}
\begin{ttfamily}\begin{flushleft}
\mbox{if~not~files:}\\
\mbox{~~~~print~"No~files~given!"}\\
\mbox{elif~option.regx~and~option.repl:}\\
\mbox{~~~~replace(option.regex,~option.repl,~files,~not~option.nobackup)}\\
\mbox{else:}\\
\mbox{~~~~print~"Missing~options~or~unrecognized~options."}\\
\mbox{~~~~print~{\_}{\_}doc{\_}{\_}~{\#}~documentation~on~how~to~use~the~script}
\end{flushleft}\end{ttfamily}
\end{quote}

A nice feature of \texttt{optparse} is that an help option is automatically
created, so \texttt{replace.py -h} (or  \texttt{replace.py --help}) will work as
you may expect:
\begin{quote}
\begin{ttfamily}\begin{flushleft}
\mbox{{\$}>~replace.py~--help}\\
\mbox{usage:~replace.py~files~[options]}\\
\mbox{}\\
\mbox{}\\
\mbox{options:}\\
\mbox{~~-h,~--help~~~~~~~~~~~show~this~help~message~and~exit}\\
\mbox{~~-xREGX,~--regx=REGX~~regular~expression}\\
\mbox{~~-rREPL,~--repl=REPL~~replacement~string}\\
\mbox{~~-n,~--nobackup~~~~~~~do~not~make~backup~copies}
\end{flushleft}\end{ttfamily}
\end{quote}

You may programmatically print the usage message by invoking  
\texttt{parser.print{\_}help()}.

At this point you may test your script and see that it works as
advertised.


%___________________________________________________________________________

\hypertarget{how-to-reduce-verbosity-and-make-your-life-with-optparse-happier}{}
\pdfbookmark[0]{How to reduce verbosity and make your life with optparse happier}{how-to-reduce-verbosity-and-make-your-life-with-optparse-happier}
\section*{How to reduce verbosity and make your life with \texttt{optparse} happier}

The approach we followed in the previous example has a disadvantage: 
it involves a certain amount of verbosity/redundance. Suppose for instance 
we want to add another option to the script, for instance the ability to 
restore the original file from the backup copy, which is quite handy in 
case something went wrong with the replace.
Then, we have to change the script in three points: in the docstring,
in the \texttt{add{\_}option} list, and in the \texttt{if .. elif .. else ...} 
statement. At least one of this is redundant.
One would be tempted to think that the information in the documentation
string is redundant, since it is already magically provided in the help
options: however, I will take the opposite view, that the information
in the help options is redundant, since it is already contained in
the docstring.  It is a kind of sacrilege to write a Python script 
without a docstring explaining what it does, so the docstring cannot 
be removed or shortened, since it must be available to automatic
documentation tools such as pydoc. So, the idea is to 
extract information from the docstring, and to avoid altogether 
the boring task of writing by hand the \texttt{parser.add{\_}option} lines. 
I implemented this idea in a cookbook recipe, by writing an 
\texttt{optionparse} module which is just a thin wrapper around \texttt{optparse}.

Here I just show how the script (including the new restore option)
will look by using my \texttt{optionparse} wrapper:
\begin{ttfamily}\begin{flushleft}
\mbox{{\#}!/usr/bin/env~python}\\
\mbox{"""}\\
\mbox{Given~a~sequence~of~text~files,~replaces~everywhere}\\
\mbox{a~regular~expression~x~with~a~replacement~string~s.}\\
\mbox{}\\
\mbox{~~usage:~{\%}prog~files~[options]}\\
\mbox{~~-x,~--regx=REGX:~regular~expression}\\
\mbox{~~-r,~--repl=REPL:~replacement~string}\\
\mbox{~~-n,~--nobackup:~do~not~make~backup~copies}\\
\mbox{~~-R,~--restore:~restore~the~original~from~the~backup}\\
\mbox{"""}\\
\mbox{import~optionparse,~os,~shutil,~re}\\
\mbox{}\\
\mbox{def~replace(regx,repl,files,backup{\_}option=True):}\\
\mbox{~~~~rx=re.compile(regx)}\\
\mbox{~~~~for~fname~in~files:}\\
\mbox{~~~~~~~~{\#}~TODO:~add~a~test~to~see~if~the~file~exists~and~can~be~read}\\
\mbox{~~~~~~~~txt=file(fname,"U").read()}\\
\mbox{~~~~~~~~if~backup{\_}option:}\\
\mbox{~~~~~~~~~~~~print~>>~file(fname+".bak","w"),~txt}\\
\mbox{~~~~~~~~print~>>~file(fname,"w"),~rx.sub(repl,txt)}\\
\mbox{}\\
\mbox{def~restore(files):}\\
\mbox{~~~~for~fname~in~files:~~~~~~~}\\
\mbox{~~~~~~~~if~os.path.exists(fname+".bak"):}\\
\mbox{~~~~~~~~~~~~shutil.copyfile(fname+".bak",fname)}\\
\mbox{~~~~~~~~else:}\\
\mbox{~~~~~~~~~~~~print~"Sorry,~there~is~no~backup~copy~for~{\%}s"~{\%}~fname}\\
\mbox{}\\
\mbox{if~{\_}{\_}name{\_}{\_}=='{\_}{\_}main{\_}{\_}':}\\
\mbox{~~~~option,files=optionparse.parse({\_}{\_}doc{\_}{\_})}\\
\mbox{~~~~{\#}~optionparse.parse~parses~both~the~docstring~and~the~command~line!}\\
\mbox{~~~~if~not~files:}\\
\mbox{~~~~~~~~optionparse.exit()}\\
\mbox{~~~~elif~option.regx~and~option.repl:}\\
\mbox{~~~~~~~~replace(option.regex,~option.repl,~files,~not~option.nobackup)}\\
\mbox{~~~~elif~option.restore:}\\
\mbox{~~~~~~~~restore(files)}\\
\mbox{~~~~else:}\\
\mbox{~~~~~~~~print~"Missing~options~or~unrecognized~options."}
\end{flushleft}\end{ttfamily}

Working a bit more, one could also devise various tricks to avoid
the redundance in the \texttt{if} statement (for instance using a
dictionary of functions and dispatching according to the name of
the given option). However this simple recipe is good enough to
provide a minimal wrapper to \texttt{optparse} which requires a minimum effort
and works well for the most common case. For instance, the paper you are 
reading now has been written by using \texttt{optionparse}: I used it to 
write a simple wrapper to docutils - the standard
Python tools to convert text files to HTML pages - to customize 
its behavior to my needs. It is also nicer to notice that internally
docutils itself uses \texttt{optparse} to do its job, so actually this
paper has been composed by using \texttt{optparse} twice!

Finally, you should keep in mind that this article only scratch the
surface of \texttt{optparse}, which is quite sophisticated. 
For instance you can specify default values, different destinations, 
a \texttt{store{\_}false} action and much more, even if often you don't need 
all this power. Still, it is handy to have the power at your disposal when 
you need it.  So, the serious user of \texttt{optparse} is strongly 
encorauged to read the documentation in the standard library, which 
is pretty good and detailed. I will think that this article has fullfilled
its function of ``appetizer'' to \texttt{optparse}, if it has stimulate 
the reader to study more.


%___________________________________________________________________________

\hypertarget{references}{}
\pdfbookmark[0]{References}{references}
\section*{References}
\begin{itemize}
\item 
\texttt{optparse} is documented in the standard library

\item 
the \texttt{optionparse} module can be found \href{http://www.phyast.pitt.edu/~micheles/python/optionparse}{here}.

\item 
I wrote a Python Cookbook \href{http://aspn.activestate.com/ASPN/Cookbook/Python/Recipe/278844}{recipe} about optionparse.

\end{itemize}

\end{document}

